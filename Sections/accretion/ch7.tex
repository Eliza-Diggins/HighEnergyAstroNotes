
In this chapter, we will explore the \textbf{standard model of accretion disk physics}: the so-called \textbf{thin disk model}. This model forms the cornerstone of modern accretion theory and underlies our understanding of systems ranging from protostellar disks to luminous quasars. Despite its simplicity, the thin disk model is remarkably predictive, yielding quantitative relations between disk structure, luminosity, and accretion rate that are broadly consistent with observations.
\par
Before we get into the details of the thin disk model, we'll want to first derive the equations of fluid dynamics in axisymmetric coordinates. Doing so will provide us the framework we need in order to model accretion disks properly.

\section{Fluid Dynamics of an Axisymmetric Rotating Disk}

We now derive the vertically integrated equations of \emph{mass}, \emph{momentum}, and \emph{angular momentum} 
for an axisymmetric viscous disk, retaining an \emph{arbitrary} rotation law $\Omega(R)$. 
This will lead us to the \textbf{general viscous diffusion equation} governing the surface density $\Sigma(R,t)$---a result 
that can later be specialized to the familiar Keplerian case.
\par
We begin by assuming \textbf{axisymmetry}, so that all quantities are independent of the azimuthal coordinate $\phi$. 
The disk is characterized by its \textbf{surface density},
\begin{equation}
\label{eq:def_surface_density}
\Sigma(R,t) \;\equiv\; \int_{-\infty}^{\infty} \rho(R,z,t)\,dz,
\end{equation}
obtained by integrating the three-dimensional density $\rho$ over the vertical coordinate $z$. 
The velocity field of the gas is denoted ${\bf u} = (v_r, v_\phi, v_z)$, with the azimuthal component dominated by rotation:
\[
v_\phi(R,t) = R\,\Omega(R,t),
\]
where $\Omega(R,t)$ is the \textbf{angular velocity field} describing the disk’s differential rotation.
\subsection{Viscosity}
As was discussed in the previous chapter, there are \textbf{viscous effects} which play a role in the theory of accretion disks. As such, we will need to consider them in our fluid dynamics.
\par
We will, for the sake of generality, consider some \textbf{shear viscosity} which is
\[
\nu = \nu(R,t).
\]
In is typical to assign the \textbf{$\alpha$-prescription} here, but not necessary for the general discussion we are going to have preliminarily.
\par
As we proved in the previous chapter, the \textbf{viscous stress tensor} has the form
\[
\sigma_{r\phi} \;\equiv\; \rho\,\nu\,R\,\frac{d\Omega}{dR}.
\]
\rmk{Remember, this is the \textbf{force per area} exerted on a surface by viscous processes.}
\par
Now, for a particular region of area $dA = R\;dzd\phi$ at a distance $R$ from the center of the disk, the \textbf{torque} is
\[
\tau = R\sigma_{r\phi} dA = R^2\sigma_{r\phi} \;dz\;d\phi = \rho \nu R^3 \Omega'\;dz\;d\phi.
\]
The \textbf{total torque} exerted on a particular cylinder is then 
\begin{equation}
    \label{eq:G_def}
G = \int_{-\infty}^{\infty} \;dz \;\int_{0}^{2\pi}\; \rho \nu R^3 \Omega' \;d\phi = 2\pi \Sigma \nu R^3 \Omega'.
\end{equation}
We also recognize that have (by choosing $\tau \propto \sigma$ instead of $\tau \propto -\sigma$) that this is the torque \textbf{on the inside of the cylinder} from the \textbf{outside of the cylinder.} As such, we $\Omega' < 0$, the inside is \textbf{losing angular momentum}, which is as expected. We are now ready to start performing the fluid dynamics in earnest.

\subsection*{The Continuity Equation}

The most general form of the \textbf{continuity equation} requires that
\[
\frac{\partial \rho}{\partial t} + \nabla \cdot (\rho {\bf u}) = 0.
\]
Because of our constraints on the symmetry of the system, we have that
\[
\frac{\partial \rho}{\partial t} + \frac{1}{R}\frac{\partial}{\partial R}\left(R\rho v_r\right) + \frac{\partial}{\partial z} \left(\rho v_z\right) = 0
\]
Integrating over the vertical extent of the disk, we have
\[
\frac{\partial \Sigma}{\partial t} + \frac{1}{R}\frac{\partial}{\partial R}\left(R\Sigma v_r\right) + \underbrace{\int_{-\infty}^\infty \rho v_z\;dz}_{0 \;\text{because}\;\lim_{z\to\infty} \rho = 0}= 0
\]
Thus, we arrive at the first of our \textbf{critical equations}:
\begin{equation}
\boxed{\;
\frac{\partial \Sigma}{\partial t}
+ \frac{1}{R}\,\frac{\partial}{\partial R}\!\left(R\,\Sigma\,v_r\right) \;=\; 0.
\;}
\label{eq:continuity_disk_general}
\end{equation}
This is our \textbf{continuity equation} in axisymmetric coordinates.

\subsection*{Angular Momentum Conservation}

In order to discuss the \textbf{conservation of angular momentum}, let's keep track of the \textbf{surface angular momentum density}, which we will denote as $\ell$ and write
\[
\ell = \Sigma R^2 \Omega.
\]
Angular momentum in a viscous disk can be transported in two distinct ways:
\begin{enumerate}
    \item \textbf{Advection:} bulk motion of the gas carries angular momentum inward or outward;
    \item \textbf{Viscous stresses:} internal friction between adjacent annuli transports angular momentum through the disk.
\end{enumerate}
\vspace{10pt}
The first transfer modality we need to pay attention to is \textbf{advection}, which can occur across a cylindrical laminae of the disk. Given a \textbf{radial drift} velocity $v_r$, a linear flux density (mass per unit length) of mass passes through the boundary carrying $\ell$, which means that
\[
f_\ell = \Sigma R^2 \Omega v_r
\]
is the \textbf{linear angular momentum flux density}. 
\par
We can also \textbf{transfer viscously}. For a laminae of the disk of radius $\delta R$, there is an inner and outer torque and the net-torque is 
\[
G_{\rm net} = G(R+\delta R) - G(R) = \frac{dG}{dR} \delta R.
\]
the resulting torque \textbf{per unit area} (angular momentum loss / gain per unit area) is
\[
\frac{(dG/dR)\delta R}{2\pi R \delta R} = \frac{1}{2\pi R} \frac{dG}{dR}.
\]
As such, if we tie these together into a conservation law,
\begin{equation}
\boxed{
\frac{\partial}{\partial t}\!\left(\Sigma R^2 \Omega\right)
+ \frac{1}{R}\frac{\partial}{\partial R}\!\left(R\,\Sigma v_r\,R^2 \Omega\right)
= \frac{1}{2\pi R}\frac{\partial G}{\partial R}.
}
\label{eq:angmom_cons_disk}
\end{equation}
The right-hand side represents the difference in viscous torque between neighboring annuli, i.e.\ the net angular momentum injected or removed by shear.
\rmk{We're getting the $2\pi R$ term because we had to integrate over a surface.}
Substituting equation~\eqref{eq:G_def} into the conservation law~\eqref{eq:angmom_cons_disk} gives the standard vertically integrated angular–momentum equation for a viscous, axisymmetric disk:
\begin{equation}
\boxed{
\frac{\partial}{\partial t}\!\left(\Sigma R^2\Omega\right)
+ \frac{1}{R}\frac{\partial}{\partial R}\!\left(R\,\Sigma\,v_r\,R^2\Omega\right)
= \frac{1}{R}\frac{\partial}{\partial R}
   \!\left(\nu\,\Sigma\,R^3\,\frac{d\Omega}{dR}\right).
}
\label{eq:angmom_cons_viscous}
\end{equation}
This equation states that the evolution of the angular momentum in a viscous disk results from the competition between advection by the radial flow (left-hand side) and viscous torques between adjacent rings (right-hand side).  
It provides the physical foundation for the thin–disk diffusion equation derived in the next section.


\subsection*{The Diffusion Equation of Disk Dynamics}
Comparing equations \eqref{eq:angmom_cons_disk} and \eqref{eq:continuity_disk_general}, one can eliminate $v_r$ in favor of the torque gradient. \rmk{This is a simple manipulation: $\Omega$ is a function of $R$ alone, so you simply clarify the time derivative and substitute continuity.} This leads directly to the well-known \textbf{diffusion equation for thin accretion disks}:
\begin{equation}
\boxed{\;
\frac{\partial \Sigma}{\partial t}
= -\,\frac{1}{R}\,\frac{\partial}{\partial R}
\left[
\frac{1}{\displaystyle \frac{d}{dR}\!\big(R^2\Omega\big)}
\;\frac{\partial}{\partial R}\!\left(\nu\,\Sigma\,R^3\,\frac{d\Omega}{dR}\right)
\right].
\;}
\label{eq:general_diffusion}
\end{equation}
Equation \eqref{eq:general_diffusion} is valid for \emph{any} axisymmetric viscous disk, independent of the specific rotation law.  
All microphysics of angular-momentum transport is encapsulated in the effective $\nu\Sigma$; 
all dynamics of orbital support enter through $\Omega(R,t)$.

\begin{remark}[Where pressure and gravity enter]
The rotation profile $\Omega(R,t)$ is set by the \emph{radial momentum} equation,
\[
\frac{v_\phi^2}{R} - \frac{1}{\rho}\frac{\partial P}{\partial R} \;=\; \frac{\partial \Phi}{\partial R},
\]
so that, if desired, pressure support can be retained via
$\Omega^2 = R^{-1}\partial_R\Phi + (R\rho)^{-1}\partial_R P$.
Equation \eqref{eq:general_diffusion} itself does not assume Keplerian rotation;  
Keplerianity (or any other choice) is imposed by specifying $\Omega(R)$ from this radial balance.
\end{remark}

For a Newtonian point mass, Keplerian motion implies $\Omega(R)=\Omega_K=(GM/R^3)^{1/2}$, hence
\[
\frac{d}{dR}\!\big(R^2\Omega_K\big) \;=\; \frac{1}{2}\,R\,\Omega_K,
\qquad
\frac{d\Omega_K}{dR} \;=\; -\frac{3}{2}\frac{\Omega_K}{R}.
\]
Substituting these into \eqref{eq:general_diffusion} gives the classic \textbf{\emph{Pringle diffusion equation}:}
\begin{equation}
\boxed{\;
\frac{\partial \Sigma}{\partial t}
= \frac{3}{R}\,\frac{\partial}{\partial R}
\left[
R^{1/2}\,\frac{\partial}{\partial R}\!\big(\nu\,\Sigma\,R^{1/2}\big)
\right].
\;}
\label{eq:Pringle}
\end{equation}

\subsection*{The Drift Velocity}
\par
In deriving the diffusion equation \eqref{eq:general_diffusion}, we eliminated the radial velocity $v_r$ in favor of the torque gradient. While this was convenient for expressing the time evolution of $\Sigma$, it is often useful to recover an explicit expression for $v_r$ itself. The radial drift velocity determines the direction and rate of mass accretion in the disk, and it connects the diffusive picture of angular-momentum transport to the physical inflow of matter.
\par
We start from the vertically integrated \textbf{continuity equation}:
\begin{equation}
\label{eq:cont_eqn}
\frac{\partial \Sigma}{\partial t} 
+ \frac{1}{R}\frac{\partial}{\partial R}\!\left(R\,\Sigma\,v_r\right) = 0.
\end{equation}
For a Keplerian disk, the \textbf{Pringle diffusion equation} \eqref{eq:Pringle} provides $\partial_t \Sigma$ as
\begin{equation}
\label{eq:pringle_diffusion}
\frac{\partial \Sigma}{\partial t}
= \frac{3}{R}\frac{\partial}{\partial R}
\left[
R^{1/2}\,\frac{\partial}{\partial R}\big(\nu\,\Sigma\,R^{1/2}\big)
\right].
\end{equation}
Substituting equation~\eqref{eq:pringle_diffusion} into \eqref{eq:cont_eqn} gives
\begin{equation}
\frac{1}{R}\frac{\partial}{\partial R}\!\left(R\,\Sigma\,v_r\right)
= -\,\frac{3}{R}\frac{\partial}{\partial R}
\left[
R^{1/2}\,\frac{\partial}{\partial R}\big(\nu\,\Sigma\,R^{1/2}\big)
\right].
\end{equation}
Multiplying through by $R$ and integrating with respect to $R$ yields
\begin{equation}
R\,\Sigma\,v_r
= -\,3\,R^{1/2}\,\frac{\partial}{\partial R}\big(\nu\,\Sigma\,R^{1/2}\big)
+ C,
\end{equation}
where $C$ is an integration constant determined by boundary conditions.
If the disk experiences no mass inflow from large radii (i.e.\ $v_r \to 0$ as $R \to \infty$),
then $C=0$. We thus obtain the standard expression for the radial drift velocity:
\begin{equation}
\label{eq:vr_solution}
\boxed{
v_r(R,t)
= -\,\frac{3}{\Sigma\,R^{1/2}}
\frac{\partial}{\partial R}\!\left(\nu\,\Sigma\,R^{1/2}\right).
}
\end{equation}

\subsection*{Solutions for Constant Viscosity}

To gain further intuition, let us now consider the case where the viscosity is taken to be a constant, $\nu=\text{const}$. In this case, the \textbf{Pringle diffusion} equation 
\begin{equation}
    \frac{\partial \Sigma}{\partial t} = 
    \frac{3\nu}{R}\frac{\partial}{\partial R}
    \left[ R^{1/2}\frac{\partial}{\partial R}\big(\Sigma R^{1/2}\big)\right]
\end{equation}
simplifies considerably. We first introduce the variable
\[
x \equiv R^{1/2}, \qquad R = x^2,
\]
so that the equation is \textbf{expressed in a more standard diffusive form}. We define a new dependent variable
\[
u(x,t) \equiv \Sigma(R,t)\, x,
\]
such that the evolution equation becomes
\begin{equation}
    \frac{\partial u}{\partial t} = \frac{12\nu}{x^2}\frac{\partial^2 u}{\partial x^2}.
\end{equation}
This form still has explicit $x$-dependence, but the choice of variables sets us up to see the equation as a diffusion-type operator. \rmk{This is quite an obvious picture: we don't like the $R^{1/2}$, so we replace with $x$. We don't like multiple terms in the derivatives: let $u = \Sigma x$, then everything is nice and standardized.}

\vspace{0.25cm}
\noindent
To proceed analytically, we assume solutions of the form
\[
u(x,t) = X(x)\,T(t).
\]
\rmk{This is a classic Cauchy-problem for a Sturm-Liouville problem. We can just proceed with separation.} Substituting into the PDE gives
\[
\frac{1}{T}\frac{dT}{dt} = \frac{12\nu}{x^2}\,\frac{1}{X}\frac{d^2X}{dx^2} = -\lambda,
\]
where $\lambda$ is a separation constant. Thus we obtain
\begin{align}
    \frac{dT}{dt} &= -\lambda T, \\
    \frac{d^2X}{dx^2} + \frac{\lambda}{12\nu}x^2 X &= 0.
\end{align}
The temporal part integrates immediately to $T(t) = e^{-\lambda t}$. The spatial equation is a Sturm--Liouville problem: \textbf{it resembles a Schrödinger-type equation with a quadratic potential.} Its solutions are linear combinations of functions related to parabolic cylinder functions.

\vspace{0.25cm}
\noindent
\textbf{Green's function approach.} \;
While separation of variables gives a formal solution in terms of special functions, the more practical approach used in the literature (Lynden-Bell \& Pringle 1974) is to construct a Green's function for the diffusion operator. The Green's function $G(R,R';t)$ is defined such that
\[
\Sigma(R,t) = \int_0^\infty G(R,R';t)\,\Sigma(R',0)\, R'\,dR',
\]
where $G(R,R';t)$ represents the surface density response at $R$ and time $t$ due to an initial $\delta$-function ring at $R'$. For constant viscosity, this Green's function can be computed explicitly and has the form of a broadened Gaussian in the $R^{1/2}$ variable.
\par
For $\nu=\text{const}$, one finds that the Green's function can be written in terms of modified Bessel functions of the first kind, $I_\nu(x)$. The result is
\begin{equation}
    G(R,R';t) = \frac{1}{4\pi \nu t}\,
    \left(\frac{R}{R'}\right)^{1/4}
    \exp\!\left[-\frac{R+R'}{4\nu t}\right]\,
    I_{1/4}\!\left(\frac{\sqrt{RR'}}{2\nu t}\right).
\end{equation}
This function satisfies the normalization condition
\[
\int_0^\infty G(R,R';t)\,R\,dR = 1,
\]
which ensures mass conservation: the total mass in the disk remains constant (unless boundary conditions at the inner radius allow accretion onto the central object).
\par
\textbf{What does this tell us?}
Here's the big takeaway, we see that the Green's Function relies on a characteristic time scale: the \textbf{viscous time scale} 
\begin{equation}
    \label{eq:viscous_timescale}
    \boxed{
    t_{\rm visc} \sim \frac{R^2}{\nu} \sim \frac{R}{v_R},
    }
\end{equation}
on which changes in the disc occur. Similarly, if the disk has spatial gradients on the scale of some length $\ell$, then the resulting time scale of evolution is
\[
t_{\rm visc} \sim \frac{\ell^2}{\nu} \sim \frac{\ell}{v_R},
\]
which means that \textbf{shorter / sharper} features in the accretion disk will \textbf{decay more quickly} than those which are less pronounced. This serves to \textbf{smooth out} the disk. We have now accomplished all of the \textbf{general} fluid dynamics we will need for the thin disk model. We can now discuss the core physics of the model!

\begin{bigidea}
    We can actually get $t_{\rm visc}$ much more heuristically. Note that the Navier-Stokes equation in the incompressible limit looks like
    \[
    \partial_t {\bf u} + \ldots = \ldots + \nu \nabla^2{\bf u},
    \]
    which means that for a characteristic length $\ell$ and time $t_{\rm visc}$,
    \[
    \frac{{\bf u}}{t_{\rm visc}} \sim \nu {\bf u}/\ell^2 \implies t_{\rm visc} = \frac{\ell^2}{\nu}.
    \]
\end{bigidea}

\newpage

\section{Overview of Thin Disk Accretion}

We are now ready to explore the \textbf{structure of the thin accretion disk model}.  Because the derivation involves many interdependent quantities, it is easy to lose sight of the physical picture.  Before diving into the detailed equations, we will first summarize the key variables and the logical structure of the model.

\subsection{The Model Structure}

The \textbf{thin accretion disk} is described by a set of coupled dynamical and thermodynamic variables.  Each depends on the radial coordinate $R$ and is connected to the others through physical relations \textbf{such as hydrostatic balance, angular momentum conservation, and radiative diffusion}. The primary variables are listed below:

\begin{table}[ht!]
\centering
\caption{Primary dynamical and thermodynamic variables of the thin accretion disk.}
\begin{tabular}{p{3cm}|p{1.5cm}|p{9cm}}
\hline
\textbf{Quantity} & \textbf{Symbol} & \textbf{Notes} \\
\hline
Density & $\rho(R)$ &
Midplane mass density of the disk gas.  Related to surface density through $\Sigma = 2\rho H$. \\[4pt]
Surface Density & $\Sigma(R)$ &
Vertically integrated density; couples to $\rho$ and $H$ through hydrostatic equilibrium. \\[4pt]
Scale Height & $H(R)$ &
Vertical pressure scale height, given by $H = c_s / \Omega_K$, where $c_s$ is the sound speed. \\[4pt]
Temperature & $T(R)$ &
Midplane temperature; sets the sound speed $c_s$, pressure, and effective viscosity. \\[4pt]
Radial Velocity & $v_r(R)$ &
Mean inflow velocity determined by mass conservation:
$\dot{M} = 2\pi R \Sigma |v_r|$. \\[4pt]
Viscosity & $\nu(R)$ &
Effective (turbulent) kinematic viscosity.  Commonly parameterized by $\nu = \alpha c_s H$, though any physically motivated prescription $\nu(\rho, T, R)$ may be used. \\[4pt]
Optical Depth & $\tau(R)$ &
Vertical optical depth, $\tau = \tfrac{1}{2}\kappa(\rho,T)\Sigma$.  
Not an independent quantity; determined from the opacity law $\kappa(\rho,T)$ and the surface density. \\[4pt]
\hline
\end{tabular}
\label{tab:disk_variables}
\end{table}

\par
To close the system, several additional \textbf{input prescriptions} are required:
\vspace{10pt}
\begin{itemize}
    \item the \textbf{viscosity law} $\nu(\rho, T, R)$, such as the $\alpha$-disk model $\nu = \alpha c_s H$;
    \item the \textbf{opacity law} $\kappa(\rho, T)$, which specifies the radiative transport regime (e.g.\ Kramers opacity or electron scattering);
    \item and the \textbf{global parameters} --- the accretion rate $\dot{M}$ and central mass $M$.
\end{itemize}
\vspace{10pt}
With these specified, the coupled equations of mass conservation, angular momentum conservation, hydrostatic balance, and energy balance can be solved self-consistently for all disk quantities: $\rho(R)$, $\Sigma(R)$, $T(R)$, $H(R)$, $\nu(R)$, $v_r(R)$, and $\tau(R)$.  From these, one can derive secondary observables such as the midplane pressure, emergent flux, and effective temperature.
\par
In what follows, we will build up the model by connecting three key aspects:
\begin{enumerate}
    \item the \textbf{density structure} of the disk, governed by hydrostatic equilibrium;
    \item the \textbf{local thermodynamics}, relating pressure, temperature, and sound speed;
    \item and the \textbf{energy balance}, linking viscous dissipation and radiative cooling.
\end{enumerate}
Together, these relations define the full self-consistent solution of the thin accretion disk.
\par
Before we proceed to the derivation, we will want to also address various assumptions that go into the model.

\subsection{Structural Assumptions}

\begin{table}[ht!]
\centering
\renewcommand{\arraystretch}{1.4}
\begin{tabular}{p{0.28\linewidth} p{0.65\linewidth}}
\toprule
\textbf{Assumption} & \textbf{Consequence / Description} \\
\midrule
\textbf{Axisymmetry} & All quantities are independent of azimuth: $\partial/\partial\phi = 0$.  The disk is fully described by $(R,z,t)$. \\[4pt]
\textbf{Geometric Thinness} & The vertical scale height is small compared to the radius: $H/R \ll 1$.  Enables vertical integration and Taylor expansion of the gravitational potential. \\[4pt]
\textbf{Nearly Circular Orbits} & The flow is dominated by rotation, $v_\phi \gg v_r, v_z$.  To leading order, the centrifugal and gravitational forces balance: $\Omega \simeq \Omega_K$. \\[4pt]
\textbf{Vertical Hydrostatic Equilibrium} & The vertical velocity is negligible ($v_z \approx 0$), and $dP/dz = -\rho \Omega_K^2 z$.  This defines the Gaussian vertical density profile and the scale height $H=c_s/\Omega_K$. \\[4pt]
\textbf{Neglect of Self-Gravity} & The disk’s own gravity is small compared to the central potential: $\Phi \simeq -GM/\sqrt{R^2+z^2}$. \\[4pt]
 {\bf Steady State}&We assume that dynamical changes in the conditions of the accretion source happen on time scales longer than those of the viscous time scales determining the structure of the disk. This permits a steady state assumption.\\
\bottomrule
\end{tabular}
\caption{Structural assumptions defining the thin disk geometry and kinematics.}
\end{table}

These are the foundational simplifications that define what we mean by a ``thin'' accretion disk. The most obvious of these is the condition of \textbf{axisymmetry}, which is very sensible. The most \textbf{important} is the concept of the disk as a \textbf{geometrically thin system}. As we will see later, this allows us to treat the problem as separable, which would not be possible in a more general scenario. We also make a number of other reasonable assumptions about the nature of the flow fields in order to keep things tractable as summarized in the following table:

While these assumptions are reasonable in many situations, there are notable situations where we cannot blindly assume that they will be sufficient. The statement of \textbf{vertical hydrostatic equilibrium} is reliant on the argument that the sound crossing time $\tau_{z} = H/c_s \ll \tau_{\rm visc}$, so the vertical structure reacts instantaneously to changes in the disk structure. This fails for \textbf{wind launching} in disks and in \textbf{warped disks}, both of which can occur.
\par
Likewise, there are some rare scenarios where self-gravity becomes relevant: particularly in circumstellar disks and in massive AGN disks.


\subsection{Physical Prescriptions}

\begin{table}[ht!]
\centering
\renewcommand{\arraystretch}{1.4}
\begin{tabular}{p{0.28\linewidth} p{0.65\linewidth}}
\toprule
\textbf{Prescription} & \textbf{Description / Consequence} \\
\midrule
\textbf{Equation of State} & The gas is barotropic or {\bf locally isothermal}: $P = \rho c_s^2$.  Pressure depends only on density through the local sound speed. \\[4pt]
\textbf{Viscous Stress} & Angular momentum transport is parameterized by a kinematic viscosity $\nu$ through the stress tensor component $T_{r\phi} = -\nu \Sigma R\,d\Omega/dR$. Really, we're assuming this is the only relevant set of stresses. This includes assuming pressure is subdominant.\\[4pt]
\textbf{Alpha Prescription} & Turbulent viscosity is modeled as $\nu = \alpha c_s H$, where $0 < \alpha < 1$.  The dimensionless parameter $\alpha$ captures the efficiency of angular momentum transport. \\[4pt]
\textbf{Local Energy Balance} & Viscous heating is locally balanced by radiative cooling: $D(R) = \sigma T_{\rm eff}^4$. This means that {\bf all} of the energy is radiated away on a timescale shorter than the viscous timescale. \\[4pt]
\textbf{Optically Thick Emission} & Radiation escapes by vertical diffusion, and each annulus radiates approximately as a blackbody with effective temperature $T_{\rm eff}(R)$. \\[4pt]
\bottomrule
\end{tabular}
\caption{Physical prescriptions specifying viscosity, pressure, and thermal behavior in the thin disk.}
\end{table}


On top of the core assumptions that we have stipulated above, we make a number of prescriptions about the physics. These assumptions \textbf{close the system of equations} by prescribing how pressure, viscosity, and energy transport behave.

\subsection{Consistency Conditions}

If the above assumptions hold, several scaling relations and inequalities follow naturally.  
These relations are not separate assumptions, but \emph{self-consistency checks} that must be satisfied within the thin disk regime.

\begin{table}[h!]
\centering
\renewcommand{\arraystretch}{1.4}
\begin{tabular}{p{0.35\linewidth} p{0.58\linewidth}}
\toprule
\textbf{Relation} & \textbf{Interpretation} \\
\midrule
$H/R = c_s / v_\phi \ll 1$ & The disk is geometrically thin. \\[4pt]
$v_r \sim \alpha (H/R)^2 v_\phi \ll c_s$ & Inflow is slow and subsonic. \\[4pt]
$\Omega \simeq \Omega_K = \sqrt{GM/R^3}$ & Rotation is Keplerian to leading order. \\[4pt]
$(1/\rho)\,dP/dR \ll GM/R^2$ & Radial pressure forces are negligible. \\[4pt]
$t_{\rm visc} \sim R^2/\nu \gg t_{\rm dyn} \sim 1/\Omega_K$ & Viscous evolution occurs on timescales much longer than orbital motion. \\[4pt]
\bottomrule
\end{tabular}
\caption{Consistency relations that characterize the thin disk regime.}
\end{table}

\bigskip
Together, these assumptions and prescriptions form the foundation of the thin disk model.  
They allow the full hydrodynamic problem to be reduced to a set of vertically integrated equations governing the surface density, torque, and energy dissipation of the disk— the so-called \emph{canonical thin-disk equations}, which we now derive.

\section{The Structure of Thin Disks}

Let's imagine that the external conditions on the accretion disk depend on time scales which are long compared to the viscous time scale. In such a case (\rmk{which is generally valid}), the disk will have a sufficient amount of time to settle into a steady state before dynamical changes can modify the behavior again. Thus, we settle into a \textbf{steady state solution}. In this section, we investigate this solution.

\subsection{The Accretion Rate}

From the continuity equation \eqref{eq:cont_eqn},
\[
\underbrace{\frac{\partial \Sigma}{\partial t} }_{=0} + \frac{1}{R} \partial_R(R \Sigma v_R) = 0 \implies R\Sigma v_R = {\rm Constant}.
\]
Just as we saw in the \textbf{Bondi accretion} derivation, this corresponds to constant mass flux across disks in order to maintain the steady state. Thus, we can immediately obtain the accretion rate equation:
\begin{equation}
    \label{eq:disc_acc_rate}
    \boxed{
    \dot{M} = 2\pi R \Sigma (-v_R).
    }
\end{equation}
Already, we have achieved a very \textbf{powerful statement about accretion rates}. This has the same benefits that it did in the Bondi scenario: we were able to self-consistently understand the constant rate of accretion in terms of the external parameters. In this case, specifying $\dot{M}$ fixes many of the internal parameters of the model! \rmk{This is the first of \textbf{many} of the constitutive equations that we will rely on to close the model parameters.}

\subsection{Boundary Conditions}

So far we have characterized the thin disk using the continuity equation, but we have not yet made any specifications for the model at the boundary.\textbf{ In order to do this, we use the angular momentum conservation equation with no time dependence.} This takes the form
\begin{equation}
\label{eq:disk_steady_integral}
    R^3 \Sigma v_R \Omega = \frac{G}{2\pi} + \frac{C}{2\pi}.
\end{equation}
As such, we see that the steady state solution is really quite simple to arrive at. Let's now determine how one constrains the value of the integration factor $C$. If we remember, 
\[
R^3 \Sigma v_r \Omega = R f_\ell = \frac{1}{2\pi}(G(R)+C).
\]
So really, $C$ is going to be constrained by the \textbf{momentum flux behavior at the boundary.} It is worth discussing in more detail how the boundary condition constrains the steady--state solution, and in particular how the integration constant $C$ is determined.
\par
At large radii, the disk rotation is nearly Keplerian and the torque $G(R)$ is the sole mechanism redistributing angular momentum. However, the situation is different at the \textbf{inner edge} of the disk, near $R_{\rm in}$, where the disk must connect to the central object. In this region, \textbf{we cannot assume perfect Keplerian rotation}: the material must transition from orbital motion in the disk to either plunging motion (for black holes) or to corotation with the stellar surface (for neutron stars or white dwarfs). This transition region is called the \textbf{boundary layer}. 

\paragraph{Surface-Free Boundary}
For systems like \textbf{black holes}, there is no viscous stress at the \textbf{inner-most stable orbit} (ISCO). As such, the torque $G(R)$ must disappear at the inner boundary. In that case, equation~\eqref{eq:disk_steady_integral} at the ISCO radius $R_I$ is
\[
R_{I}^3 \Sigma_I v_{r,I} \Omega_{I} = \frac{C}{2\pi}.
\]
We know that the accretion rate is precisely, see equation~\eqref{eq:disc_acc_rate},
\[
\dot{M} = 2\pi R \Sigma (-v_R) \implies C = - \dot{M} R_I^2 \Omega
\]
As such, substitution back into equation~\eqref{eq:disk_steady_integral}, we find
\[
R^3 \Sigma v_R \Omega = \frac{1}{2\pi}\left[G(R) - \dot{M} R_I^2 \Omega_I\right]
\]
We know $G(R)$ (equation~\ref{eq:disc_torque}) takes the form
\[
G(R) = -2\pi R^2 \nu \Sigma \frac{\partial \Omega}{\partial R},
\]
so
\[
2\pi R^3 \Omega v_R \Sigma = -\dot{M} R^2 \Omega = -2\pi R^2 \nu \Sigma \frac{\partial \Omega}{\partial R} - \dot{M} R_I^2 \Omega_I.
\]
If we insist that $\Omega \propto R^{-3/2}$, then $\partial_R \Omega = -(3/2)\Omega/R$, so
\begin{equation}
    \label{eq:viscous_density_free_surface}
    \boxed{
    \nu\Sigma = \frac{\dot{M}}{3\pi}\left[1 - \left(\frac{R_{I}}{R}\right)^{1/2}\right].
    }
\end{equation}
This expression tells us that the surface density (weighted by viscosity) is \textbf{proportional to the accretion rate} and the term in the parenthesis scales from $0$ (at ISCO) out to $1$ at large radii. \rmk{This is the \textbf{second} of our critical relations!}

\vspace{10pt}
\paragraph{Surface Boundary}
In a scenario where the accretor has a \textbf{material surface} (e.g.\ a neutron star), the inner boundary at $R_*$ may exert a finite torque $G_*$ on the disk. In this case, equation~\eqref{eq:disk_steady_integral} becomes
\[
R^3 \Sigma v_R \Omega = \frac{1}{2\pi}\left[G(R) - G_*\right].
\]
Using the definition of the accretion rate, equation~\eqref{eq:disc_acc_rate},
\[
\dot{M} = 2\pi R \Sigma (-v_R) \quad \implies \quad R^3 \Sigma v_R \Omega = -\frac{\dot{M}}{2\pi}R^2\Omega,
\]
we may substitute back to obtain
\[
-\dot{M}R^2\Omega = -2\pi R^2 \nu \Sigma \frac{\partial \Omega}{\partial R} - G_*.
\]
If we again assume Keplerian rotation, $\Omega \propto R^{-3/2}$ so that $\partial_R \Omega = -(3/2)\Omega/R$, the equation simplifies to
\[
-\dot{M}R^2\Omega = 3\pi R \nu \Sigma \Omega - G_*.
\]
Rearranging gives
\begin{equation}
    \label{eq:viscous_density_surface}
    \boxed{
    \nu \Sigma = \frac{\dot{M}}{3\pi}\left[1 - \frac{j_* - G_*/\dot{M}}{j(R)}\right],
    }
\end{equation}
where $j(R) = R^2\Omega(R)$ is the specific angular momentum at radius $R$ and $j_* = R_*^2 \Omega_*$ is that at the stellar surface.  

This expression shows that the surface density (weighted by viscosity) is again \textbf{proportional to the accretion rate}, but now explicitly depends on the \emph{torque applied at the surface}. For $G_*=0$ (no torque, as in the black hole case), we recover equation~\eqref{eq:viscous_density_free_surface}. For a material surface with spin, however, $G_*$ modifies the inner boundary behavior and alters the dissipation profile in the inner disk.
\par
\begin{bigidea}
We have now established the \textbf{two foundational relations} that form the backbone of the thin--disk model:
\[
\begin{aligned}
\dot{M} &= 2\pi R \Sigma (-v_R)
& \hspace{2em} & 
(\dot{M},\, \Sigma,\, v_R)
\\[1em]
\nu\Sigma &= \frac{\dot{M}}{3\pi}\!\left[1 - \left(\frac{R_{\rm in}}{R}\right)^{1/2}\right]
& &
(\nu,\, \Sigma,\, \dot{M},\, R_{\rm in},\, R)
\label{eq:summary_viscous_surface_density}
\end{aligned}
\]
The first expresses \textbf{mass conservation}---the constancy of the mass flux through the disk at steady state.  
The second arises from \textbf{angular momentum conservation} and encodes how the combination $\nu\Sigma$ adjusts to enforce the specified inner boundary condition.  
Together, they define the global flow of mass and angular momentum in the disk.
\end{bigidea}

\vspace{10pt}
\subsection{Energy Dissipation in the Thin Disk}
Having established the steady--state surface density structure of the disk, we now turn to the question of \textbf{energy dissipation}. In an accretion disk, viscous stresses transport angular momentum outward, and the resulting loss of gravitational binding energy is dissipated locally as heat. This viscous dissipation is ultimately responsible for the observed radiative luminosity of the disk.
\vspace{0.3cm}
\noindent
The dissipation rate per unit surface area of the disk at radius $R$ is, from equation~\eqref{eq:disk_dissipation_per_unit_area},
\begin{equation}
    D(R) = \frac{1}{2}\,\nu\Sigma\,
    \left(R\frac{d\Omega}{dR}\right)^2.
\end{equation}
For a Keplerian rotation law, $\Omega(R) = (GM/R^3)^{1/2}$, we have
\[
\frac{d\Omega}{dR} = -\frac{3}{2}\frac{\Omega}{R}.
\]
Substituting this yields
\begin{equation}
    \label{eq:disk_dissipation_profile}
    \boxed{
    D(R) = \frac{9}{8}\,\nu\Sigma\,\Omega^2.
    }
\end{equation}
\vspace{0.3cm}
\noindent
To proceed, we insert the steady--state expression for $\nu\Sigma$ obtained previously.  
In the \textbf{free-surface case} (appropriate to black holes, where the torque vanishes at the ISCO), 
we found
\[
\nu\Sigma = \frac{\dot{M}}{3\pi}
\left[1 - \left(\frac{R_{\rm in}}{R}\right)^{1/2}\right].
\]
Substituting into equation~\eqref{eq:disk_dissipation_profile} gives
\begin{equation}
\label{eq:free_surface_disip_rate}
\boxed{
D(R) = \frac{3}{8\pi}\frac{GM\dot{M}}{R^3}
\left[1 - \left(\frac{R_{\rm in}}{R}\right)^{1/2}\right].
}
\end{equation}
This is the standard dissipation profile of a Keplerian thin disk around a black hole.  
The dissipation vanishes at $R=R_{\rm in}$ because the viscous torque is zero there.

\vspace{0.3cm}
\noindent
If instead the accretor possesses a \textbf{finite surface torque} $G_*$ (as for a star with a solid surface), 
the expression becomes
\begin{equation}
    D(R) = \frac{3}{8\pi}\frac{GM\dot{M}}{R^3}
    \left[1 - \frac{j_* - G_*/\dot{M}}{j(R)}\right],
\end{equation}
where $j(R)=R^2\Omega(R)$ is the specific angular momentum at $R$, and 
$j_*=R_*^2\Omega_*$ that at the stellar surface.
\vspace{0.4cm}
\subsubsection*{Integrated Luminosity}
If all dissipated energy is radiated locally, the luminosity emitted between radii $R_1$ and $R_2$ is
\begin{equation}
    L(R_1,R_2) = 2\pi \int_{R_1}^{R_2} R\,D(R)\,dR.
\end{equation}
For the free--surface case, integrating equation~\eqref{eq:free_surface_disip_rate} from 
$R_{\rm in}$ to infinity gives
\begin{equation}
    \label{L_disk}
    \boxed{
    L_{\rm disk} = \frac{GM\dot{M}}{2R_{\rm in}} = \tfrac{1}{2}L_{\rm acc}.
    }
\end{equation}
Thus, a thin accretion disk radiates away exactly one--half of the gravitational energy 
released by infall from infinity to $R_{\rm in}$.  
The other half remains as kinetic energy in the orbiting gas and \textbf{is carried inward with the flow.  }
For black holes this energy disappears through the event horizon, while for neutron stars 
and white dwarfs it is released in the \textbf{boundary layer} as the accreting gas is brought into 
corotation with the stellar surface. We were able to make this same argument just on the basis of the available energy budget, but we are now showing that we can actually get \textbf{all of that energy out}!

\vspace{0.4cm}
\subsubsection*{Radial Dependence of Dissipation}
The dissipation profile from equation~\eqref{eq:free_surface_disip_rate} is
\[
D(R) = \frac{3GM\dot{M}}{8\pi R^3}
\left[1 - \left(\frac{R_{\rm in}}{R}\right)^{1/2}\right].
\]
At large radii ($R \gg R_{\rm in}$), this scales as $D \propto R^{-3}$, while near $R_{\rm in}$ the 
dissipation is suppressed by the vanishing torque condition. Integrating this profile shows that roughly three--quarters of the total luminosity is emitted within a factor of two of the inner edge:
\begin{equation}
    L(R < 2R_{\rm in}) \approx \frac{3}{4} L_{\rm disk}.
\end{equation}
This concentration of dissipation explains why the innermost disk regions dominate the observed luminosity, especially in X-rays for compact accretors. It is also instructive to compare the dissipation rate with the local rate of release of gravitational binding energy. Material of mass $\dot{M}\,dt$ moving inward releases energy at a rate
\begin{equation}
    \frac{dL_{\rm bind}}{dR} = \frac{GM\dot{M}}{2R^2}.
\end{equation}
By contrast, the actual rate of viscous dissipation in an annulus is
\begin{equation}
    \frac{dL_{\rm diss}}{dR} = 2\pi R D(R) 
    = \frac{3GM\dot{M}}{2R^2}\left[1 - \left(\frac{R_{\rm in}}{R}\right)^{1/2}\right].
\end{equation}
The two are related by
\begin{equation}
    \frac{dL_{\rm diss}}{dR} = \frac{dL_{\rm bind}}{dR}
    \left[ 3 - 3\left(\frac{R_{\rm in}}{R}\right)^{1/2} \right].
\end{equation}
At large radii ($R \gg R_{\rm in}$), the bracket tends to $3$, indicating that each annulus radiates three times more power than it gains from its own local gravitational energy release.  The excess energy originates from viscous torques, which transport energy outward from smaller radii.  At $R = (9/4)R_{\rm in}$ the two rates are equal, marking the transition between inner and outer disk behavior:
\begin{itemize}
    \item \textbf{Inner disk} ($R < 9R_{\rm in}/4$): dissipation is \emph{less} than local binding energy release, 
    since part of the energy is carried outward.  
    \item \textbf{Outer disk} ($R > 9R_{\rm in}/4$): dissipation exceeds local energy release, powered by energy 
    transported outward by viscous torques.  
\end{itemize}
This redistribution of energy explains both the centrally concentrated emission and the fact that 
the outer disk shines more brightly than would be expected from its own gravitational potential energy alone.

\begin{bigidea}
The \textbf{dissipation law for a steady, Keplerian thin disk} is
\begin{equation}
\boxed{
\
D(R) = \frac{3}{8\pi}\,\frac{GM\dot{M}}{R^3}
\left[1 - \left(\frac{R_{\rm in}}{R}\right)^{1/2}\right].\;\;\;\;(D,\, R,\, R_{\rm in},\, M,\, \dot{M})
}
\end{equation}
This is the central result connecting angular momentum transport, viscous heating, and the emergent luminosity of the thin disk.
\end{bigidea}


\subsection{Vertical Structure of the Thin Disk}

We now turn to the \textbf{vertical structure} of the thin accretion disk.  Because the disk is rotationally supported in the radial direction and the \textbf{vertical velocity is negligible} ($v_z \approx 0$),  the disk must be in \textbf{vertical hydrostatic equilibrium} (HSE). This means that the vertical pressure gradient balances the vertical component of gravity. In cylindrical coordinates $(R, \phi, z)$, neglecting disk self-gravity, the vertical component of the \textbf{Euler Equation} is
\begin{equation}
\frac{1}{\rho}\frac{dP}{dz} = -\,\frac{\partial \Phi}{\partial z},
\end{equation}
where $\Phi$ is the gravitational potential of the central object.  
For a point mass $M$,
\[
\Phi(R,z) = -\,\frac{GM}{\sqrt{R^2 + z^2}}.
\]
The vertical gravitational acceleration is therefore
\[
\frac{\partial \Phi}{\partial z} = \frac{GMz}{(R^2 + z^2)^{3/2}}.
\]
Since the thin-disk assumption requires $z \ll R$, we can expand this expression using a binomial expansion:
\[
(R^2 + z^2)^{-3/2} \simeq R^{-3}\left(1 - \frac{3z^2}{2R^2} + \cdots\right),
\]
and keep only the leading term:
\begin{equation}
\frac{\partial \Phi}{\partial z} \simeq \frac{GM}{R^3}\,z \;=\; \Omega_K^2\,z,
\end{equation}
where $\Omega_K = \sqrt{GM/R^3}$ is the Keplerian angular velocity. The hydrostatic equilibrium equation now becomes
\begin{equation}
\frac{dP}{dz} = -\,\rho\,\Omega_K^2\,z.
\label{eq:vertical_hse}
\end{equation}
To close this equation, we assume an isothermal equation of state in the vertical direction:
\[
P = \rho\,c_s^2,
\]
where $c_s$ is the isothermal sound speed, assumed constant with height. Substituting into equation~\eqref{eq:vertical_hse} gives
\[
c_s^2\,\frac{d\rho}{dz} = -\,\rho\,\Omega_K^2\,z.
\]
Separating variables and integrating,
\[
\int \frac{d\rho}{\rho} = -\,\frac{\Omega_K^2}{c_s^2}\int z\,dz
\quad\Rightarrow\quad
\ln\rho = -\,\frac{z^2}{2H^2} + \text{const},
\]
where we define the \textbf{scale height}
\begin{equation}
\boxed{
H \;\equiv\; \frac{c_s}{\Omega_K}.
}
\end{equation}
Thus, the vertical density profile is Gaussian:
\begin{equation}
\boxed{
\rho(R,z) = \rho_0(R)\,
\exp\!\left[-\,\frac{z^2}{2H^2}\right],
}
\end{equation}
where $\rho_0(R)$ is the midplane density. Notably, this is why we so frequently care to model \textbf{Gaussian Disks}! If we integrate over this, we obtain
\begin{equation}
    \boxed{
    \Sigma = \sqrt{2\pi} \rho_0 H.
    }
\end{equation}
Likewise, if we want to know the pressure balance, we can return to the differential equation. Clearly
\[
dP = - \rho \Omega_k^2 z\;dz = - \rho_0 \exp\left(\frac{-z^2}{2H^2}\right) \Omega_k^2z \;dz.
\]
If we integrate this equation from $0$ to $\infty$, we find
\[
P_0 = \rho_0 \Omega_K^2 H^2 \implies P_0 = \frac{1}{\sqrt{2\pi}} \Sigma \Omega_k^2 H.
\]
As such, we find a relationship between the central pressure at the scale height:
\begin{equation}
    \boxed{
    P_0 = \frac{1}{\sqrt{2\pi}} \Sigma \Omega_k^2 H.
    }
\end{equation}
\par
The \textbf{thin-disk approximation} requires that the disk be geometrically thin, i.e.
\begin{equation}
\frac{H}{R} = \frac{c_s}{v_\phi} \ll 1,
\end{equation}
since $v_\phi \simeq R\,\Omega_K$ is the orbital velocity.  This means that the gas must be \emph{cold} compared to the orbital motion—\textbf{its sound speed must be significantly less than the orbital velocity:}
\begin{equation}
\boxed{
c_s \ll v_\phi.
}
\end{equation}
If this condition is violated (i.e.\ if $c_s$ approaches or exceeds $v_\phi$), the vertical pressure forces would cause the disk to \textbf{puff up}, and the thin-disk approximation would no longer be valid. Such a flow becomes a \textbf{thick disk} or \textbf{advection-dominated} accretion flow (ADAF), which requires a different treatment.
\par
\subsubsection*{Checking the Keplerian Nature of Orbits}

Up to this point, we have repeatedly assumed that the tangential velocity of the disk material is \textbf{Keplerian}, yet we have not explicitly shown this to be the case.  We now verify this assumption by examining the \textbf{radial component} of the Euler equation, and by checking the relative importance of each contributing term. The steady--state, axisymmetric Euler equation in the radial direction is
\begin{equation}
v_r \frac{dv_r}{dR} - \frac{v_\phi^2}{R}
= -\,\frac{1}{\rho}\frac{dP}{dR} - \frac{GM}{R^2}.
\label{eq:radial_euler}
\end{equation}
The terms represent, respectively, the inertial, centrifugal, pressure--gradient, and gravitational forces per unit mass.  We can estimate the relative importance of each term by recalling that in a geometrically thin disk,
\[
\frac{H}{R} \equiv \frac{c_s}{v_\phi} \ll 1,
\]
where $H$ is the scale height and $c_s$ the sound speed. Using the standard $\alpha$--prescription for viscosity,
\[
\nu = \alpha c_s H,
\]
and the steady--state radial velocity derived earlier,
\[
v_r \sim \frac{\nu}{R} \sim \alpha \left(\frac{H}{R}\right)^2 v_\phi,
\]
we can compare the typical magnitudes of the radial Euler terms.

\begin{center}
\renewcommand{\arraystretch}{1.4}
\begin{tabular}{lcc}
\toprule
\textbf{Term} & \textbf{Typical Magnitude} & \textbf{Relative to Gravity ($GM/R^2$)} \\
\midrule
Centrifugal, $v_\phi^2/R$ & $\sim GM/R^2$ & $1$ \\
Pressure gradient, $(1/\rho)\,dP/dR$ & $\sim c_s^2 / R$ & $(H/R)^2 \ll 1$ \\
Radial acceleration, $v_r dv_r/dR$ & $\sim v_r^2 / R$ & $\sim \alpha^2 (H/R)^4 \ll (H/R)^2$ \\
\bottomrule
\end{tabular}
\end{center}

The table clearly shows that the \textbf{pressure} and \textbf{radial inertial} terms are negligibly small compared to the gravitational and centrifugal forces. Thus, to leading order,
\begin{equation}
\frac{v_\phi^2}{R} \simeq \frac{GM}{R^2},
\end{equation}
which immediately implies
\begin{equation}
\boxed{
v_\phi(R) \simeq v_K(R) = \sqrt{\frac{GM}{R}}.
}
\end{equation}

\begin{bigidea}
We have now established two of the key \textbf{vertical structural relations} that connect the thermodynamic and geometric properties of the thin disk:
\[
\begin{aligned}
H &= \frac{c_s}{\Omega_K}
& \hspace{3em} &
(H,\, c_s,\, \Omega_K)
\\[1em]
\rho_0 &= \frac{\Sigma}{\sqrt{2\pi}\,H}
& &
(\rho_0,\, \Sigma,\, H)
\end{aligned}
\]
The first defines the \textbf{vertical scale height} in hydrostatic equilibrium, showing that vertical thickness is set by the ratio of sound speed to orbital speed.  
The second links the \textbf{midplane density} to the surface density through the Gaussian vertical structure.  
Together, these provide the bridge between local thermodynamics (through $c_s$ and $T$) and the global mass distribution (through $\Sigma$).
\end{bigidea}

\subsection{Radiative Transport in Thin Disks}

At this point, we have solved \textbf{independently} for the structure of the vertical disk:
\[
\rho(R,z) = \rho_0(R)\,
\exp\!\left[-\,\frac{z^2}{2H^2}\right],
\]
but we still have the task of relating this to the \textbf{disk itself}. Furthermore, we'd like to know information about the emission, the temperature, etc. from the disk. We therefore need to turn our attention toward \textbf{radiative transfer}.
\par
In order to specify the scale height $H \sim c_s/\Omega_k$, we need to know the \textbf{density and the pressure} in the disk. We can \textbf{estimate the density} as 
\[
\rho \sim \frac{\Sigma}{H},
\]
but the pressure will depend on both the \textbf{ideal gas EOS} and on \textbf{radiation pressure}:
\[
P = \frac{\rho k_B T}{\mu m_p} + \frac{4\sigma_{\rm SB}}{3c}T^4.
\]
\bigskip
\noindent
\par
Clearly, this equation of state alone does not \emph{close} the system of disk equations: we still lack a relation describing how energy generated by viscous dissipation is transported and radiated away. To proceed, we therefore introduce an \textbf{energy equation} based on radiative diffusion.

\subsubsection*{Radiative Transfer and the Diffusion Approximation}

In an optically thick medium---such as a geometrically thin accretion disk---photons are repeatedly absorbed, re-emitted, and scattered before escaping the surface. In this regime, the radiation field is nearly isotropic and can be described by the \textbf{diffusion approximation}. The specific intensity $I_\nu$ of radiation obeys the \textbf{radiative transfer equation}:
\begin{equation}
\frac{dI_\nu}{ds} = -\kappa_\nu \rho\, I_\nu + \kappa_\nu \rho\, S_\nu,
\label{eq:radiative_transfer_eqn}
\end{equation}
where $\kappa_\nu$ is the opacity (per unit mass), $\rho$ the density, and $S_\nu$ the source function.  
In local thermodynamic equilibrium (LTE), $S_\nu = B_\nu(T)$, where $B_\nu(T)$ is the Planck function.
In the diffusion limit, the radiation field deviates only slightly from isotropy, allowing us to expand it as
\[
I_\nu(\hat{\bf n}) = B_\nu(T) + \delta I_\nu(\hat{\bf n}),
\qquad
\text{with } |\delta I_\nu| \ll B_\nu.
\]
Integrating equation~\eqref{eq:radiative_transfer_eqn} over solid angle and using this expansion leads to the \textbf{radiative flux} at frequency $\nu$:
\begin{equation}
F_\nu = -\,\frac{4\pi}{3\kappa_\nu\rho}\,\frac{dB_\nu}{dz}.
\label{eq:freq_diffusion_flux}
\end{equation}
This expresses \textbf{the diffusive nature of radiative transport}: energy flows down the temperature gradient, with a ``radiative conductivity'' proportional to $1/(\kappa_\nu\rho)$. The total flux is the sum over all frequencies:
\begin{equation}
F = \int_0^\infty F_\nu\,d\nu
= -\,\frac{4\pi}{3\rho}\int_0^\infty
\frac{1}{\kappa_\nu}\,\frac{dB_\nu}{dz}\,d\nu.
\end{equation}
Since $B_\nu$ depends on $T$, we can write
\[
\frac{dB_\nu}{dz} = \frac{dB_\nu}{dT}\frac{dT}{dz},
\]
so that
\begin{equation}
F = -\,\frac{4\pi}{3\rho}\,\frac{dT}{dz}
\int_0^\infty \frac{1}{\kappa_\nu}\,\frac{dB_\nu}{dT}\,d\nu.
\label{eq:flux_integral_dB}
\end{equation}
We now seek to define a single \emph{effective} opacity $\kappa_R$ such that this expression reproduces the familiar frequency-integrated diffusion law we would obtain if $\kappa_\nu$ was frequency independent:
\begin{equation}
F = -\,\frac{16\sigma_{\rm sb} T^3}{3\kappa_R\rho}\,\frac{dT}{dz}.
\label{eq:diffusion_equation}
\end{equation}
To do so, we equate equations~\eqref{eq:flux_integral_dB} and \eqref{eq:diffusion_equation}, and note that
\[
\int_0^\infty \frac{dB_\nu}{dT}\,d\nu = \frac{4\sigma_{\rm sb}T^3}{\pi}.
\]
We thus define the \textbf{Rosseland mean opacity}:
\begin{equation}
\boxed{
\frac{1}{\kappa_R} =
\frac{\displaystyle \int_0^\infty \frac{1}{\kappa_\nu}
\frac{\partial B_\nu}{\partial T}\,d\nu}
{\displaystyle \int_0^\infty \frac{\partial B_\nu}{\partial T}\,d\nu}.
}
\label{eq:rosseland_mean}
\end{equation}
This is a \emph{harmonic mean} of the frequency-dependent opacity, weighted by $\partial B_\nu/\partial T$, which emphasizes the frequencies that most efficiently carry energy (those where $\kappa_\nu$ is smallest).  
Physically, the Rosseland mean represents the ``effective resistance'' to radiative energy flow in an optically thick medium, analogous to a set of parallel resistors: photons escape preferentially through low-opacity windows.
\par
Finally, energy conservation in the steady state requires that the vertical divergence of the radiative flux balances the local viscous heating rate per unit volume, $q^+$:
\begin{equation}
\frac{dF}{dz} = q^+ = \frac{9}{4}\,\nu\,\rho\,\Omega_K^2.
\end{equation}
Integrating this from the midplane ($z=0$, where $F=0$ by symmetry) to the disk surface ($z=H$, where $F=F_{\rm surf}$) yields the emergent flux:
\[
\boxed{
F_{\rm surf} = \frac{9}{8}\,\nu\,\Sigma\,\Omega_K^2,
}
\]
which must equal the radiative flux escaping from each face of the disk:
\begin{equation}
\boxed{
F_{\rm surf} = \sigma_{\rm sb}T_{\rm eff}^4 = \frac{9}{8}\nu \Sigma\Omega_K^2.
}
\end{equation}
This condition provides the final \textbf{closure relation} linking the vertical temperature gradient, midplane temperature, and effective surface temperature via radiative diffusion, completing the thermal structure of the thin disk.
\par
We'd also like to know about the \textbf{core temperature} not just the \textbf{effective temperature}. We can do this using the \textbf{mean diffusion equation} written above:
\begin{equation}
F(z)\;=\;-\frac{16\sigma_{\rm sb}T^3}{3\,\kappa_R\rho}\,\frac{dT}{dz},
\label{eq:diffusion_depth_start}
\end{equation}
introduce the Rosseland optical depth $\tau$ measured downward from the surface,
\[
d\tau \;\equiv\; -\,\kappa_R\,\rho\,dz,
\qquad\Rightarrow\qquad
\frac{d}{dz} \;=\; -\,\kappa_R\rho\,\frac{d}{d\tau}.
\]
Equation~\eqref{eq:diffusion_depth_start} then becomes
\begin{equation}
F(\tau)\;=\;\frac{16\sigma_{\rm sb}T^3}{3}\,\frac{dT}{d\tau}
\;=\;\frac{4\sigma_{\rm sb}}{3}\,\frac{dT^4}{d\tau}.
\label{eq:F_tau_relation}
\end{equation}

In radiative equilibrium (grey atmosphere with the Eddington closure), the vertical flux is
depth–independent and equal to the emergent flux from one disk face:
\[
F(\tau)\;=\;F_{\rm surf}\;=\;\sigma_{\rm sb}T_{\rm eff}^4
\quad \text{(constant in $z$)}.
\]
Substituting into \eqref{eq:F_tau_relation} and integrating from the surface
($\tau=0$) downward gives
\begin{equation}
T^4(\tau)\;=\;\frac{3}{4}\,T_{\rm eff}^4\,\big(\tau + q\big),
\label{eq:grey_profile_general}
\end{equation}
where $q$ is an order–unity boundary constant that depends on the closure. For the
standard Eddington boundary condition one has $q=2/3$, yielding the familiar grey–atmosphere profile
\begin{equation}
\boxed{ \; T^4(\tau)\;=\;\frac{3}{4}\,T_{\rm eff}^4\!\left(\tau + \frac{2}{3}\right). \;}
\label{eq:grey_profile}
\end{equation}

For a thin disk with two identical faces, the (one–sided) Rosseland depth to the midplane is
\[
\tau_c \;\equiv\; \int_0^{\infty}\kappa_R\rho\,dz
\;=\;\frac{1}{2}\,\kappa_R\,\Sigma,
\]
so the midplane temperature $T_c\equiv T(\tau_c)$ follows from \eqref{eq:grey_profile}:
\begin{equation}
\boxed{ \;
T_c^4 \;=\;\frac{3}{4}\,T_{\rm eff}^4\!\left(\tau_c + \frac{2}{3}\right)
\;=\;\frac{3}{4}\,T_{\rm eff}^4\!\left(\frac{\kappa_R\Sigma}{2} + \frac{2}{3}\right).
\;}
\label{eq:Tc_Teff_exact}
\end{equation}
In the optically thick limit $\tau_c\gg1$ this reduces to the widely used relation
\begin{equation}
\boxed{ \;
T_c^4 \;\simeq\; \frac{3}{4}\,\tau_c\,T_{\rm eff}^4
\;=\;\frac{3}{8}\,\kappa_R\,\Sigma\,T_{\rm eff}^4.
\;}
\label{eq:Tc_Teff_tau}
\end{equation}
\begin{itemize}
\item The opacity enters through $\tau_c=\tfrac{1}{2}\kappa_R(\rho,T_c)\,\Sigma$, so
\eqref{eq:Tc_Teff_tau} implicitly couples $T_c$ to $(\Sigma,\rho)$ and the chosen opacity law.
\item Equation~\eqref{eq:grey_profile} assumes a grey, isotropic radiation field with the Eddington boundary condition. Alternative closures shift the constant $2/3$ but leave the $\propto (\tau+{\rm const})$ structure intact.
\item If vertical viscous heating is explicitly distributed with height (so that $dF/dz=q^+(z)$), one obtains a slightly modified $T(\tau)$; in practice, for $\tau_c\gg1$, the midplane relation \eqref{eq:Tc_Teff_tau} remains accurate to order unity.
\end{itemize}
\begin{bigidea}
The \textbf{radiative flux} emerging from each face of a steady, optically thick, viscous accretion disk is
\begin{equation}
\boxed{
F_{\rm surf} \;=\; \sigma_{\rm sb}\,T_{\rm eff}^4
\;=\;
\frac{9}{8}\,\nu\,\Sigma\,\Omega_K^2.
}
\end{equation}
This relation closes the thin–disk model by connecting the \textbf{thermal emission} 
($T_{\rm eff}$) to the \textbf{dynamical quantities} ($\nu$, $\Sigma$, and $\Omega_K$) that govern viscous energy dissipation.
We also derived that
\[
\boxed{
T_c^4 = \frac{3}{8}\kappa_R \Sigma T_{\rm eff}^4.
}
\]
\end{bigidea}

\section{The Thin Disk Model Equations}

We now have all of the ingredients necessary to summarize and connect the equations that define the \textbf{standard thin accretion disk model}. Each relation we have derived describes one aspect of the physics: mass conservation, angular momentum transport, hydrostatic support, and energy balance. Together, these form a \textbf{closed system} that fully specifies the disk structure once the parameters $\dot{M}$ (accretion rate), $M$ (central mass), and $\nu$ (viscosity law) and opacity $\kappa$ are prescribed.

\subsection*{Governing Equations}

\paragraph{1. Mass Conservation}
From the steady--state continuity equation,
\begin{equation}
\boxed{
\dot{M} = 2\pi R \Sigma (-v_R),
}
\label{eq:thin_disk_mass}
\end{equation}
which enforces a constant mass flux through the disk.  
This connects the local surface density $\Sigma(R)$, radial inflow velocity $v_R$, and the global accretion rate $\dot{M}$.

\vspace{1em}
\paragraph{2. Angular Momentum Conservation}
From the steady--state angular momentum equation and the torque definition,
\begin{equation}
\boxed{
\nu \Sigma = \frac{\dot{M}}{3\pi}
\left[1 - \left(\frac{R_{\rm in}}{R}\right)^{1/2}\right].
}
\label{eq:thin_disk_viscous}
\end{equation}
This expresses how the combination $\nu\Sigma$ adjusts with radius to maintain a steady flow and the chosen inner boundary condition (e.g.\ zero torque at $R_{\rm in}$).

\vspace{1em}
\paragraph{3. Vertical Hydrostatic Equilibrium}
The vertical pressure gradient balances the vertical component of gravity:
\begin{equation}
\boxed{
\frac{dP}{dz} = -\,\rho\,\Omega_K^2\,z,
\qquad
H = \frac{c_s}{\Omega_K}.
}
\label{eq:thin_disk_hse}
\end{equation}
This defines the \textbf{scale height} $H$ in terms of the sound speed $c_s$, linking the disk’s thermodynamics to its geometry.

\vspace{1em}
\paragraph{4. Vertical Density Structure}
Assuming an isothermal vertical equation of state $P = \rho c_s^2$, the vertical density profile is Gaussian:
\begin{equation}
\boxed{
\rho(R,z) = \rho_0(R)\,\exp\!\left[-\,\frac{z^2}{2H^2}\right],
\qquad
\Sigma = \sqrt{2\pi}\,\rho_0\,H.
}
\label{eq:thin_disk_density}
\end{equation}
This connects the midplane density $\rho_0$ to the surface density $\Sigma$, providing a bridge between vertical and radial structure. 

\vspace{1em}
\paragraph{5. Energy Dissipation (Viscous Heating)}
Viscous stresses convert gravitational binding energy into heat at a rate per unit area:
\begin{equation}
\boxed{
D(R) = \frac{3}{8\pi}\frac{GM\dot{M}}{R^3}
\left[1 - \left(\frac{R_{\rm in}}{R}\right)^{1/2}\right].
}
\label{eq:thin_disk_dissipation}
\end{equation}
This determines where in the disk energy is released, and it peaks near the inner edge ($R \sim 2R_{\rm in}$).

\vspace{1em}
\paragraph{6. Radiative Flux and Effective Temperature}
Assuming local energy balance ($D = \sigma_{\rm sb} T_{\rm eff}^4$), the emergent radiative flux from each face of the disk is
\begin{equation}
\boxed{
F_{\rm surf} = \sigma_{\rm sb}T_{\rm eff}^4
= \frac{9}{8}\,\nu\,\Sigma\,\Omega_K^2.
}
\label{eq:thin_disk_flux}
\end{equation}
This connects the observable surface temperature $T_{\rm eff}$ to the viscous heating rate via $\nu$, $\Sigma$, and $\Omega_K$.

\vspace{1em}
\paragraph{7. Equation of State and Sound Speed}
Pressure arises from both gas and radiation:
\begin{equation}
\boxed{
P = \frac{\rho k_B T}{\mu m_p} + \frac{4\sigma_{\rm sb}}{3c}T^4,
\qquad
c_s = \sqrt{\frac{P}{\rho}}.
}
\label{eq:thin_disk_eos}
\end{equation}
This relation closes the thermodynamic part of the model, allowing the temperature to determine both $H$ and the viscosity.
\paragraph{8. Temperature Relation}
The temperatures (core and effective) are related by
\[
\boxed{
T_c^4 = \frac{3}{8}\kappa_R \Sigma T_{\rm eff}^4.
}
\]

\bigskip
\subsection*{Model Parameters and Interdependencies}

The primary \textbf{inputs} to the thin disk model are:
\[
(M,\, \dot{M},\, \alpha,\, \kappa(\rho,T)),
\]
where $M$ is the central mass, $\dot{M}$ is the accretion rate, $\alpha$ parameterizes the viscosity through $\nu = \alpha c_s H$, and $\kappa(\rho,T)$ specifies the opacity law (e.g.\ Kramers or Thomson).

Given these inputs, the system of equations 
\eqref{eq:thin_disk_mass}–\eqref{eq:thin_disk_eos}
can be solved self–consistently for the disk structure:
\[
\{\Sigma(R),\, \rho_0(R),\, T(R),\, H(R),\, \nu(R),\, v_R(R),\, T_{\rm eff}(R)\}.
\]
Each variable depends on the others through one or more of the relations above.  
In practice, one typically solves iteratively for $T(R)$, since it enters both through the sound speed and the radiative diffusion law.

\subsection*{Derived Quantities}

Once the primary structure is known, several important derived observables can be computed:

\begin{itemize}
    \item \textbf{Luminosity:}
    \[
    L_{\rm disk} = 2\pi \int_{R_{\rm in}}^{R_{\rm out}} R\,D(R)\,dR
    = \frac{GM\dot{M}}{2R_{\rm in}}.
    \]
    \item \textbf{Effective Temperature Profile:}
    \[
    T_{\rm eff}(R)
    = \left[\frac{3GM\dot{M}}{8\pi\sigma_{\rm sb}R^3}
    \left(1 - \sqrt{\frac{R_{\rm in}}{R}}\right)\right]^{1/4}.
    \]
    \item \textbf{Spectral Flux:}
    Treating each annulus as a blackbody emitter,
    \[
    F_\nu = \frac{4\pi h\nu^3\cos i}{c^2 D^2}
    \int_{R_{\rm in}}^{R_{\rm out}}
    \frac{R\,dR}{\exp[h\nu/(k_B T_{\rm eff}(R))]-1},
    \]
    which yields the multi–temperature blackbody spectrum characteristic of thin disks.
\end{itemize}

\bigskip
\noindent
\textbf{Summary.}
The thin disk model is thus governed by a small number of physically transparent equations:
mass conservation, angular momentum conservation, vertical hydrostatic equilibrium, and radiative energy balance.  
Once $\nu(\rho,T,R)$ and $\kappa(\rho,T)$ are specified, all other quantities—geometry, thermodynamics, and emission—follow self–consistently from these relations.


\section{The Emitted Spectrum}

In the previous sections, we established that a geometrically thin accretion disk is \textbf{optically thick} and therefore radiates approximately as a \textbf{blackbody} at each radius.  The emergent flux from each face of the disk satisfies
\[
F(R) = \sigma_{\rm SB}\,T_{\rm eff}^4(R),
\]
where $T_{\rm eff}(R)$ is the \textbf{effective temperature} of the photosphere at radius $R$.  Importantly, $T_{\rm eff}(R)$ represents the radiating temperature at the disk surface, not the (larger) midplane temperature $T_c(R)$ that sets the internal pressure support.
\par
From the balance between viscous heating and radiative cooling,
\[
F(R) = D(R) = \frac{9}{8}\,\nu\,\Sigma\,\Omega_K^2,
\]
and for a steady accretion rate $\dot{M}$, we have $\nu\Sigma = \dot{M}/(3\pi)\left[1 - \sqrt{R_{\rm in}/R}\right]$. Substituting into the expression above gives the standard thin–disk temperature law:
\begin{equation}
\boxed{
\sigma_{\rm SB}\,T_{\rm eff}^4(R)
= \frac{3GM\dot{M}}{8\pi R^3}
\!\left[1 - \sqrt{\frac{R_{\rm in}}{R}}\right].
}
\label{eq:thin_disk_temperature}
\end{equation}
In the outer disk ($R \gg R_{\rm in}$), the bracket approaches unity, and we find the asymptotic scaling
\begin{equation}
\boxed{
T_{\rm eff}(R) \propto R^{-3/4}.
}
\label{eq:temperature_scaling}
\end{equation}
More concretely, 
\begin{equation}
    T = \left(\frac{R}{R_\star}\right)^{-3/4} \cdot\begin{cases}4.1 \times 10^4\;\dot{M}_{\rm 16}^{1/4} m_1^{1/4} R_9^{-3/4}\;{\rm K},&\text{(WD)}\\
    1.3 \times 10^7\;\dot{M}_{\rm 17}^{1/4} m_1^{1/4} R_6^{-3/4}\;{\rm K},&\text{(NS)}
    \end{cases}
\end{equation}
Thus, while the outer regions of disks radiate p\textbf{rimarily in the optical or infrared, the inner regions can dominate the emission at ultraviolet or X–ray wavelengths, depending on the depth of the potential well.}

\subsection*{Multi–Temperature Blackbody Spectrum}

Each annulus of the disk radiates as a blackbody at its local $T_{\rm eff}(R)$, with specific intensity
\[
I_\nu(R) = B_\nu[T_{\rm eff}(R)]
= \frac{2h\nu^3}{c^2}\left[\exp\!\left(\frac{h\nu}{kT_{\rm eff}(R)}\right)-1\right]^{-1}.
\]
The total observed flux (for a disk viewed at inclination $i$) is obtained by integrating over radius:
\begin{equation}
F_\nu = \frac{2\pi\cos i}{D^2}
\int_{R_{\rm in}}^{R_{\rm out}} B_\nu[T_{\rm eff}(R)]\,R\,dR,
\label{eq:disk_spectrum_integral}
\end{equation}
where $D$ is the source distance. Because $T_{\rm eff}(R)$ decreases outward, this represents a \textbf{sum of blackbodies} spanning a wide range of temperatures—an \emph{extended blackbody spectrum}. The disk’s emission thus forms a continuous spectrum rather than a single-temperature Planck curve.e

\subsection*{Asymptotic Behavior of the Spectrum}

The integral in equation~\eqref{eq:disk_spectrum_integral} yields three characteristic regimes:

\begin{itemize}
\item \textbf{Rayleigh–Jeans limit ($h\nu \ll kT_{\rm out}$):}  
   The entire disk contributes in the RJ regime, and since $B_\nu \propto \nu^2 T$,  
   integrating over $R$ gives
   \[
   F_\nu \propto \nu^2.
   \]

\item \textbf{Intermediate regime ($kT_{\rm out} \ll h\nu \ll kT_{\rm in}$):}  
   Only the annulus where $h\nu \sim kT_{\rm eff}(R)$ contributes significantly.  
   Using $T_{\rm eff}\propto R^{-3/4}$ and $B_\nu \propto \nu^3/(\exp(h\nu/kT)-1)$, one finds
   \[
   \boxed{F_\nu \propto \nu^{1/3},}
   \]
   the celebrated spectral slope of a multi–temperature blackbody disk.

\item \textbf{Wien limit ($h\nu \gg kT_{\rm in}$):}  
   The exponential cutoff of the Planck function dominates, giving  
   \[
   F_\nu \propto \nu^3\,\exp(-h\nu/kT_{\rm in}).
   \]
\end{itemize}

\medskip
\noindent
These regimes together produce the characteristic \textbf{multi–temperature disk spectrum}, rising as $\nu^2$ at low frequencies, flattening to $\nu^{1/3}$ over a broad intermediate range, and falling exponentially beyond the high–energy cutoff.  
The result is remarkably insensitive to the detailed viscosity prescription—only the temperature profile $T_{\rm eff}(R)$ matters.

\section{The $\alpha$-Disk}
We have now completed our derivation of the \textbf{thin-disk} accretion model and we may now start generating solutions to the coupled equations which can be applied to data! In order to generate a solution to the problem, we need to insist on a prescription for both the \textbf{viscosity} and the \textbf{opacity} of the material. In the famous $\boldsymbol{\alpha}$-\textbf{disk}, we make a single prescription for $\nu$:
\[
\boxed{
\nu = \alpha c_s H,
}
\]
which is our standard $\alpha$ prescription. This alone does not given us a solution as we need the \textbf{opacity law}. In practice, we choose between \textbf{two different opacity laws}:
\vspace{20pt}
\begin{enumerate}
    \item \textbf{Kramer's Opacity}: Describes the opacity due to \textbf{bound-free} absorption where an atom is ionized while absorbing a photo-ionizing photon. This is an effective prescription which is commonly used to describe this type of radiative transfer:
\begin{equation}
    \kappa_{\rm kramer} = 5\times 10^{24}\; \rho T_c^{-7/2}\; {\rm cm^2\;g^{-1}.}
\end{equation}
Critically, \textbf{Kramer's Law} is an effective model of opacities between about $T\sim 10^4$, where ionization fails and $T^{7}$, when electron scattering becomes the dominant source of opacity. 
    \item \textbf{Electron Scattering Opacity}: When temperatures transition to a fully ionized medium, bound-free transitions become considerably less favorable and we instead have relatively generic scattering due to electrons: \textbf{Thompson scattering}. This comes with a generic prescription for the opacity:
\begin{equation}
    \kappa_{\rm thompson} = 0.2 \;(1+X)\;{\rm cm^2\;g^{-1}},
\end{equation}
where $X$ is the Hydrogen abundance fraction.
\end{enumerate}
\vspace{20pt}
We then also consider scenarios where either \textbf{gas pressure dominates}, or \textbf{radiation pressure dominates}. This gives us \textbf{4 distinct region}, which each have different general solutions, but all of which have algebraic closure. In the following sections, we'll discuss each of the regimes.

\begin{remark}
    Before we dive into the solutions below, there are a few thematic manipulations to keep track off. First off, we have
    \[
    \nu \Sigma \sim \dot{M} f^4,
    \]
    and $\nu = \alpha c_s H = \alpha P/ \rho \Omega_k$. We can use the fact that $P \sim \Sigma \Omega_k^2 H$ and $\rho \sim \Sigma / H$ to produce this manipulation. This is almost universal, but the procedure following that will depend on the nature of the pressure support and the opacity.
\end{remark}

\subsection{Region A: Radiation Pressure \& Electron Scattering}
\label{sec:region_a}

In the inner most region of the disk, we are at the \textbf{hottest region of the disk} and anticipate that the disk will be radiation pressure dominated with electron scattering opacity. 

We begin with perhaps the simplest algebraic regime of the $\alpha$-disk model: the  \textbf{radiation pressure–dominated, electron scattering} region of the disk. This case applies to the innermost parts of bright accretion disks around compact objects, where temperatures are sufficiently high that the gas is fully ionized and radiation pressure exceeds gas pressure. The physical assumptions defining this region are:
\begin{equation}
\boxed{
P \simeq P_{\rm rad} = \frac{a}{3}T_c^4, \qquad \kappa = \kappa_{\rm es} = \text{constant}.
}
\end{equation}

\subsubsection{Relevance}

\textbf{Region A} should only occur in the \textbf{hottest}, inner most regions of the disk. To be concrete, there are two conditions which both provide constraints:
\vspace{10pt}
\begin{itemize}
    \item \textbf{The Opacity}: Since the opacity must be \textbf{electron-scattering dominated}, we need
    \[
    \kappa_{\rm es} > \kappa_{\rm kramer}.
    \]
    As we'll see later, this places an effective boundary between \textbf{regions A/B} and \textbf{regions C/D}.
    \item \textbf{The Pressure}: We are also assuming that the pressure is dominated by the \textbf{radiation pressure}, so
    \[
    \frac{aT^4}{3} \gg \frac{\rho kT}{\mu m_p}\implies T^3 \gg \rho \frac{3k_B}{\mu m_p a}.
    \]
    For a ``typical'' density of $10^{-5} \;{\rm g/cm^3}$, this requires that
    \[
    T \gtrsim 10^6\;\left(\frac{\rho}{10^{-5}\;{\rm g/cm^3}}\right)\; {\rm K}.
    \]
    This means that either or both of these assumptions could fail at low temperatures.
\end{itemize}

\subsubsection{Scalings}

Let's now derive the scalings. We'll start with the surface density and then everything else will follow relatively trivially. We have angular momentum conservation, which requires (equation~\eqref{eq:thin_disk_viscous})
\begin{equation}
\nu \Sigma = \frac{\dot{M}}{3\pi}
\left[1 - \left(\frac{R_{\rm in}}{R}\right)^{1/2}\right]
\;\equiv\; \dot{M}\, f^4,
\end{equation}
where we defined, for compactness,
\[
f^4 = \frac{1}{3\pi}\left[1 - \left(\frac{R_{\rm in}}{R}\right)^{1/2}\right].
\]
Now, I currently have $\Sigma$ in terms of two fundamentals $(\dot{M}, f)$ and $\nu$. We really want to be able to reduce the expression for $\nu$. To do so, we use the \textbf{$\alpha$-disk assumption:}
\[
\nu = \alpha c_s H = \alpha \frac{c_s^2}{\Omega_K}
= \alpha \frac{P}{\rho\,\Omega_K}.
\]
Using vertical hydrostatic equilibrium,
\[
P = \rho \Omega_k^2 H^2,\;\text{and},\; \Sigma = \sqrt{2\pi} \rho H,\;\text{so}\; H = \sqrt{2\pi} P \Sigma^{-1} \Omega_k^{-2}.
\]
Substituting in,
\[
\nu = \sqrt{2\pi} \frac{\alpha}{\Omega_k}\frac{PH}{\Sigma} = 2\pi \frac{\alpha}{\Omega_k^3} \Sigma^{-2} P^2.
\]
This important relation shows that, in any disk, the \textbf{viscosity depends quadratically on the pressure}. 

We now need to find a closure for $P$ in terms of primitive variables. To do this, we'll connect to radiative transfer.
The radiation pressure gives
\[
P = \frac{a}{3}T_c^4.
\]
From radiative diffusion (equation~\eqref{eq:Tc_Teff_tau}) and our dissipation closure,
\[
T_c^4 = \frac{3}{8}\kappa_{\rm es}\Sigma T_{\rm eff}^4,
\qquad
\sigma_{\rm sb}T_{\rm eff}^4 = \frac{3GM\dot{M}}{8\pi R^3}
\left[1 - \left(\frac{R_{\rm in}}{R}\right)^{1/2}\right].
\]
Substituting and simplifying yields
\[
P = \frac{3a}{64\pi \sigma_{\rm sb}} \kappa_{\rm es} \Sigma \frac{GM\dot{M}}{R^3} f^4 = \frac{3}{16\pi c} \kappa_{\rm es} \Sigma \frac{GM\dot{M}}{R^3}f^4.
\]
We can now start combining pieces again. Since we have $P$, we know that
\[
\boxed{
\nu = \frac{2\pi \alpha}{\Omega_K^3} \frac{P^2}{\Sigma^2} = \frac{9G^2}{128\pi c^2}\alpha\kappa_{\rm es}^2 M^{^2} \dot{M}^{2} \Omega_k^{-3} R^{-6} f^8= \frac{9G^{1/2}}{128 \pi c^2} \alpha \kappa_{\rm es}^2 M^{1/2} \dot{M}^2 R^{-3/2} f^8.
}
\]
As such,
\[
\boxed{\Sigma = \frac{\dot{M} f^4}{3\pi \nu} = \frac{128}{27} c^2 G^{-1/2} \kappa_{\rm es}^{-2} \;\;\alpha^{-1} M^{-1/2}\dot{M}^{-1} f^{-4} R^{3/2} 
}
\]
Using the pressure, we can get the scale height:
\[
H = \sqrt{2\pi} \frac{P}{\Sigma} \Omega_k^{-2} = \frac{3}{16\pi c} \sqrt{2\pi} \kappa_{\rm es} \frac{GM\dot{M}}{R^3} f^4 \Omega_k^{-2}.
\]
Simplifying,
\[
\boxed{
H = \frac{3}{16c}\sqrt{\frac{2}{\pi}} \kappa_{\rm es} \dot{M} f^4.
}
\]
The \textbf{effective temperature} is
\[
T_{\rm eff} = \left(\frac{3G}{8\pi \sigma_{\rm SB}}\right)^{1/4} M^{1/4} \dot{M}^{1/4} R^{-3/4} f.
\]
The \textbf{core temperature} is
\[
T_{\rm c} = \left(\frac{3}{8}\right)^{1/4} \kappa_{\rm es}^{1/4} \Sigma^{1/4} T_{\rm eff} = \left(\frac{9G}{64 \pi \sigma_{\rm SB}}\right)^{1/4}\kappa_{\rm es}^{1/4} \Sigma^{1/4} M^{1/4} \dot{M}^{1/4} R^{-3/4} f.
\]
Using our equation for $\Sigma$, we find
\[
\Sigma^{1/4}
=
\left(\frac{128}{27}\right)^{1/4}
c^{1/2}
G^{-1/8}
\kappa_{\rm es}^{-1/2}
\alpha^{-1/4}
M^{-1/8}
\dot{M}^{-1/4}
f^{-1}
R^{3/8}.
\]
Substituting into the midplane temperature expression,
\[
T_{\rm c}
=
\left(\frac{9G}{64 \pi \sigma_{\rm SB}}\right)^{1/4}
\kappa_{\rm es}^{1/4}
\Sigma^{1/4}
M^{1/4} \dot{M}^{1/4} R^{-3/4} f,
\]
we obtain
\[
\boxed{
T_{\rm c}
=
\left(\frac{9G}{64 \pi \sigma_{\rm SB}}\right)^{1/4}
\left(\frac{128}{27}\right)^{1/4}
c^{1/2}
G^{-1/8}
\kappa_{\rm es}^{-1/4}
\alpha^{-1/4}
M^{1/8}
R^{-3/8}
f^{-1}.
}
\]
Moving on to the \textbf{optical depth}, 
\[
\tau = \frac{1}{2}\kappa_{\rm es}\Sigma,
\]
thus
\[
\boxed{
\tau
=
\frac{64}{27}
c^2
G^{-1/2}
\kappa_{\rm es}^{-1}
\alpha^{-1}
M^{-1/2}
\dot{M}^{-1}
f^{-4}
R^{3/2}.
}
\]
We can, of course, use $\Sigma$ and $H$ to derive the midplane density as
\[
\rho = \frac{\Sigma}{\sqrt{2\pi}H},
\qquad 
H = \frac{3}{16c}\sqrt{\frac{2}{\pi}} \kappa_{\rm es} \dot{M} f^4,
\]
so
\[
\rho =
\frac{
\frac{128}{27} c^2 G^{-1/2} \kappa_{\rm es}^{-2}
\alpha^{-1} M^{-1/2} \dot{M}^{-1} f^{-4} R^{3/2}
}{
\frac{3}{8c} \kappa_{\rm es}\dot{M} f^{4}
}.
\]
After simplification,
\[
\boxed{
\rho =
\frac{1024}{81}
c^3
G^{-1/2}
\kappa_{\rm es}^{-3}
\alpha^{-1}
M^{-1/2}
\dot{M}^{-2}
f^{-8}
R^{3/2}.
}
\]
Finally, we can compute the radial inflow speed as 
\[
v_r = -\frac{\dot{M}}{2\pi R \Sigma},
\]
so using $\Sigma$,
\[
v_r 
= 
-\frac{\dot{M}}{2\pi R}
\left(\frac{27}{128}\right)
c^{-2}
G^{1/2}
\kappa_{\rm es}^2
\alpha
M^{1/2}
\dot{M}
f^{4}
R^{-3/2}.
\]
Thus,
\[
\boxed{
v_r
=
-\frac{27}{256\pi}
\alpha\,\kappa_{\rm es}^2\,
c^{-2}\,
G^{1/2}\,
M^{1/2}\,
\dot{M}^{2}
R^{-5/2}
f^{4}.
}
\]

Thus, for \textbf{standard scalings},
\begin{equation}
    \boxed{
    \begin{aligned}
        \Sigma =& 854 \;\alpha_{-1}^{-1} M_1^{-1/2} \dot{M}_{16}^{-1} R_{7}^{3/2} f^{-4}\;{\rm g \; cm^{-2}}\\
        \nu = &1.1\times10^{10} \;\alpha_{-1} M_1^{1/2} \dot{M}_{16}^{2} R_{7}^{-3/2} f^8\;{\rm cm^2\;s^{-1}}\\
        H =&1.8\times 10^4\;\dot{M} f^4\;{\rm cm}\\
        T_C =& 1.36\times10^{7}\;\alpha_{-1}^{-1/4} M_{1}^{1/8} R_{7}^{-3/8} f^{-1}\;{\rm K}\\
        \tau =& 1.671\times10^4\;\alpha_{-1}^{-1} M_1^{-1/2} \dot{M}_{16}^{-1} R_{7}^{3/2} f^{-4}\\
        \rho =& 1.8\times10^{-2} \;\alpha_{-1}^{-1} M_{1}^{-1/2} \dot{M}_{16}^{-2} R^{3/2}_{7} f^{-8}\;\;{\rm g/cm^3} \\
        v_r =& 1.6\times10^3 \alpha_{-1} M_1^{1/2} \dot{M}_{16}^2 R_{7}^{-5/2} f^4\;{\rm cm/s}\\
    \end{aligned}
    }
\end{equation}

\subsubsection{Failures of the Thin Disk Assumption}

There is a very important result which emerges from this set of solutions: it is relatively \textbf{easy to violate the thin disk assumption}. Recall that, in Region~A,
\[
H = \frac{3}{16c}\sqrt{\frac{2}{\pi}} \,\kappa_{\rm es}\,\dot{M}\,f^4,
\]
so the scale height is independent of radius and depends linearly on the accretion rate.

It is convenient to express this in terms of the \textbf{Eddington accretion rate}. Using
\[
\dot{M}_{\rm Edd}
= \frac{L_{\rm Edd}}{\eta c^2}
= \frac{4\pi G M}{\kappa_{\rm es}\,\eta\,c}
= \frac{4\pi c R_g}{\kappa_{\rm es}\,\eta},
\qquad
R_g \equiv \frac{GM}{c^2},
\]
we define the dimensionless accretion rate
\[
\dot{m} \equiv \frac{\dot{M}}{\dot{M}_{\rm Edd}}.
\]
Then
\begin{align}
H
&= \frac{3}{16c}\sqrt{\frac{2}{\pi}} \,\kappa_{\rm es}\,\dot{M}\,f^4 \\
&= \frac{3}{16c}\sqrt{\frac{2}{\pi}} \,\kappa_{\rm es}\,\dot{m}\,\dot{M}_{\rm Edd}\,f^4 \\
&= \frac{3}{16c}\sqrt{\frac{2}{\pi}} \,\kappa_{\rm es}\,\dot{m}\,
\left(\frac{4\pi c R_g}{\kappa_{\rm es}\,\eta}\right) f^4 \\
&= \frac{3\pi}{4}\sqrt{\frac{2}{\pi}}\,\frac{\dot{m}}{\eta}\,R_g\,f^4
\end{align}
Thus,
\begin{equation}
H \;\simeq\; 1.9\,\frac{\dot{m}}{\eta}\,R_g\,f^4.
\end{equation}

The \textbf{geometrical thin disk condition} requires $H/R \ll 1$. Using the above,
\begin{equation}
\frac{H}{R}
\simeq 1.9\,\frac{\dot{m}}{\eta}\,\frac{R_g}{R}\,f^4.
\end{equation}
Outside of the \textbf{very innermost disk}, (where $f \sim 1$), this becomes
\begin{equation}
\frac{H}{R}
\simeq 1.9\,\frac{\dot{m}}{\eta}\,\left(\frac{R_g}{R}\right).
\end{equation}
The thin disk approximation breaks down when $H/R \sim 1$, which corresponds to a critical dimensionless accretion rate
\begin{equation}
\dot{m}_{\rm crit}
\sim \frac{\eta}{1.9}\left(\frac{R}{R_g}\right) f^{-4}
\simeq 0.5\,\eta\left(\frac{R}{R_g}\right) f^{-4}.
\end{equation}

If we adopt a characteristic radiative efficiency $\eta \sim 0.1$ and evaluate near the inner disk, say $R \sim 10\,R_g$ and $f \approx 1$, then
\[
\dot{m}_{\rm crit} \sim 0.5 \times 0.1 \times 10 \;\sim\; 0.5.
\]
Thus, \textbf{even for \emph{sub–Eddington} accretion rates,}
\[
\dot{M} \sim 0.5\,\dot{M}_{\rm Edd},
\]
the inner, radiation–pressure dominated regions of the disk are already becoming geometrically thick with $H/R \sim \mathcal{O}(1)$.

This illustrates a key lesson: in the radiation–pressure dominated inner disk,\textbf{ it is relatively easy to violate the thin disk assumption}. Long before the global accretion rate becomes strongly super–Eddington, the inner regions may puff up into a geometrically thick configuration, signaling the breakdown of the standard thin $\alpha$–disk description and the onset of the ``slim'' or advection–dominated regime.


\subsection{Region B: Gas Pressure \& Electron Scattering}
\label{sec:region_b}

We now turn to the next regime of the $\alpha$–disk model, where the disk remains \textbf{electron–scattering dominated} in opacity but \textbf{gas pressure} provides the dominant support against gravity. This typically applies to intermediate radii in luminous accretion disks, beyond the radiation–pressure–dominated zone but before the outer free–free opacity region takes over. In many scenarios, this is a very \textbf{thin region} and may not even appear at all in some solutions.

\subsubsection{Relevance}

As we saw in \textbf{region A}, the assumption of \textbf{electron scattering opacity} breaks down around temperatures of $T\sim 10^6$ or $T\sim10^7$ but the radiation pressure domination requires temperatures around $T\sim 10^8$ depending on the density. Therefore, in some scenarios, there is an \textbf{intermediate regime} in which the disk is \textbf{gas pressure dominated} and \textbf{electron scattered}.

The physical assumptions defining this region are:
\begin{equation}
\boxed{
P \simeq P_{\rm gas} = \frac{\rho k_B T_c}{\mu m_p},
\qquad
\kappa = \kappa_{\rm es} = \text{constant}.
}
\end{equation}
\subsubsection{Scalings}

Let's now derive the scalings. In large part, we can \textbf{borrow from Region A}. We still have the same relationships between $\nu, \rho, \Sigma, H,$ and $P$, so specifically
\[
\nu = \sqrt{2\pi} \frac{\alpha}{\Omega_k}\frac{PH}{\Sigma} = 2\pi \frac{\alpha}{\Omega_k^3} \Sigma^{-2} P^2
\]
still holds. The difference now is that we have a remnant \textbf{pressure contribution} that depends on $\rho$. Thus, we'll instead want to utilize the form
\[
\nu = \alpha c_s H = \alpha \frac{c_s^2}{\Omega_K}
= \alpha \frac{P}{\rho\,\Omega_K}.
\]
As before, we now need to find a closure for $P$ in terms of primitive variables. To do this, we'll connect to radiative transfer.
The \textbf{gas pressure} gives
\[
\frac{P}{\rho} = \frac{k_B T}{m_p \mu}.
\]
From radiative diffusion (equation~\eqref{eq:Tc_Teff_tau}) and our dissipation closure,
\[
T_c^4 = \frac{3}{8}\kappa_{\rm es}\Sigma T_{\rm eff}^4,
\qquad
\sigma_{\rm sb}T_{\rm eff}^4 = \frac{3GM\dot{M}}{8\pi R^3}
\left[1 - \left(\frac{R_{\rm in}}{R}\right)^{1/2}\right].
\]
Substituting and simplifying yields
\[
\frac{P}{\rho} = \frac{k_B}{\mu m_p}\left(\frac{9}{64\pi\sigma_{\rm sb}} \kappa_{\rm es} \Sigma \frac{GM\dot{M}}{R^3} f^4\right)^{1/4}
\]
In this case, we have
\[
\nu = \frac{\alpha}{\Omega_k} \frac{k_B}{\mu m_p}\left(\frac{9}{64\pi\sigma_{\rm sb}} \kappa_{\rm es} \Sigma \frac{GM\dot{M}}{R^3} f^4\right)^{1/4},
\]
and
\[
\Sigma = \frac{\dot{M} f^4}{\nu} = \frac{\Omega_k}{\alpha} f^3 \dot{M}\frac{\mu m_p}{k_B} \left(\frac{9}{64\pi\sigma_{\rm sb}} \kappa_{\rm es} \Sigma \frac{GM\dot{M}}{R^3} \right)^{-1/4}
\]
\[
\Sigma^{5/4} = \frac{\Omega_k}{\alpha} f^3 \dot{M}\frac{\mu m_p}{k_B} \left(\frac{9}{64\pi\sigma_{\rm sb}} \kappa_{\rm es} \frac{GM\dot{M}}{R^3} \right)^{-1/4}
\]
so
\begin{equation}
\boxed{
    \Sigma = \left(\frac{\mu m_p}{k_B}\right)^{4/5} \left(\frac{9\kappa_{\rm es}}{64 \pi\sigma_{\rm sb}}\right)^{-1/5} \alpha^{-4/5} \dot{M}^{3/5}  M^{1/5} G^{1/5} R^{-3/5} f^{12/5}
    }
\end{equation}
Having established the scaling for $\Sigma$, we can recover $\nu$ quickly
\begin{equation}
\boxed{
    \nu = \Sigma^{-1} \dot{M} f^4 = \left(\frac{\mu m_p}{k_B}\right)^{-4/5} \left(\frac{9\kappa_{\rm es}}{64 \pi\sigma_{\rm sb}}\right)^{1/5} \alpha^{4/5} \dot{M}^{2/5}  M^{-1/5} G^{-1/5} R^{3/5} f^{8/5}
    }
\end{equation}
Following this, we have the relation for the scale height in terms of the viscosity and $\alpha$:
\[
\nu = \alpha c_s H = \alpha \sqrt{\frac{P}{\rho}} H,
\]
so,

\begin{equation} 
\boxed{ H = \left(\frac{\mu m_p}{k_B}\right)^{-2/5} \left(\frac{9\kappa_{\rm es}}{64 \pi\sigma_{\rm sb}}\right)^{1/10} \alpha^{-1/10} \dot{M}^{1/5} M^{-7/20} G^{-7/20} R^{21/20} f^{4/5} } 
\end{equation}
We can, of course, use $\Sigma$ and $H$ to derive the midplane density as

\[
\rho = \frac{\Sigma}{\sqrt{2\pi}H} = \frac{1}{\sqrt{2\pi}} \left(\frac{\mu m_p}{k_B}\right)^{6/5} \left(\frac{9\kappa_{\rm es}}{64 \pi\sigma_{\rm sb}}\right)^{-3/10} \alpha^{-7/10} \dot{M}^{2/5}  M^{11/20} G^{11/20} R^{-33/20} f^{8/5}
\]

\rmk{We should complete these scalings at some point.}

Thus, for \textbf{standard scalings},
\begin{equation}
    \boxed{
    \begin{aligned}
        \Sigma =&  422\; \alpha_{-1}^{-4/5} \dot{M}_{16}^{3/5}  M_1^{1/5} R_{10}^{-3/5} f^{12/5}\;{\rm g/cm^2}\\
        \nu =& 2.37 \times 10^{13}\; \alpha_{-1}^{4/5} \dot{M}_{16}^{2/5}  M_1^{-1/5} R_{10}^{3/5} f^{8/5}\;{\rm cm^2/s}\\
        H =&1.8\times 10^4\;\dot{M} f^4\;{\rm cm}\\
        T_C =& 1.36\times10^{7}\;\alpha_{-1}^{-1/4} M_{1}^{1/8} R_{7}^{-3/8} f^{-1}\;{\rm K}\\
        \tau =& 1.671\times10^4\;\alpha_{-1}^{-1} M_1^{-1/2} \dot{M}_{16}^{-1} R_{7}^{3/2} f^{-4}\\
        \rho =& 1.8\times10^{-2} \;\alpha_{-1}^{-1} M_{1}^{-1/2} \dot{M}_{16}^{-2} R^{3/2}_{7} f^{-8}\;\;{\rm g/cm^3} \\
        v_r =& 1.6\times10^3 \alpha_{-1} M_1^{1/2} \dot{M}_{16}^2 R_{7}^{-5/2} f^4\;{\rm cm/s}\\
    \end{aligned}
    }
\end{equation}



\subsection{Region C: Gas Pressure \& Free-Free Scattering}
\label{sec:region_c}

We now turn to the final regime of the $\alpha$–disk model, where the disk becomes susceptible to \textbf{Kramer opacity} but \textbf{gas pressure} provides the dominant support against gravity. This applies to the \textbf{intermediate / outer regions} of the disc and is considered the \textbf{canonical disk solution} by most.

Here the midplane pressure is gas pressure and the Rosseland mean opacity is
set by free--free processes (Kramers law):
\[
\boxed{
P \simeq P_{\rm gas} = \frac{\rho k_B T_c}{\mu m_p}, 
\qquad
\kappa = \kappa_0\,\rho\,T_c^{-7/2},
\quad \kappa_0 \approx 5\times10^{24}\;{\rm cm^2\,g^{-2}\,K^{7/2}}.}
\]

\subsubsection{Scalings}

Fortunately, we do not need to solve these by hand as they are worked out in the original work by \citet{1973A&A....24..337S}. The scalings are solved for in much the same manner as we did in the previous cases but with relevant modifications for the corresponding changes in the pressure and opacity prescription.

The resulting scaling relations are 
\begin{equation}
\boxed{
\begin{aligned}
\Sigma &\;=\;5.2\;
\alpha^{-4/5}\,
\dot{M}_{16}^{7/10}\,
M_1^{1/4}\,
R_{10}^{-3/4}\,
f^{14/5}\quad [{\rm g\,cm^{-2}}],
\\[6pt]
T_c &\;=\; 1.4 \times 10^{4}\;
\alpha^{-1/5}\,
\dot{M}_{16}^{3/10}\,
M_1^{1/4}\,
R_{10}^{-3/4}\,
f^{6/5}\quad [{\rm K}],
\\[6pt]
H &\;=\; 1.7\times10^{8}\;
\alpha^{-1/10}\,
\dot{M}_{16}^{3/20}\,
M_1^{-3/8}\,
R_{10}^{9/8}\,
f^{3/5}\quad [{\rm cm}],
\\[6pt]
\rho &\;=\; 3.1\times10^{-8}\;
\alpha^{-7/10}\,
\dot{M}_{16}^{11/20}\,
M_1^{5/8}\,
R_{10}^{-15/8}\,
f^{11/5}\quad [{\rm g\,cm^{-3}}],
\\[6pt]
\tau &\;=\; 190\;
\alpha^{-4/5}\,
\dot{M}_{16}^{1/5}\,
f^{4/5},
\\[6pt]
\nu &\;=\; 1.8\times10^{14}\;
\alpha^{4/5}\,
\dot{M}_{16}^{3/10}\,
M_1^{-1/4}\,
R_{10}^{3/4}\,
f^{6/5}\quad [{\rm cm^2\,s^{-1}}],
\\[6pt]
v_R &\;=\; 2.7\times10^{4}\;
\alpha^{4/5}\,
\dot{M}_{16}^{3/10}\,
M_1^{-1/4}\,
R_{10}^{-1/4}\,
f^{-14/5}\quad [{\rm cm\,s^{-1}}],
\\[6pt]
&\text{with}\;\; f^4 \;\equiv\; 1-\sqrt{\frac{R_{\rm in}}{R}}.
\end{aligned}
}
\end{equation}

\subsection{The Transitions Between Regions}
\label{sec:region_transitions}
The boundaries between the different $\alpha$–disk regimes occur where one physical contribution overtakes another. Specifically:
\begin{enumerate}
    \item The transition between \textbf{radiation–pressure} and \textbf{gas–pressure} dominance is obtained from $P_{\rm rad} = P_{\rm gas}$.
    \item The transition between \textbf{electron–scattering} and \textbf{free–free (Kramers)} opacity is found from $\kappa_{\rm es} = \kappa_{\rm ff}$.
\end{enumerate}
Each of these boundaries defines a locus in the $\dot{M}$–$R$ plane, as illustrated in Figure~\ref{fig:accretion_regimes}. These curves determine where the physical assumptions underlying Regions~A–C change. In this section, we'll \textbf{assume Region C} solutions and determine when they break down and require the other regions to come into play.

\subsubsection{The Opacity Transition}
The first interesting boundary marks where \textbf{the opacity changes from electron scattering to free–free absorption}:
\[
\kappa_{\rm es} = \kappa_{\rm ff} = 5\times10^{24}  \rho T_c^{-7/2}\;{\rm cm^2/g}.
\]
For a standard hydrogen abundance of $X$, the \textbf{opacity ratio} is
\[
\chi_{\rm opacity} = \frac{\kappa_{\rm ff}}{\kappa_{\rm Kramer}} = 4\times 10^{-26}\; (1+X) \rho^{-1} T_c^{7/2}.
\]
In the \textbf{outer region (Region C)}, we find that
\[
\chi_{\rm opacity} = 4\times 10^{-4} \;(1+X) \dot{M}_{16}^{1/2} M_1^{1/4} R_{10}^{-3/4} f^2
\]
Equivalently,
\[
R_{\rm opacity} = 2.5\times10^{7} \dot{M}_{16}^{2/3}\;M_1^{1/3} f^{8/3}\;{\rm cm}.
\]
For a \textbf{solar mass accretor}, the gravitational radius is
\[
R_g \sim 1.4\times10^{5}\;{\rm cm},
\]
so we see this transition around $100 R_g$ in the disk. Within this region, we need to use regions \textbf{B} and \textbf{A} instead.
\begin{remark}
    For a \textbf{Schwarschild blackhole}, this is pretty easy to get because $r_{\rm isco} \sim 6 R_g$. Even for a \textbf{maximally spinning black hole}, we can still easily get this sort of a disk since $r_{\rm isco} \sim 9 R_g$. For accretion disks around white dwarfs, we \textbf{certainly will not} have this transition because the radius far exceeds the boundary for the opacity boundary unless there is a really extreme accretion rate.
\end{remark}

\begin{bigidea}
    Inside of accretion disks around \textbf{compact objects}, there is typically a regime in which the \textbf{opacity} transitions from \textbf{Kramer Opacity} to \textbf{Electron Scattering} and the disk structure changes from the \textbf{Region C} solution to \textbf{Regions A/B}.
\end{bigidea}

\subsubsection{The Radiation Pressure Transition}

The second important transition is the transition from radiation pressure to gas pressure. Radiation pressure \textbf{dominates} when $P_{\rm rad } \gg P_{\rm gas}$. We know that
\[
P_{\rm rad} = \frac{a}{3}T_c^4
\quad\text{and}\quad
P_{\rm gas} = \frac{\rho k_B T_c}{\mu m_p}.
\]
The ratio of these two quantities is
\[
\chi_{\rm rad} = \frac{P_{\rm rad}}{P_{\rm gas}} = \frac{4\sigma_{\rm SB} m_p \mu}{3c  k_B} T_c^3 \rho^{-1}.
\]
With standard scalings from \textbf{Region C}, this ratio is
\begin{equation}
    \boxed{
    \chi_{\rm rad} \sim 2.8 \times 10^{-3} \alpha ^{1/10} \dot{M}_{\rm 16}^{7/10} R_{\rm 10}^{-3/8} f^{7/5}.
    }
\end{equation}
Clearly, for \textbf{sufficiently small radii}, we can get $\chi_{\rm rad}$ close to $1$; however, the resulting radius is around $1\times10^{3}\;{\rm cm}$, which is clearly within the other regime, meaning we cannot trust these scalings.
In general, these scenarios are \textbf{only relevant} for neutron stars and black holes and only at relatively \textbf{high accretion rates}.

\subsubsection{The Temperature Transition}

The final relevant transition to keep track of is the fact that at \textbf{large radii}, the disk solution may produce temperatures for which the opacity is no longer modeled by Kramer's opacity because of how low the temperature is.

Since the disk's material will cease to be highly ionized at $T \sim 10 {\rm eV} \sim 10^4\;{\rm K}$, we will need to allow for other behaviors beyond 
\[
T_c \le 10^4 \implies 10^4 {\rm K} \ge 1.4 \times 10^{4}\;
\alpha^{-1/5}\,
\dot{M}_{16}^{3/10}\,
M_1^{1/4}\,
R_{10}^{-3/4}\,
f^{6/5}\quad {\rm K},
\]
This corresponds to a \textbf{critical radius}, such that
\[
R^{3/4}_{\rm crit-temp,10} = 1.4\; \alpha^{-1/5}\,\dot{M}_{16}^{3/10}\,M_1^{1/4}\,f^{6/5},
\]
Thus,
\[
R_{\rm crit-temp} = 1.5\times10^{10}\; \alpha^{-4/15} \dot{M}_{16}^{4/10} M_1^{1/3} f^{8/5}\;{\rm cm}.
\]
\begin{figure}
    \centering
    \includegraphics[width=0.9\linewidth]{Pictures/figures/accretion_regimes.png}
    \caption{
    The division of the standard Shakura–Sunyaev $\alpha$–disk into its analytic regimes.
    The thick curves show the loci where $P_{\rm rad}=P_{\rm gas}$ and
    $\kappa_{\rm es}=\kappa_{\rm ff}$,
    separating Regions~A, B, and C.  The horizontal line near $\dot M_{\rm Edd}$ marks the
    Eddington accretion rate for a neutron star and a black hole.}
    \label{fig:accretion_regimes}
\end{figure}

\section{Observations of Thin Disks}

Our final discussion point for this topic will be the confrontation of these disks with observation. As we have discussed previously, there is a reasonably small domain of systems in which the "classical" thin disk can be applied. Let's look briefly at the conditions for the \textbf{Region C disk} that we've derived earlier. In order to be within region $C$, we need to maintain a \textbf{Kramer's Opacity} dominated scenario, so
\[
R_{\rm opacity} = 2.5\times10^{7} \dot{M}_{16}^{2/3}\;M_1^{1/3} f^{8/3}\;{\rm cm}.
\]
Now, we can choose stellar accretion onto either BH, WD, or NS. Regardless, we know that the emission from the disk is dominated by emission in the \textbf{UV/IR} for a standard disk: $T\sim 10^5 \;{\rm K}$ for fiducial values of the parameters. We therefore want \textbf{small donor stars} so that we do not interrupt the emission from the disk. 

If we have a compact object donor, we know that the gravitational radius will be a good proxy for the inner radius of the disk. This corresponds to $R_{\rm min} \sim 10^5{\rm cm}$, so there will be an \textbf{opacity transition} away from the \textbf{Region C disk}. We therefore would want to potentially use a white dwarf: Thus, we use a \textbf{cataclysmic variable star}.

To test the assumption, we can observe a transit of the donor in front of the disk in different wave bands. Because the different wavebands select for different radii in the disk, the widths of the lightcurve dips will differ and provide a way to measure $T(R)$.


