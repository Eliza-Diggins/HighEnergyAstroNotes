Even more ubiquitous in astrophysics than accretion onto a single source is accretion between binary systems. This is also where we are able to learn the most about accretion because, by their nature, binary systems reveal more about themselves than individual systems do.
\par
In this section of notes, we'll work through the detailed geometry and dynamics of binary accretion mechanisms, which will lead us toward the theory of disk accretion.

\section{Dynamics of Binary Systems}

Before we dive into the fluid dynamics of binary accretion, we'll first discuss the dynamics of these systems from a gravitational standpoint. There are several important results and will come from this and help along the way later on.

\subsection{The Two-Body Problem}
To understand accretion in binary systems, it’s helpful to briefly recall the dynamics of the classical \textbf{two-body problem}.  We consider two point masses, $M_1$ and $M_2$, interacting only through gravity. Their positions relative to an inertial frame are ${\bf r}_1$ and ${\bf r}_2$, and the vector separating them is  
\[
{\bf r} = {\bf r}_2 - {\bf r}_1.
\]

The equations of motion are
\[
M_1 \ddot{\bf r}_1 = G \frac{M_1 M_2}{r^3} {\bf r}, 
\qquad
M_2 \ddot{\bf r}_2 = -G \frac{M_1 M_2}{r^3} {\bf r}.
\]
Adding these two gives conservation of the \textbf{center of mass (COM)}:
\[
M_1 \ddot{\bf r}_1 + M_2 \ddot{\bf r}_2 = 0 
\quad \Rightarrow \quad 
\ddot{\bf R}_{\rm COM} = 0,
\]
where 
\[
{\bf R}_{\rm COM} = \frac{M_1 {\bf r}_1 + M_2 {\bf r}_2}{M_1 + M_2}.
\]
It is therefore natural to work in the COM frame, where ${\bf R}_{\rm COM} = 0$.  
Defining the \textbf{reduced mass} 
\[
\mu = \frac{M_1 M_2}{M_1 + M_2},
\]
the relative motion reduces to a single particle of mass $\mu$ moving under the potential of the total mass $M = M_1 + M_2$:
\[
\mu \ddot{\bf r} = - \frac{G M_1 M_2}{r^3} {\bf r}
\quad \Rightarrow \quad 
\ddot{\bf r} = - \frac{G M}{r^3} {\bf r}.
\]
Thus, the two-body problem reduces to the motion of a single body in a central potential.  
The orbits are conic sections—elliptical for bound systems—with angular momentum per unit mass
\[
{\bf L} = {\bf r} \times \dot{\bf r},
\]
and total specific energy
\[
E = \frac{1}{2}\dot{r}^2 - \frac{GM}{r}.
\]
While this is a fairly elementary problem to solve and is well known to most astronomy students, it is worth noting that our standard intuition for orbits must be stretched somewhat in this case. We now recognize that it is the \textbf{separation} between the two bodies which follows a standard elliptical orbit and that we must be careful not to mistake our intuition for planetary orbits with those here.
\par
Now, the \textbf{main observable of a binary is the period}, therefore, we will use that to determine other properties. Using \textbf{Kepler's Law},
\[
a^3 = \frac{GM_{\rm tot}}{4\pi^2} P^2.
\]
\rmk{It's worth remembering here that this is the \textit{relative orbit} in the equivalent 1-body problem. Thus, its the proper distance \textit{between the companions}, but it is NOT the orbital radius.}
We will also find it convenient to introduce the \textbf{mass ratio} $q = M_2/M_1$. We may write $a$ in terms of the mass ratio and the period as
\begin{equation}
\label{eq:binary_orbit_semi_major_axis}
\boxed{
    a = \left(\frac{G}{4\pi^2}\right)^{1/3} M_1^{1/3}(1+q)^{1/3} P^{2/3}.
}
\end{equation}
In units of solar masses, this reduces to
\begin{equation}
    \label{eq:binary_orbit_semi_major_axis_united}
    \boxed{
    a = \begin{cases}
        1.5\times 10^{13} \left(\frac{M_1}{M_\odot}\right)^{1/3} (1+q)^{1/3} P_{\rm years}^{2/3}\; {\rm cm},&\\
                2.9\times 10^{11} \left(\frac{M_1}{M_\odot}\right)^{1/3} (1+q)^{1/3} P_{\rm days}^{2/3}\; {\rm cm},&\\
                        3.5\times10^{10} \left(\frac{M_1}{M_\odot}\right)^{1/3} (1+q)^{1/3} P_{\rm hrs}^{2/3}\; {\rm cm},&\\
    \end{cases}
    }
\end{equation}
\begin{remark}
    Notice that $q > 0$ no matter what, so we always know that leaving out the $1+q$ term provides a \textbf{lower bound} on $a$. In principle, $q \to \infty$ can occur, in which case $a \to \infty$; however, this isn't really relevant in practice.
\end{remark}

In the COM frame, the distance to each of the companions is determined by the fact that ${\bf r} = {\bf r}_1 - {\bf r}_2 = a$, which means that (\rmk{easy enough to work this out from the definition of the COM}),
\[
{\bf r}_1 = \frac{M_2}{M_1+M_2} {\bf r},\; {\bf r}_2 = - \frac{M_1}{M_1+M_2} {\bf r}.
\]
In terms of $q$, 
\[
{\bf r}_1 = \frac{1}{1+q} {\bf r},\; {\bf r}_2 = - \frac{q}{1+q} {\bf r}.
\]


\subsection{Tidal Forces in Binary Systems}

One important feature of accreting binaries is that they are \textbf{close binaries}: therefore, we cannot ignore the role that tidal forces play in the dynamics. This serves two really important purposes which we will exploit in developing the theory:

\begin{enumerate}
    \item Tidal forces \textbf{circularize binary orbits}, and
    \item tidal forces \textbf{synchronize axial rotations}.
\end{enumerate}

To understand these phenomena, we consider a qualitative picture: Let $M_1$ and $M_2$ be the masses of two stars in binary orbit. If $M_1$ experiences tidal distortion, it will bulge along the \textit{line of centers}. Now, the star is also rotating, which means that the bulge will need to shift backward to align with the line of centers. In a perfectly elastic star, this shift could happen rapidly, but that is not generally the case. 
\par
As a result, the tidal bulge tends to either lead or follow the \textit{line of centers}, which means that the COM of the star is shifted either forward or backward in the orbit. If it is shifted backward, then the companion will pull it forward, slowing the rotation and increasing the angular momentum in the orbit to match. Likewise, if it is shifted forward, it will be pulled backward and the rotation will increase and angular momentum will be pulled out of the orbit. Internally, this rearranging of material will lead to dissipative processes which serve to reduce the overall energy of the orbit to the minimum possible for a given fixed angular momentum. This will correspond to synchronous rotation of the binaries and circular orbits.
\par
In general, this process is quite efficient:
\[
t_{\rm tide} \sim \left(\frac{a}{R}\right)^5\;\text{or faster},
\]
where $a$ is the semi-major axis of the orbit and $R$ is the radius of the tidally effected star. It is therefore \textit{generally} the case that these evolutions occur on time scales short relative to the timescale of accretion.
\rmk{(This is not \textit{always} the case: there are exceptions!)}
\paragraph{Implications}
The result of this is that, for most relevant systems, we can treat the problem of accretion in a somewhat simpler scenario: we have circular, Keplerian orbits and the systems are tidally locked. This means that in a frame centered on the COM and co-rotating with the orbit, the bodies appear to be \textbf{stationary}.

\subsection{The Restricted 3-Body Problem}

Having established the dynamics of the two companion stars, we now ask a natural question:
\vspace{0.25cm}
\begin{center}
    \textit{What is the motion of a test particle released in the combined gravitational field of the two companions?}
\end{center}
\vspace{0.25cm}

To study this problem, we work in the \textbf{co-rotating center-of-mass frame}. In this frame, the two companions remain \textbf{fixed} at their orbital radii and appear \textbf{non-rotating}. The tradeoff is that we must introduce the \textbf{fictitious forces} associated with the non-inertial frame. Defining the rotational velocity vector
\[
\boldsymbol{\Omega} = \Omega \hat{\bf z} 
= \left(\frac{GM}{a^3}\right)^{1/2}\hat{\bf z},
\]
the fictitious accelerations are:
\vspace{0.5cm}
\begin{enumerate}
    \item \textbf{Coriolis Force}: ${\bf F}_{\rm cor} = -2m\, \boldsymbol{\Omega} \times \dot{\bf r}$,
    \item \textbf{Centrifugal Force}: ${\bf F}_{\rm cent} = -m\, \boldsymbol{\Omega} \times (\boldsymbol{\Omega} \times {\bf r})$,
    \item \textbf{Euler Force}: ${\bf F}_{\rm Euler} = -m\, \dot{\boldsymbol{\Omega}} \times {\bf r}$.
\end{enumerate}
\vspace{0.5cm}

For binaries of constant orbital period, $\dot{\boldsymbol{\Omega}}=0$, so the Euler force can be neglected. The Euler equation in the rotating frame becomes
\[
\frac{D{\bf u}}{Dt} = - \frac{\nabla P}{\rho} - \nabla \Phi_{\rm Roche} - \underbrace{2 \boldsymbol{\Omega} \times \dot{\bf r}}_{\text{Coriolis Force}},
\]
where $\Phi_{\rm Roche}$ is the combined gravitational and centrifugal potential, called the \textbf{Roche Potential}:
\begin{equation}
    \label{eq:roche_potential}
    \boxed{
    \Phi_{\rm Roche} =
    - GM_2 \left[\frac{1}{q|{\bf r} - {\bf r}_1|} + \frac{1}{\,|{\bf r}-{\bf r}_2|}\right]
    - \tfrac{1}{2}\left|\boldsymbol{\Omega}\times{\bf r}\right|^2
    }.
\end{equation}

\begin{figure}
    \centering
    \includegraphics[width=0.75\linewidth]{Pictures/figures/roche_potential.png}
    \caption{The Roche potential in the orbital plane for a representative mass ratio. Equipotential contours illustrate the balance between gravitational and centrifugal forces in the co-rotating frame.}
    \label{fig:roche_potential}
\end{figure}

\subsubsection{Lagrange Points}

Inspection of the Roche potential (Figure~\ref{fig:roche_potential}) reveals the existence of \textbf{five equilibrium points}, known as the \textbf{Lagrange points}. At these locations, the effective force vanishes in the co-rotating frame. Of these, the most important for mass transfer is the inner point, $L_1$, located between the two stars. This is a \textbf{saddle point} of the potential and defines the boundary between the two \textbf{Roche lobes}. Material within a lobe is gravitationally bound to its host star, while material near $L_1$ can pass through to the companion. 

In two dimensions, the Roche lobes take the form of a figure-eight, while in three dimensions they resemble a dumbbell (Figure~\ref{fig:roche_lobes}). The geometry of these lobes determines when a star will begin to lose mass through Roche lobe overflow.

\begin{figure}
    \centering
    \includegraphics[width=0.75\linewidth]{Pictures/figures/roche_lobes.png}
    \caption{The Roche lobes of two stars in a binary, shown as equipotential surfaces of the Roche potential. The $L_1$ point at the intersection of the lobes sets the critical condition for Roche lobe overflow.}
    \label{fig:roche_lobes}
\end{figure}

\section{Roche Lobe Overflow}

We now discuss the nature of material accreted onto a companion from a progenitor star. Consider the scenario given above when the radii of the individual companions ($R_1$ and $R_2$) are each much, much smaller that their respective Roche Lobes. From the Euler Equation,
\[
\frac{D{\bf u}}{Dt} = - \frac{\nabla P}{\rho} - \nabla \Phi_{\rm Roche} - 2 \boldsymbol{\Omega} \times {\bf u},
\]
we see that in steady state,
\[
0 = - \frac{\nabla P}{\rho} - \nabla \Phi_{\rm Roche},
\]
which implies that the surface (defined by $\nabla P = 0$) will coincide with one of the inner most equipotentials deep within the respective Roche lobe. What do we learn from this? \textbf{For binaries which are each deep within their own Roche Lobes, there is no impetus for mass transfer.} Such binaries are called \textbf{detached binaries}.
\par
Now we might consider another scenario where one companion (the so-called \textbf{secondary} or \textbf{non-accretor}) expands (due to any number of relevant processes) such that it begins to fill its Roche Lobe. Here's the critical idea:
\vspace{0.25cm}
\begin{center}
    When material on the surface (which is on equipotentials of $\Phi_{\rm Roche}$) begins to fill the Roche-Lobe, that material will lie close to the $L_1$ point. In turn, perturbations can \textbf{displace material across $L_1$ into the corresponding Roche Lobe of the primary WITHOUT needing additional energy input.} 
\end{center}
\vspace{0.25cm}
Such binaries are called \textbf{semi-detached binaries} and are able to efficiently transfer mass across the $L_1$ point. Another interesting scenario occurs when both objects fill their Roche-Lobes, creating a so-called \textbf{contact binary}.
\begin{remark}
    It will be seen that the rate of material flow is quite fast. As such, we rarely get a companion which is much larger than its own Roche Lobe.
\end{remark}

\subsection{Geometry of Binary Accretion}

Before we can work out the details of binary accretion, we'll need to do a bit of geometry. Our goal here is to answer the following general questions:
\vspace{10pt}
\begin{enumerate}
    \item \textbf{Where} is the L1 point relative to the primary in a binary system?
    \item \textbf{How big} is the Roche Lobe for a given mass?
    \item \textbf{What} determines the size and shape of the Roche Lobe?
\end{enumerate}
\vspace{10pt}
\medskip
\par
Looking again at the equation for the Roche Potential, we have \eqref{eq:roche_potential}
\[
    \Phi_{\rm Roche} =
    - GM_2 \left[\frac{1}{q|{\bf r} - {\bf r}_1|} + \frac{1}{\,|{\bf r}-{\bf r}_2|}\right]
    - \tfrac{1}{2}\left|\boldsymbol{\Omega}\times{\bf r}\right|^2.
\]
If we imagine that the binary orbit is circular, then their distance is determined unambiguously from their period (or, conversely, their period is determined from their distance). As such, we see that there are only really two major ingredients in determining the scale and shape of the Roche Lobes:
\vspace{10pt}
\begin{enumerate}
    \item The \textbf{separation} determines a scale for the Roche Geometry,
    \item The \textbf{mass ratio} determines the shape of the geometry.
\end{enumerate}
\vspace{10pt}
The implication here is quite clear: we \textbf{should} be able to say a great deal about the geometry of the Roche System without a direct reference to anything other than $q$. Unfortunately, it is generally a task requiring numerical solution, but there are some quasi-analytic formulae which will be relevant to our discussion of these phenomena.

\subsubsection{The Roche Lobe Radius}

One such option is the \textbf{Eggleton Formula}, which dictates the \textbf{shape of the Roche Lobe} as a function of the mass ratio $q$. It is a numerical fit to computation data taking the form:
\begin{equation}
\label{eq:eggleton_formula}
\boxed{
    \frac{R_2}{a} = \frac{\alpha q^{2/3}}{\beta q^{2/3} + \log(1+q^{1/3})}
    }
\end{equation}
where $\alpha = 0.49$, and $\beta = 0.6$. \rmk{We can get at $R_1$ from replacing $q$ with $q^{-1}$.} Here, $R_2$ is the \textit{effective radius} of the lobe defined as the radius of a sphere with the \textit{same volume as the lobe.} There are also two other approximations worthy of mention:

\begin{enumerate}
    \item $(0.1 \le q \le 0.8)$: We can use a simplified version from \citet{1971ARA&A...9..183P}:
    \begin{equation}
        \label{eq:paczynski_lobe_radius}
        \boxed{
        \frac{R_2}{a} = \frac{2}{3^{4/3}}\left(\frac{q}{1+q}\right)^{1/3} = 0.462\left(\frac{M_2}{M_1+M_2}\right)^{1/3}.}
    \end{equation}
This proves to be a \textbf{massively important} equation because it permits us to discuss many different scaling relationships for different types of primaries and secondaries.
    \item $(0.03 < q < 1)$ also allows
\[
\frac{R_1}{R_2} = \left(\frac{M_1}{M_2}\right)^{0.45}.
\]

\end{enumerate}

\subsubsection{The Location of the Lagrange Point}

We are also interested in the distance between the \textit{primary} (the accretor) and the Roche-Lobe critical point. To good accuracy, a fitted formula suffices \citep{1964BAICz..15..165P}:
\begin{equation}
    \label{eq:roche_overflow_distance}
    \frac{b_1}{a} = 0.5 - 0.227 \log q,
\end{equation}
where $a$ is the semi-major axis (the distance between the primary and the secondary due to circularization) and $b_1$ is the distance between the primary and the Roche point. Clearly, for $q = 1$, the Roche point is right between the primary and the secondary. For $q \gg 1$ (corresponding to a low mass accretor and high mass donor), the overflow point get \textbf{closer to the primary} since the secondary can bind material more effectively. In the opposite scenario, $q \ll 1$, the overflow point gets closer to the secondary. To \textbf{first order} in $q$, this is
\[
\frac{b_1}{a} \approx 0.5 - 0.227(q-1).
\]
\subsection{Scaling Relations for Roche Lobe Overflow}

One of the most useful aspects of equation~\eqref{eq:paczynski_lobe_radius} is that it expresses the Roche–lobe radius $R_2$ entirely in terms of the binary separation $a$ and the mass ratio $q = M_2 / M_1$.  
Using Kepler’s Third Law, we can replace the separation by the observable orbital period $P$:
\begin{equation}
a = \left[\frac{G (M_1 + M_2)}{4\pi^2}\right]^{1/3} P^{2/3}.
\end{equation}
Substituting into the Roche–lobe relation gives
\begin{align}
R_2 &\approx a\,\frac{2}{3^{4/3}} \left(\frac{q}{1+q}\right)^{1/3} \\[4pt]
&= \frac{2}{3^{4/3}}\!\left(\frac{G}{4\pi^2}\right)^{1/3}
   P^{2/3} (M_1 + M_2)^{1/3}
   \left(\frac{M_2}{M_1 + M_2}\right)^{1/3} \\[4pt]
&= \frac{2}{3^{4/3}}\!\left(\frac{G}{4\pi^2}\right)^{1/3}
   P^{2/3} M_2^{1/3}.
\end{align}
Thus, to good approximation, the Roche–lobe radius of the donor depends only on its own mass and the binary period.  
Evaluating the constants yields the very useful scaling:
\[
\boxed{
R_2 \simeq 6.2 \times 10^{11}
\left(\frac{P}{{\rm 1\,day}}\right)^{2/3}
\left(\frac{M_2}{M_\odot}\right)^{1/3}
{\rm cm}.
}
\]

\subsubsection{The Period–Density Relation}

Since a Roche–lobe filling star must satisfy this radius–period relation, its mean density follows immediately:
\begin{equation}
\bar{\rho} = \frac{3M_2}{4\pi R_2^3}.
\end{equation}
Substituting for $R_2$, we find
\begin{align}
\bar{\rho} 
   &\sim \frac{3M_2}{4\pi}
      \left[\frac{3^{4/3}}{2}
      \left(\frac{4\pi^2}{G}\right)^{1/3}
      P^{-2/3} M_2^{-1/3}\right]^3 \\[4pt]
   &= \frac{3^5 \pi}{8 G P^2}.
\end{align}
Hence the donor’s mean density depends \emph{only on the orbital period}, not on the stellar mass or composition:
\[
\boxed{
\bar{\rho} \;\approx\; 110\,P_{\rm hr}^{-2}\;{\rm g\,cm^{-3}},
}
\]
where $P_{\rm hr}$ is the orbital period in hours.

\medskip
\noindent
\textbf{Physical meaning:} systems with shorter periods must contain denser donors.  
This simple $P^{-2}$ scaling makes the orbital period a direct diagnostic of the donor’s mean density—and thus its evolutionary state.

\subsubsection{The Mass–Period Relation for Polytropic Donors}

If the donor follows a polytropic structure with index $n$, then
\begin{equation}
R \propto M^{\frac{1-n}{3-n}},
\end{equation}
and the corresponding mean density is
\begin{equation}
\bar{\rho} \propto M^{\tfrac{3n}{3-n}}.
\end{equation}
Since Roche–lobe filling enforces $\bar{\rho} \propto P^{-2}$, we obtain the general
\textbf{mass–period relation}:
\[
\boxed{
P \propto M^{-\tfrac{3n}{2(3-n)}}.
}
\]

\paragraph{Main Sequence Donors}
For main sequence stars, an empirical relation $R \propto M^{3/4}$ is a good approximation.  
Then
\begin{align}
\bar{\rho}
   &= \frac{3M}{4\pi R^3}
   \approx 1.4
   \left(\frac{M}{M_\odot}\right)^{-5/4}
   {\rm g\,cm^{-3}},
\end{align}
and equating with $\bar{\rho} \approx 110\,P_{\rm hr}^{-2}$ gives
\[
P_{\rm hr} \approx 8.8
\left(\frac{M}{M_\odot}\right)^{5/8},
\qquad
\boxed{
\frac{M}{M_\odot} \approx 0.03\,P_{\rm hr}^{8/5}.
}
\]
Because main sequence stars cannot exist below about $0.1\,M_\odot$, this relation implies a lower limit near 
$P_{\rm min} \sim 2\,{\rm hr}$—shorter-period systems must therefore contain evolved or degenerate donors.

\paragraph{Degenerate Donors}
For fully degenerate (non-relativistic) donors, $R \propto M^{-1/3}$, so that
\[
\bar{\rho} \propto M^{2},
\qquad
P \propto M^{-1}.
\]
Hence more massive degenerate donors correspond to shorter orbital periods.  
A typical scaling is
\[
\boxed{
P_{\rm sec} \approx 150
\left(\frac{M}{M_\odot}\right)^{-1},
}
\]
appropriate for low-mass white dwarf donors.

\subsubsection{The Minimum Period in Compact Binaries}

These two contrasting mass–period scalings explain the observed \textbf{period bounce} in compact binaries:
\begin{itemize}
    \item For \textbf{main sequence donors} ($R \propto M^{3/4}$):  
    $P \propto M^{8/5}$, so as mass is lost, the orbital period \emph{decreases}.
    \item For \textbf{degenerate donors} ($R \propto M^{-1/3}$):  
    $P \propto M^{-1}$, so further mass loss causes the period to \emph{increase}.
\end{itemize}
The system therefore evolves toward shorter periods until the donor becomes degenerate, reaches a minimum period, and then expands again.  
This transition defines the \textbf{minimum orbital period}:
\[
P_{\rm min} \sim 1\,{\rm hr}, 
\qquad
M_2 \sim 0.03\text{--}0.04\,M_\odot.
\]

\begin{bigidea}
\textbf{Key Concepts}
\begin{itemize}
    \item The Roche–lobe size depends only on the mass ratio $q$ and separation $a$.
    \item For a Roche–lobe filling star, the \textbf{mean density depends solely on the orbital period}:
    \[
    \bar{\rho} \approx 110\,P_{\rm hr}^{-2}\;{\rm g\,cm^{-3}}.
    \]
    \item Combining this with stellar structure relations $R(M)$ gives characteristic \textbf{mass–period laws}:
    \begin{itemize}
        \item \textbf{Main sequence donors:} $P_{\rm hr} \approx 8.8(M/M_\odot)^{5/8}$.
        \item \textbf{Degenerate donors:} $P_{\rm sec} \approx 150(M/M_\odot)^{-1}$.
    \end{itemize}
    \item The \textbf{period bounce} arises where these two regimes meet, marking the transition from main sequence to degenerate donors.
\end{itemize}
\end{bigidea}

\section{Binary Evolution}

Having now established all of the relevant details of \textbf{Roche-Lobe Overflow}, we are now in a position to describe, in detail, the behavior of binary stars over time when one star fills its RL. This will require a number of detailed analyses under different sets of assumptions, but will lead us to the following information:
\vspace{10pt}
\begin{itemize}
    \item How do orbital parameters $(P, a, \ldots)$ change over time?
    \item How does the geometry of the Roche Lobe $(R_{L})$ change over time?
    \item What implications does this have for mass and angular momentum transfer?
\end{itemize}
\vspace{10pt}

The answers to these questions hinge on only a \textbf{handful of features} of the system:
\vspace{10pt}
\begin{enumerate}
    \item The \textbf{orbital distance} $a$ and \textbf{orbital period} $P$ are dependent on the mass and angular momentum of the system and will change as the system evolves.
    \item The \textbf{Roche-Lobe} geometry, \textbf{stellar evolution} history, and \textbf{accretion history} all play a role in determining how the system \textbf{losses and transfers mass}.
    \item The \textbf{angular momentum} of the system plays a key role in constraining different behaviors of the system.
\end{enumerate}
\vspace{10pt}

\subsection{Governing Principles}

The evolution of a binary system is governed by the exchange (or loss) of two fundamental quantities:
\textbf{mass} and \textbf{angular momentum}.  
Any process that redistributes or removes either quantity inevitably alters the orbital configuration of the system, often in a way that feeds back on the rate and stability of mass transfer itself. As described above, there are really \textbf{three phenomena at play} which we will now discuss individually before moving on to see how they interact.

To begin, we'll introduce a relatively basic piece of machinery: the \textbf{mass loss fraction} $\chi$. We'll assume that the \textbf{accretor} does \textbf{not lose mass} during the evolution and that any mass transfer comes from the \textbf{donor}. That donor will lose mass through a variety of mechanisms, but only $\chi$ of it will accrete onto the accretor:
\[
\dot{M}_1 = - \chi \dot{M}_2,\;\text{and},\; \dot{M}_T = (1-\chi) \dot{M}_2.
\]
As such, we only need to pay attention to changes in $M_2$.

\vspace{10pt}
\subsubsection{Angular Momentum}

For a binary system composed of stars $M_1$ and $M_2$ at a distance $a$ from one another and orbiting with angular velocity $\omega$, the \textbf{angular momentum} is
\[
J = (M_1 a_1^2 + M_2 a_2^2)\,\omega,
\]
where $a_1$ and $a_2$ are the distances of each component from the center of mass:
\[
a_1 = \frac{M_2}{M_1 + M_2}a,
\qquad
a_2 = \frac{M_1}{M_1 + M_2}a.
\]
Substituting these expressions and invoking Kepler’s third law,
\[
\omega^2 a^3 = G(M_1 + M_2),
\]
we obtain the compact and widely used form:
\[
\boxed{
J = M_1 M_2 \left(\frac{G a}{M_1 + M_2}\right)^{1/2}.
}
\]
This is one of the \textbf{fundamental equations of this theory.}

A useful relation is obtained by taking the \textbf{logarithmic derivative} of this expression to find
\begin{equation}
\frac{\dot{a}}{a} = \frac{2\dot{J}}{J} - \frac{2\dot{M}_1}{M_1} - \frac{2\dot{M}_2}{M_2} + \frac{\dot{M}_T}{M_T}
\end{equation}
Replacing $\dot{M}_1 = -\chi \dot{M}_2$ and $\dot{M}_T = (1-\chi)\dot{M}_2$,
\begin{equation}
\boxed{
    \frac{\dot{a}}{a} = \frac{2\dot{J}}{J} + \frac{2(-\dot{M}_2)}{M_2}\left[1-\chi q - \frac{q}{2(q+1)}(1-\chi) \right]}
\end{equation}
We can also write an equivalent statement in terms of the \textbf{orbital period} since
\[
P^2 \propto a^3 \implies 2\frac{\dot{P}}{P} = 3 \frac{\dot{a}}{a},
\]
so
\begin{equation}
    \label{eq:ang_mom_period_binary_ev}
    \boxed{
\frac{\dot{P}}{P} = \frac{3\dot{J}}{J} + \frac{3(-\dot{M}_2)}{M_2}\left[1-\chi q - \frac{q}{2(q+1)}(1-\chi) \right]}
\end{equation}
This statement is \textbf{entirely equivalent} to the one above, simply recast in terms of the period.

\subsubsection{Roche Geometry}

Let's now ask the following question: \textbf{what happens to the Roche Lobe during evolution?} Since we already know how $a$ evolves in response to changes in mass and momentum, we can use the \textbf{Paczynski equation} we saw earlier (equation~\eqref{eq:paczynski_lobe_radius}) to deduce the behavior of the Roche Radius.
\par
From equation~\eqref{eq:paczynski_lobe_radius}, we can logarithmically differentiate to find
\[
\boxed{
\frac{\dot{R}_{2,L}}{R_{2,L}} = \frac{\dot{a}}{a} + \frac{\dot{M}_2}{3M_2} -\frac{\dot{M}_T}{3M_T} = \frac{\dot{a}}{a} + \frac{2(-\dot{M}_2)}{M_2}\left[\frac{1}{6} + \frac{q}{1+q} \frac{(1-\chi)}{6}\right],
}
\]
which can then be substituted into \eqref{eq:ang_mom_binary_ev} to obtain
\begin{equation}
\label{eq:roche_lobe_stability}
\boxed{
    \frac{\dot{R}_{2,L}}{R_{2,L}} = \frac{2\dot{J}}{J} + \frac{2(-\dot{M}_2)}{M_2}\left(\frac{5}{6}-\chi q - \frac{q}{q+1}\frac{1-\chi}{3}\right).
}
\end{equation}

\subsubsection{The Radius--Mass Relation}

The final ingredient we require is the response of the donor star to the loss of
mass.  In binary evolution theory, this is quantified through the
\emph{radius--mass exponent}
\[
\boxed{
\zeta_2 \equiv \frac{d \ln R_2}{d \ln M_2},
}
\]
so that
\[
\frac{\dot{R}_2}{R_2}
=
\zeta_2 \frac{\dot{M}_2}{M_2}.
\]
Since mass loss implies $\dot{M}_2 < 0$, the sign of $\zeta_2$ determines whether the
donor \emph{shrinks} ($\zeta_2 > 0$) or \emph{expands} ($\zeta_2 < 0$) as material is
stripped away.  This response depends sensitively on the internal structure of the
donor—particularly on whether the envelope is radiative or convective, or whether the
material is degenerate.

It is useful to summarize typical values of $\zeta_2$ for astrophysically relevant
donor classes:

\begin{center}
\renewcommand{\arraystretch}{1.25}
\begin{tabular}{p{4cm}|p{2cm}|p{7cm}}
\hline
\textbf{Donor Type} & \textbf{$\zeta_2$} & \textbf{Qualitative Response} \\
\hline
Low-mass MS (fully convective) 
    & $\approx -1/3$ 
    & Expands under mass loss; strongly unstable \\

Solar-type MS (partially convective)
    & $0$ to $+0.3$
    & Slight shrinkage or near-neutral response; marginal stability \\

Intermediate/high-mass MS (radiative envelope) 
    & $\sim +0.5$--$+1$ 
    & Shrinks significantly under mass loss; typically stable for $q\!\lesssim\!1$ \\

Giant-branch donors (deep convective envelope) 
    & $\ll 0$ (very negative) 
    & Strong expansion; rapid runaway leading to CE evolution \\

Wolf--Rayet / stripped He stars (radiative) 
    & $\sim +1$ 
    & Strong shrinkage; stable if mass ratio is modest \\

Cold white dwarf (degenerate; $R \propto M^{-1/3}$) 
    & $-1/3$ 
    & Expands under mass loss; stable only for $q\!\lesssim\!2/3$ \\

Hot / thermally inflated white dwarf 
    & $-0.5$ to $0$ 
    & Expansion or neutral response; prone to instability \\

Partially degenerate helium donor 
    & $-0.1$ to $+0.2$ 
    & Weak expansion or weak shrinkage; stability depends on $q$ \\

AGB envelopes (very extended, low binding) 
    & $\ll -1$ 
    & Catastrophic expansion; inevitably unstable (CE) \\
\hline
\end{tabular}
\end{center}

\begin{bigidea}
\textbf{
Radiative donors shrink under mass loss ($\zeta_2 > 0$),  
convective donors expand ($\zeta_2 < 0$),  
and degenerate donors also expand ($\zeta_2 \approx -1/3$).}
These structural trends largely determine whether mass transfer proceeds stably
or triggers runaway evolution.
\end{bigidea}


\subsection{Fully Conservative Mass Transfer}

We now return to the full problem of binary evolution under the assumption of
\textbf{fully conservative mass transfer}, meaning both \emph{mass} and \emph{angular
momentum} are conserved:
\[
\boxed{
\dot{M}_1 + \dot{M}_2 = 0, 
\qquad
\dot{J} = 0.
}
\]
This idealization allows us to isolate the intrinsic feedback between the donor's
radius, the Roche--lobe radius, and the orbital separation. This is, in many ways, the \textbf{simplest scenario}, but it is not all that common in realistic systems since there are so many other dynamical effects at play.

In this case, we have the following 3 equations derived above:

\begin{equation}
\begin{aligned}
    \frac{\dot{R_2}}{R_2} &= \frac{\dot{M_2}}{M_2} \zeta_2, &\text{(Donor Response)},\\
        \frac{\dot{R}_{2,L}}{R_{2,L}} &= \frac{2(-\dot{M}_2)}{M_2}\left(\frac{5}{6}- q\right),&\text{(Roche-Lobe Response),}\\
            \frac{\dot{a}}{a} &= \frac{2(-\dot{M}_2)}{M_2}\left[1-q \right], &\text{(Binary Radius Response).}
\end{aligned}
\end{equation}
Let's now determine the various \textbf{regimes of mass transfer} in this scenario! If we begin with a small mass loss $\delta M_2 < 0$ produces the fractional responses
\[
\frac{\delta R_2}{R_2} = \zeta_2 \frac{\delta M_2}{M_2},
\qquad
\frac{\delta R_{2,L}}{R_{2,L}}
= -2\left(\frac{5}{6}-q\right)\frac{\delta M_2}{M_2}.
\]
Since we are interested in whether the donor overfills or detaches from its Roche
Lobe, the key quantity is the change in the ratio $R_2/R_{2,L}$:
\[
\delta\ln\!\left(\frac{R_2}{R_{2,L}}\right)
=
\delta\ln R_2 - \delta\ln R_{2,L}
=
\left[
\zeta_2 + 2\left(\frac{5}{6}-q\right)
\right]\frac{\delta M_2}{M_2}.
\]
For a Roche--lobe filling star, $R_2 = R_{2,L}$ initially.  
\emph{Stability} requires that a small additional mass loss does not cause the
star to overfill its Roche Lobe more deeply, i.e.
\[
\delta\ln(R_2/R_{2,L}) \;\lesssim\; 0
\quad\text{for}\quad
\delta M_2 < 0.
\]
Because $\delta M_2/M_2 < 0$, this translates into the inequality
\begin{equation}
\label{eq:conservative_stability_condition}
\boxed{
\zeta_2 + 2\left(\frac{5}{6}-q\right) \;\ge\; 0
\quad\Longleftrightarrow\quad
\zeta_2 \;\ge\; 2\left(q - \frac{5}{6}\right).
}
\end{equation}
This single expression cleanly encodes the competition between the donor’s
structural response (through $\zeta_2$) and the Roche geometry (through $q$).
For $q > 5/6$, we have an \textbf{contracting Roche Lobe} and therefore want the donor to contract in tandem with the lobe. Likewise, for $q < 5/6$, we have an \textbf{expanding Roche Lobe} and therefore want to have the star expand at a reasonable rate. This leads to a handful of important scenarios, which we can organize most naturally by the mass ratio $q$ and then interpret in terms of the donor structure (i.e.\ $\zeta_2$).

\subsubsection{Stability Regimes in the Fully Conservative Limit}

From the expressions above, 

\paragraph{Scenario A: $\boldsymbol{q < 5/6}$ (Expanding Roche Lobe).}

When $q < 5/6$, the term $\left(\tfrac{5}{6}-q\right)$ is positive, and the
Roche Lobe \emph{expands} as the donor loses mass:
\[
\frac{\dot{R}_{2,L}}{R_{2,L}}
=
2\frac{(-\dot{M}_2)}{M_2}\left(\frac{5}{6}-q\right) > 0.
\]
In other words, mass transfer tends to \emph{relax} the confinement: the lobe moves
outward, making \textbf{contact easier to break rather than harder to maintain.}

From the stability condition~\eqref{eq:conservative_stability_condition}, the
right-hand side,
\[
2\left(q - \frac{5}{6}\right) < 0,
\]
as such, donors that \emph{expand} under mass loss
($\zeta_2 < 0$) can satisfy the stability condition, provided their expansion is
not too violent:
\[
\zeta_2 \;\ge\; 2\left(q - \frac{5}{6}\right).
\]
Physically, the picture for $q<5/6$ is therefore quite benign: the orbit expands
and the Roche Lobe expands with it.  
If the donor does not expand too quickly, contact gradually relaxes and the
system can settle into a quasi-steady mass transfer state on thermal or nuclear
timescales.  
This is the classical regime of \emph{stable, conservative mass transfer}, often
encountered in compact binaries with low-mass donors and more massive accretors.

\paragraph{Scenario B: $\boldsymbol{q > 5/6}$ (Shrinking Roche Lobe).}

When $q > 5/6$, the dynamics become strongly destabilizing.  
The sign of the Roche--lobe response reverses:
\[
\frac{\dot{R}_{2,L}}{R_{2,L}}
=
2\frac{(-\dot{M}_2)}{M_2}\left(\frac{5}{6}-q\right) < 0,
\]
so\textbf{ the Roche Lobe \emph{shrinks} }as mass is removed from the donor.  
In this case, mass transfer acts to \emph{tighten} the constraint on the donor:
as soon as material is lost, the lobe contracts around the star, encouraging even
more overflow.

The stability condition~\eqref{eq:conservative_stability_condition} now demands
\[
\zeta_2 \;\ge\; 2\left(q - \frac{5}{6}\right) > 0,
\]
so the donor must not only \emph{shrink} under mass loss, it must shrink 
\emph{fast enough} to keep up with the shrinking Roche Lobe.  

Thus, while equation~\eqref{eq:conservative_stability_condition} leaves open a
formal window of stability at $q>5/6$ for donors with large positive $\zeta_2$,
\emph{real astrophysical stars} rarely find themselves in this favorable corner
of parameter space.  
For most donors of interest—convective, giant, or degenerate envelopes—the
combination of a shrinking Roche Lobe and an expanding star leads to 
\textbf{catastrophic, dynamically unstable mass transfer}.

\begin{bigidea}
In the fully conservative limit, stability is easiest to achieve \textbf{when the donor is
less massive than the accretor ($q < 5/6$)} and does not expand too violently
under mass loss.  
As $q$ approaches and exceeds $5/6$, the Roche Lobe ceases to help and then
actively \emph{works against} stability.  
For the bulk of realistic donor structures, mass transfer in this high-$q$
regime becomes runaway and quickly evolves toward a common-envelope phase.
\end{bigidea}

\subsubsection{Feedback Processes}

The stability of mass transfer is ultimately governed by a competition between
the donor’s structural response and the changing binary geometry.  These two
components are tightly coupled: any change in one inevitably feeds back on the
other.  This feedback is responsible for the wide range of evolutionary
behaviors observed in interacting binaries, from gentle, thermally regulated
transfer to catastrophic, dynamical runaway.

A particularly important feature of this feedback is the way it shapes the
evolution of the mass ratio $q = M_2/M_1$.  
If the donor is initially more massive than the accretor ($q > 1$), even a
modest amount of mass loss can push the system toward instability.
The reason is straightforward: when $q > 1$, mass transfer from the more massive
to the less massive component tends to \emph{shrink} the orbit.  
A shrinking orbit reduces the Roche--lobe radius, forcing the donor to overfill
its lobe more deeply and accelerating further mass loss.  
The result is a classic \emph{runaway}: the donor sheds mass at an ever-growing
rate, often on a dynamical timescale.  
In many cases the envelope is removed so rapidly that the system plunges toward
a common-envelope phase long before the mass ratio has time to appreciably
change.

However, if the system manages to avoid dynamical disruption---for example,
because the donor contracts efficiently under mass loss, or because only a brief
burst of unstable transfer occurs---the evolution naturally drives the mass ratio
toward unity.  
As $q$ approaches $1$, two stabilizing effects emerge.  
First, the Roche geometry becomes less sensitive to mass loss: the rate at which
the Roche Lobe shrinks decreases sharply, reducing the geometric push toward
runaway.  
Second, the orbital response weakens: the shrinkage of the orbit slows, and the
feedback loop between orbital contraction and Roche--lobe overflow becomes
less severe.

A striking consequence is that \emph{nature prefers} mass-transfer states near
$q \simeq 1$.  
Catastrophic episodes of mass loss tend to drive the system \emph{toward} this
mass-ratio equalization, because the donor loses mass while the accretor gains
it.  
Once $q \sim 1$ is reached, the most violent geometric feedbacks shut off,
leaving the system in a marginally stable configuration.  
From this point forward, the subsequent evolution is determined not by runaway
instability, but by slower processes such as thermal relaxation, nuclear
evolution, or angular-momentum losses.

In this sense, even though dynamical instability is often triggered at
$q \gtrsim 1$, the long-term outcome---if the system survives---is to drive the
binary into a configuration where the two masses are nearly equal, a point at
which the mass transfer becomes far more delicate and self-regulated.

\begin{bigidea}
The essential conclusion is this: in the fully conservative limit, mass transfer
from a more massive donor to a less massive accretor is generically
\textbf{dynamically unstable}.  
The combined response of the Roche geometry and the orbit drives the system
rapidly toward $q \simeq 1$, often through a brief episode of catastrophic mass
loss.  
Once the mass ratio reverses so that $q < 1$, the direction of transfer changes
and the Roche Lobe begins to \emph{expand} with mass loss.  
Stable mass transfer can then occur, but only if the donor's own structural
response---typically some degree of expansion---keeps pace with the widening of
the Roche Lobe.
\end{bigidea}

\subsection{Conservative Mass Transfer}

Up to this point, we have examined the simplest possible scenario:
\textbf{fully conservative mass transfer}, in which both mass and angular
momentum are conserved.  Real astrophysical binaries, however, are almost 
never so idealized.  Even if the mass transfer remains \textbf{conservative},
that is, even if no material is lost from the system, the binary will still lose
orbital angular momentum through a variety of physical mechanisms.

These angular momentum losses play a decisive role in shaping the stability and
rate of mass transfer.  They can \emph{trigger} mass transfer by contracting the
orbit until the donor fills its Roche Lobe, or \emph{stabilize} a configuration
that would otherwise evolve unstably.  We begin with a brief overview of the
dominant angular momentum sinks before turning to gravitational radiation, the
most important mechanism in compact binaries.

\subsubsection{Important Mechanisms of Angular Momentum Loss}

\textbf{Angular momentum loss (AML)} shrinks the orbital separation $a$ and 
therefore contracts the Roche Lobe, tightening the confinement of the donor.  
The principal AML mechanisms are:

\begin{enumerate}
    \item \textbf{Gravitational Wave Radiation (GWR).}
    General relativity predicts that gravitational waves extract angular
    momentum at a rate
    \[
        \dot{J}_{\rm GW}
        = -\frac{32}{5}\frac{G^{7/2}}{c^5}
          \frac{M_1^2 M_2^2 M_T^{1/2}}{a^{7/2}}.
    \]
    GWR dominates in ultracompact binaries and double–degenerate systems.

    \item \textbf{Magnetic Braking (MB).}
    For main–sequence stars with convective envelopes, magnetized winds 
    remove angular momentum approximately as
    \[
        \dot{J}_{\rm MB} \propto -\omega^3 R_2^4,
    \]
    where $\omega$ is the tidally locked spin of the donor.  
    MB is a primary driver in cataclysmic variables and many low–mass X-ray binaries.

    \item \textbf{Tidal Dissipation and Structural Evolution.}
    Changes in the donor's internal structure or tidal coupling between spin 
    and orbit can act as effective sinks or sources of angular momentum.
    These processes typically operate on long timescales and modulate, rather
    than dominate, mass transfer.
\end{enumerate}

In what follows we restrict attention to strictly \textbf{conservative mass
transfer},
\[
\dot{M}_1 + \dot{M}_2 = 0,
\]
while allowing for angular momentum loss:
\[
\dot{J} < 0.
\]
The orbital response is then
\[
\frac{\dot{a}}{a}
=
\frac{2\dot{J}}{J}
+
\frac{2(-\dot{M}_2)}{M_2}(1-q),
\]
and the Roche–lobe response,
using the Paczyński formula, becomes
\[
\frac{\dot{R}_{2,L}}{R_{2,L}}
=
\frac{2\dot{J}}{J}
+
\frac{2(-\dot{M}_2)}{M_2}\!\left(\frac{5}{6}-q\right).
\]

\subsubsection{Conditions for Stability with Angular Momentum Loss}

A Roche–lobe filling donor remains stable when a small amount of mass loss does
not increase the degree of overflow:
\[
\frac{\dot{R}_2}{R_2} \lesssim \frac{\dot{R}_{2,L}}{R_{2,L}}.
\]
Since 
\[
\frac{\dot{R}_2}{R_2} = \zeta_2\,\frac{\dot{M}_2}{M_2},
\]
we obtain
\[
\zeta_2\frac{\dot{M}_2}{M_2}
\;\lesssim\;
\frac{2\dot{J}}{J}
+
\frac{2(-\dot{M}_2)}{M_2}\left(\frac{5}{6}-q\right).
\]
Because $\dot{M}_2 < 0$, this becomes the stability condition
\begin{equation}
\boxed{
\zeta_2 \;\ge\; 2\left(q - \frac{5}{6}\right)
\;+\; 
\frac{M_2}{-\dot{M}_2}\frac{2\dot{J}}{J}
}
\label{eq:AML_stability_condition}
\end{equation}
where the final term represents the stabilizing influence of AML.  
Since $\dot{J}<0$, the AML term is positive: \emph{angular momentum loss widens
the range of $\zeta_2$ for which mass transfer is stable}.

The consequences depend on whether AML is subdominant or dominant compared to
the Roche–lobe response from mass transfer itself.

\paragraph{Scenario 1: Subdominant AML.}

If $|\dot{J}/J|$ is small compared with the geometric mass–transfer feedback,
the \textbf{system behaves much like the fully conservative case.  }
For $q<5/6$, AML weakens the tendency of the Roche Lobe to expand, making 
marginally stable configurations more robust.  
For $q>5/6$, AML accelerates Roche–lobe contraction and strengthens the 
conditions for runaway, unless the donor contracts rapidly enough 
(i.e.\ very large $\zeta_2$).

\paragraph{Scenario 2: Dominant AML.}

If angular momentum losses dominate,
\[
\left| \frac{2\dot{J}}{J} \right| \gg 
\left|\frac{2(-\dot{M}_2)}{M_2}\left(\frac{5}{6}-q\right)\right|,
\]
then the stability condition reduces to
\[
\zeta_2 \gtrsim \frac{M_2}{-\dot{M}_2}\frac{2\dot{J}}{J}.
\]
Equivalently,
\[
\frac{-\dot{M}_2}{M_2} 
\;\gtrsim\;
\frac{2}{\zeta_2}\frac{\dot{J}}{J}.
\]
In this AML–driven regime, \textbf{the mass–transfer rate is set by the requirement
that the donor keeps pace with the Roche–lobe contraction induced by angular
momentum loss itself.}

\subsubsection{A Roche--Lobe Filling Binary Driven by Gravitational Waves}

Let us now consider the case where angular momentum loss is driven 
\emph{entirely} by gravitational radiation.  
For a circular binary,
\[
\frac{\dot{J}_{\rm GW}}{J}
=
-\frac{32}{5}\frac{G^{3}}{c^5}
  \frac{M_1M_2(M_1+M_2)}{a^{4}}.
\]
If the system contains no mass transfer, this yields
\[
\frac{\dot{a}}{a}
= 2\frac{\dot{J}_{\rm GW}}{J},
\qquad
\frac{\dot{P}}{P}
= \frac{3}{2}\frac{\dot{a}}{a}.
\]
Defining the gravitational–wave inspiral timescale,
\[
t_{\rm GW}\equiv\left|\frac{a}{\dot{a}}\right|
= \frac{5}{64}\frac{c^5 a^4}{G^3 M_1M_2(M_1+M_2)},
\]
we obtain (using $a\propto P^{2/3}$)
\[
t_{\rm GW} \simeq 1.0\times10^{8}\ {\rm yr}\;
\left(\frac{P_{\rm hr}}{2}\right)^{8/3}
\left(\frac{\mathcal{M}}{0.3\,M_\odot}\right)^{-5/3},
\]
with $\mathcal{M}$ the chirp mass.  
Thus GW losses become astrophysically dominant only at \textbf{very short} orbital periods.

\medskip
\noindent
Let us now include \textbf{conservative mass transfer}.  
A Roche–lobe filling donor satisfies
\[
\frac{\dot{R}_2}{R_2}=\frac{\dot{R}_{2,L}}{R_{2,L}}.
\]
For a donor with $R_2\propto M_2^{n}$,
\[
\frac{\dot{R}_2}{R_2} = n\,\frac{\dot{M}_2}{M_2}.
\]
Equating to the Roche–lobe response gives
\[
n\frac{\dot{M}_2}{M_2}
=
2\frac{\dot{J}}{J}
+
2\frac{(-\dot{M}_2)}{M_2}\left(\frac{5}{6}-q\right),
\]
or
\[
\frac{\dot{M}_2}{M_2}
=
\frac{2(\dot{J}/J)}{\,n + \tfrac{5}{3} - 2q\,}.
\]
Hence the gravitational–wave–driven mass–transfer rate is
\begin{equation}
\boxed{
\frac{\dot{M}_2}{M_2}
= 
\frac{2\,(\dot{J}_{\rm GW}/J)}{n + \tfrac{5}{3} - 2q}
}
\label{eq:mass_transfer_gw_general}
\end{equation}
with
\[
\frac{\dot{J}_{\rm GW}}{J}
= -\frac{32}{5}\frac{G^{3}}{c^5}
  \frac{M_1M_2(M_1 + M_2)}{a^{4}}.
\]
Using Kepler’s law to eliminate $a$, we obtain
\begin{equation}
\boxed{
-\dot{M}_2
= \frac{64}{5}\frac{G^{5/3}}{c^{5}}(2\pi)^{8/3}
  \frac{M_1M_2^2}{(M_1+M_2)^{1/3}}
  \frac{P^{-8/3}}{\,n + \tfrac{5}{3} - 2q\,}.
}
\label{eq:mdot_gw_general}
\end{equation}

Two limiting cases illustrate the behavior:

\begin{itemize}
    \item \textbf{Main--Sequence donors ($n\approx 1$).}  
    With $q\ll 1$, $n+\tfrac{5}{3}-2q\simeq\tfrac{8}{3}$:
    \[
    -\dot{M}_2 \simeq
    10^{-10}\,M_\odot\,{\rm yr^{-1}}
    \left(\frac{P_{\rm hr}}{2}\right)^{-2/3}.
    \]

    \item \textbf{Degenerate donors ($n=-1/3$).}  
    With $q\ll 1$, $n+\tfrac{5}{3}-2q\simeq\tfrac{4}{3}$:
    \[
    -\dot{M}_2 \simeq
    10^{-12}\,M_\odot\,{\rm yr^{-1}}
    \left(\frac{P_{\rm hr}}{2}\right)^{-14/3}.
    \]
\end{itemize}

\begin{bigidea}
Gravitational waves provide an external, precisely determined sink of angular 
momentum.  
For a Roche--lobe filling donor, the mass–transfer rate is uniquely fixed by the
requirement that the donor keep pace with the Roche–lobe contraction driven
by GWR.  
This coupling produces the characteristic ``GW–driven'' mass–transfer phases of
ultracompact binaries, where the orbital evolution and mass loss are locked
together by the interplay between geometry, donor structure, and general 
relativity.
\end{bigidea}


\subsection{Non-Conservative Mass Transfer}

\subsubsection{Important Mechanisms of Mass Loss}

\subsubsection{Important Mechanisms of Angular Momentum Loss}

\subsubsection{Conditions for Stability}


\subsection{Common Envelope Scenarios}

\subsection{Scenarios With CSM Formation}




\section{Disk Formation}

We now consider the extremely interesting scenario of \textbf{disk formation}. The principle behind this process stems from the fact that (from the perspective of the primary), the critical point of the Roche-Potential appears to orbit with velocity $v_\perp = b_1\omega$, which means that as material is accreted, it is accreted \textbf{as if squirted through a nozzel spinning around the primary}. As such, conservation of angular momentum prevents the material from simply falling onto the primary.
\par
In a non-rotating frame centered on the primary, the nozzel appears to move with $v_\perp = b_1\omega$. Material forced through the nozzle is forced through via pressure, which means \textbf{it must be subsonic parallel to the line of centers}:
\[
v_\parallel \sim c_s.
\]
Given that $b_1 \sim 0.5a$ (unless $q \gg 1$), we may express the perpendicular rotation as
\[
v_\perp \sim 100\; \left(\frac{M_1}{M_\odot}\right)^{1/3} (1+q)^{1/3} P_{\rm day}^{-1/3} \;{\rm km\;s^{-1}}.\;\text{(From Kepler's 3rd Law)}
\]
For a typical set of conditions on the gas, $v_\parallel \sim 10\;{\rm km\;s^{-1}}$. Thus, we know the following facts:
\begin{enumerate}
    \item \textbf{The tangential velocity component is dominant}, and
    \item \textbf{The flow is super-sonic}.
\end{enumerate}
The second of these is incredibly important:\\
\framebox{Because the flow is supersonic, pressure becomes irrelevant} on timescales relevant to the flow. We can therefore treat the problem \textbf{ballistically}.
\par
The above arguments provide the following qualitative picture:
\vspace{0.5cm}
\begin{bigidea}
    The effective motion of the material falling onto the primary is that of a particle \textbf{released from rest} with a \textbf{specific angular momentum} from the $L_1$ point. That material will then follow an \textbf{elliptical orbit} around the primary (determined by the field of the primary alone). The \textbf{secondary} contributes perturbative alterations which cause \textbf{precession of the orbit}. Because the orbit precesses, the flows \textbf{interact with themselves}. This leads to \textbf{shocks and other dissipative processes}, which serve to rid the material of \textbf{energy}, but (critically) \textbf{not of angular momentum}. 
\par
Because the material loses energy at constant angular momentum, the orbit will \textbf{decay into a circular orbit} determined by the \textbf{specific angular momentum} of the material.
\end{bigidea}
\vspace{0.5cm}
From a qualitative standpoint, we discuss the \textbf{circularization radius} $R_{\rm circ}$, where the material ends up. On the basis of conservation of angular momentum, we know that
\[
v_{\perp}(R_{\rm circ}) \underbrace{=}_{\rm centripetal} \left(\frac{GM_1}{R_{\rm circ}}\right)^{1/2}.
\]
(\rmk{we're just using centripetal force here})
The specific angular momentum is conserved so
\[
R_{\rm circ} v_\perp(R_{\rm circ}) = b_1^2 \omega,
\]
(\rmk{this comes from the angular momentum at injection, which is just $b_1 \times b_1 \omega$, etc.}). We can now use 
\begin{equation}
\label{eq:cicularization_radius}
    \boxed{
    \frac{R_{\rm circ}}{a} = \left(\frac{4 \pi^2}{GM_{1} P^2}\right)a^3 \left(\frac{b_1}{a}\right)^4 \underbrace{=}_{\text{K3L}} (1+q)\left(\frac{b_1}{a}\right)^{4} \approx \left(1+q\right) \left(0.5 - 0.227\log q\right)^4
    }
\end{equation}
Using the typical scalings for $a$ (equation~\eqref{eq:binary_orbit_semi_major_axis_united}), we find
\[
\boxed{
R_{\rm circ} \approx 4(1+q)^{4/3} [0.5 - 0.227 q]^4 P^{2/3}_{\rm day} \;R_\odot.
}
\]
\par
\textbf{What do we learn from this}? Because $R_{\rm circ}$ is typically of order $R_\odot$, for example,
\[
R_{\rm circ} \approx 1.2 P_{\rm day}^{2/3} R_\odot,\;q=0.3,
\]
we do see different behavior depending in whether or not the primary is \textbf{extended or compact}. When this accretion is onto an extended object and $R_{\rm circ} < R_{\star}$, then \textbf{no disk formation occurs} and the material falls obliquely onto the surface. This can dissipate some energy in shocks, but is not particularly efficient. \textbf{MUCH more importantly}, when $R_{\rm circ} \gg R_\star$ as is the case in compact object accretion, the disk formation is much more efficient and the resulting accretion can be more efficient as well. In general, for \textbf{realistic parameters},
\[
R_{\rm circ} \ge 3.5 \times 10^{9}\; P^{2/3}_{\rm hr}\;{\rm cm},
\]
so we \textbf{cannot get binary accretion onto objects larger than white dwarfs}!
\par
We now arrive at an idea so fundamental it can hardly be stated with sufficient attention:
\begin{bigidea}
    When accretion occurs from a binary partner onto a compact companion, the result is the \textbf{circularization of accreted material} into an \textbf{accretion disk}. This \textbf{accretion disk} will experience various \textbf{dissipitive effects}, including radiative cooling, viscous dissipation, etc. Because it loses energy, it will want to \textbf{sink further into the potential}; however, it must lose angular momentum to do so. The timescale for momentum transfer is \textbf{much much longer} than for energy transfer, which means
    \newline
    effectively \textbf{all the energy must be dissipated} before falling into a lower orbit. This makes accretion disks \textbf{incredibly efficient}.
\end{bigidea}
\par
In general, the \textbf{self-gravity of the disk} is negligible and therefore the disk's circular annuli are Keplerian with angular velocity determined from centripetal force:
\[
\Omega_K(R) = \left(\frac{GM_1}{R^3}\right)^{1/2}.
\]
Now, in a \textbf{circular orbit} at radius $R_\star$ at the surface of the accretor, the \textbf{specific orbital energy} is
\[
K = \frac{1}{2}\Omega_K^2R^2 - \frac{GM_1}{R} = -\frac{1}{2} \frac{GM_1}{R}.
\]
Since the accreted material has negligible binding energy when it falls in from the Roche boundary, we see that the luminosity of the disk should be
\begin{equation}
    \boxed{
    L_{\rm disk} = \frac{GM_1\dot{M}}{2R} = \frac{1}{2}L_{\rm acc}.
    }
\end{equation}
This is \textbf{incredible}. Nearly 50\% of the binding energy \textbf{can} be dissipated away (we're not guaranteed that it will)!
\par
Just as the energy of the material must be lost during its inspiral, so too much the angular momentum. This will rely on outward angular momentum transfer vis-a-vis viscous torques on the material.

\section{Viscous Torques and Angular Momentum Transport}

In this final section regarding binary accretion scenarios, we discuss the nature of angular momentum transport and viscosity. This will lay the ground work for our next chapter regarding the nature of thin disk accretion.

\subsection{Microscopic Intuition for Viscous Flows}
On a \textbf{statistical level}, viscosity arises due to random motions bringing momentum across laminae. These motions may be \textbf{thermal} or they may be \textbf{turbulent} or have some other driving mechanism; however, the formal treatement is the same nonetheless. We introduce a \textbf{standard length scale} $\tilde{\lambda}$ and a \textbf{standard velocity} of the random motions $\tilde{v}$. These two quantities determine the effective viscosity as we will now derive.

Consider two laminae at \textbf{slightly different vertical positions}, moving with horizontal velocity $v(z)$. 
Although there is no net \emph{mass} flux across the interface (the exchange is symmetric),  the parcels that cross carry their horizontal momentum with them. \rmk{This is much like pedestrians wandering from the sidewalk into the street: they carry their momentum and disrupt the flow of traffic, even though the net number of people on the street does not change.}

Over a characteristic time interval $\Delta t$, a fluid mass of order
\[
\Delta m \;\sim\; \rho\, \tilde{v}\,\Delta t
\]
crosses between the laminae, where $\rho$ is the density. Because the velocity difference between the two laminae is of order $\partial_z v \,\tilde{\lambda}$,  the exchanged parcels transport a momentum
\[
\Delta p \;\sim\; \Delta m \,\bigl(\partial_z v \,\tilde{\lambda} \bigr)
      \;\sim\; \rho\, \tilde{v} \,\tilde{\lambda} \,\partial_z v \,\Delta t .
\]
Thus the turbulent momentum flux is proportional to the local shear $\partial_z v$, with an \textbf{effective turbulent viscosity}
\[
\nu_{\rm turb} \;\sim\; \tilde{v}\tilde{\lambda} ,\implies \eta = \rho\nu_{\rm turb} = \rho \tilde{v}\tilde{\lambda}.
\]
so that
\[
\frac{\Delta p_x}{\Delta t} = - \eta \frac{\partial v}{\partial z}.
\]
which is the \textbf{canonical form we anticipate}:
\[
\sigma_{xz} = - \eta \frac{\partial v}{\partial z}.
\]
\medskip
\textbf{A very similar argument applies to accretion disks}.  Consider two neighboring annuli at radii $R$ and $R+\tilde{\lambda}$, where $\tilde{\lambda}$ is the radial displacement of turbulent eddies. In steady state the mean radial velocity vanishes, so there is no net mass flux across the annulus boundary. Nonetheless, turbulence exchanges fluid parcels between the annuli.

The mass exchanged across the boundary per unit arc length $dL$ in a time interval $dt$ is
\[
dm \;\sim\; \rho \, v_{\rm turb} \, H \, dL \, dt ,
\]
where $\rho$ is the midplane density, $H$ is the disk scale height, and $v_{\rm turb}$ is the characteristic turbulent velocity.  

Each parcel carries its specific angular momentum
\[
\ell(R) \;=\; R^2 \Omega(R).
\]
Expanding to first order in $\tilde{\lambda}$, the difference between parcels exchanged from 
$R+\tilde{\lambda}$ and $R-\tilde{\lambda}$ is
\[
\ell(R+\tilde{\lambda}/2) - \ell(R-\tilde{\lambda}/2)
   \;\approx (R+\tilde{\lambda}/2)^2\Omega(R+\tilde{\lambda}/2) - (R-\tilde{\lambda}/2)^2\Omega(R-\tilde{\lambda}/2).
\]
This is effectively,
\[
\ell(R+\tilde{\lambda}/2)-\ell(R-\tilde{\lambda}/2) \approx R^2 \Omega(R+\tilde{\lambda}/2) - R^2\Omega(R-\tilde{\lambda}/2) \sim\tilde{\lambda} R^2 \frac{d\Omega}{dR}.
\]
The bulk $2R\Omega$ term cancels in the symmetric exchange, leaving only the shear contribution:
\[
\Delta \ell \;\approx\;\tilde{\lambda} \, R^2 \frac{d\Omega}{dR}.
\]
\rmk{Formally, one evaluates this in a frame comoving with the inner annulus; in that frame only the differential shear, not the bulk motion, contributes.} The net angular momentum flux across the boundary per unit arc length is
\[
F_J \;\equiv\; \frac{dJ}{dA\,dt}
   \;\sim\; \rho \, v_{\rm turb}\, \Delta \ell
   \;\sim\; \rho \, v_{\rm turb} \,\tilde{\lambda} \, R^2 \frac{d\Omega}{dR}.
\]
We must now be careful because we have made an implicit \textbf{sign convention} here. $\Delta \ell$ is the momentum change of the \textbf{inner annulus} and is the negative of the change in the \textbf{outer annulus}. As such, if $d\Omega / dR <0$ as is the case in Keplerian disks, we see that $F_J < 0$ meaning we are moving angular momentum \textbf{outward.} In what follows, we will swap the argument so we match the standard convention in accretion physics.
\par
The \textbf{momentum flux through the boundary} is a \textbf{torque per area}. If we want \textbf{force per area}, we have
\[
\sigma_{r\phi} = \frac{F_J}{R} \sim \rho v_{\rm turb} \tilde{\lambda} R \Omega'.
\]
by definition, the \textbf{shear viscosity (kinematic)} is defined such that
\[
\sigma_{r\phi} = \eta R\Omega' \implies \eta = \rho v_{\rm turb} \tilde{\lambda}.
\]
Thus,
\begin{equation}
    \boxed{
    \nu = \tilde{\lambda}v_{\rm turb}.
    }
\end{equation}

\subsection{Viscous Dissipation}

Now that we have the \textbf{torque density} on the surface between the laminae, we can determine the \textbf{total torque} on the single laminae:
\[
\frac{dJ}{dt} = - 2 \pi \Sigma \nu R^3 \frac{d\Omega}{dR}.
\]
Now the \textbf{net torque} on a single lamina between $R$ and $R+\delta R$ is 
\[
G = \frac{dJ}{dt}_{R+\delta R} - \frac{dJ}{dt}_{R} = \frac{\partial}{\partial R} \left(\frac{dJ}{dt}\right) \delta R.
\]
This torque will, of course, perform work in its action. A general torque $\tau$ does work $dW = \tau d\phi$, so
\[
P = \Omega \frac{\partial}{\partial R} \left(\frac{dJ}{dt}\right) \delta R.
\]
is the power dissipated in the disk lamina. If we use the chain rule,
\[
P = \delta R\left[\frac{\partial(\Omega dJ/dt)}{\partial R} - \frac{dJ}{dt} \Omega'\right].
\]
This first term represents \textbf{advected work} which is done on both edges of the lamina. The second term is the \textbf{heat dissipated}. If we proceed from this, the dissipation per unit surface area is
\[
D(R) = - \frac{\Omega' (dJ/dt)}{4\pi R} = \frac{1}{2}\nu \Sigma \left(R\frac{d\Omega}{dR}\right)^2.
\]

\medskip
Dividing by the annular surface area $2\pi R\,dR$ (for each disk face), 
and then accounting for both sides of the disk, gives the \textbf{dissipation rate per unit surface area}:
\begin{equation}
    \label{eq:disk_dissipation_per_unit_area}
    \boxed{D(R) = \frac{1}{2}\,\nu \Sigma \,\Bigl(R \frac{d\Omega}{dR}\Bigr)^2.}
\end{equation}
\medskip
Thus the viscous heating rate is directly proportional to the square of the local velocity shear. 
In a Keplerian disk, where $\Omega \propto R^{-3/2}$, we have
\begin{equation}
R \frac{d\Omega}{dR} = -\tfrac{3}{2}\,\Omega,
\end{equation}
so that
\begin{equation}
\label{eq:keplerian_disk_heating_law}
\boxed{
D(R) = \frac{9}{8}\,\nu\,\Sigma\,\Omega^2,}
\end{equation}
\textbf{the familiar thin--disk heating law.}

\subsection{The Magnitude of Viscosity}

We have yet to assess how important viscosity really is in disk flows.  \textbf{A simple \emph{scaling argument} helps clarify this point.}
\medskip
\noindent
Viscosity gives rise to stresses of order
\begin{equation}
f \;\sim\; \sigma_{r\phi}\, dA \;\sim\; \rho \nu \frac{dv_\phi}{dR}\, dA ,
\end{equation}
where $\nu$ is the kinematic viscosity.  
Differentiating once more to account for variations across a fluid element, the net viscous force scales like
\begin{equation}
df \;\sim\; \rho \nu \frac{d^2 v_\phi}{dR^2}\, dR\, dA .
\end{equation}
Dividing by the volume $dV = dR\,dA$ gives the \textbf{viscous force density}
\begin{equation}
f_{\rm visc} \;\sim\; \rho \nu \,\frac{\partial^2 v_\phi}{\partial R^2}
                 \;\sim\; \rho \nu \,\frac{v_\phi}{R^2}.
\end{equation}
\rmk{This is only dimensional analysis; the exact expression has geometric correction terms, but they are the same order of magnitude.}
\medskip
\noindent
Meanwhile, the inertial force density associated with advection scales as
\begin{equation}
f_{\rm in} \;\sim\; (\mathbf{u}\cdot\nabla)\mathbf{u}
              \;\sim\; \frac{v_\phi^2}{R}.
\end{equation}
\medskip
\noindent
Taking the ratio of inertial to viscous forces defines the 
\textbf{Reynolds number},
\begin{equation}
{\rm Re} \;\sim\; \frac{f_{\rm in}}{f_{\rm visc}}
        \;\sim\; \frac{v_\phi^2/R}{\nu v_\phi/R^2}
        \;\sim\; \frac{R v_\phi}{\nu}.
\end{equation}

\rmk{This is the definition of the Reynold's number!}
\medskip
\noindent
\par
For \textbf{molecular viscosity}, $\nu$ is extremely small on astrophysical scales, 
so the Reynolds number is enormous and molecular viscosity is dynamically irrelevant.  
This motivates the standard assumption that disk flows are turbulent.  
Since we lack a predictive theory of turbulence, the effective viscosity is usually parameterized 
by the \emph{$\alpha$--model}, where one writes
\[
\nu \;=\; \alpha \, c_s \, H,
\]
with $c_s$ the sound speed, $H$ the disk scale height, and $\alpha$ a dimensionless parameter
that encodes our ignorance of the true turbulent transport.
