
Supernovae represent some of the most energetic events in the universe, releasing a total energy on the order of
\[
E_{\mathrm{SN}} \sim 10^{51}\ {\rm erg},
\]
the majority of which is initially carried as \textbf{kinetic energy} in the expanding stellar ejecta.
This tremendous release drives powerful shock waves into the surrounding interstellar or circumstellar medium (ISM/CSM),
transforming the explosion’s mechanical energy into \textbf{heat, radiation, and cosmic rays.}

To first order, the kinetic energy distribution of the ejecta can be described as
\[
E_{\rm kin} = \frac{1}{2} M_{\rm ej} v_{\rm ej}^2,
\]
where $M_{\rm ej}$ is the total ejecta mass (typically $1$--$10\,M_\odot$ for core-collapse events) and $v_{\rm ej}$ the characteristic expansion speed.
For $v_{\rm ej} \sim 10^4\ {\rm km\,s^{-1}}$, this yields $E_{\rm kin} \sim 10^{51}$ erg, 
in good agreement with the energetics inferred from observations of Type~II and Type~Ia supernovae.
\medskip

\noindent
As the explosion expands, this kinetic energy is redistributed by shocks and compression waves,
which convert part of the bulk motion into thermal energy and observable radiation.
The morphology and evolution of these shocks encode key information about the explosion dynamics,
the structure of the progenitor, and the environment into which the remnant expands.

\section{The Dynamics of SNR Shocks}

To begin our discussion of SNR shocks, we will want to characterize the \textbf{three stages of a SN shock}. In each phase, different physical properties are dominant and drive different sorts of evolutionary processes. 
Before we begin, it is worth recalling the conditions for the \textbf{Sedov-Taylor Blastwave}:

A \textbf{blastwave} is a \textbf{point-like explosion} of total energy $E_0$ released at $t=0$ into a uniform medium of constant density $\rho_0$.
The explosion energy is assumed to be \textbf{deposited instantaneously} and to \textbf{remain conserved}:
\[
E_0 = \text{constant}.
\]
We further assume:

\begin{enumerate}
    \item The \textbf{shock is strong} ($M \to \infty$), so the upstream pressure $P_0$ is negligible compared to the post-shock pressure $P_1$.
    \item The \textbf{flow is adiabatic} with adiabatic index $\gamma$.
    \item The medium is \textbf{spherically symmetric}, with no gravitational or external forces.
\end{enumerate}

As we will see, there are two important breaking points for this model when it comes to SN shocks:
\vspace{1cm}
\begin{enumerate}
    \item \textbf{Radiative Losses}: At late times after the explosion, the shocked material will cool sufficiently to make cooling efficient and consequently cause a significant change in the structure of the remnant. This marks the \textbf{late end} of the Sedov-like expansion.
    \item \textbf{Point-Like Explosion}: A more subtle element of the Sedov solution is that it assumes a point-like explosion: all of the energy $E_0$ is \textbf{instantly thermalized} at the origin. This scenario is \textbf{not correct} during the early phase of the explosion when most of the energy is stuck in the ejecta.
\end{enumerate}
\vspace{1cm}
As such, we will have 3 phases in which the Sedov blastwave will make up only the middle part of the theory.

\subsection{Free-Expansion (Ejecta-Dominated) Phase}

The \textbf{earliest stage} of a supernova remnant’s evolution begins immediately after the stellar envelope has been ejected and the forward shock has broken out of the progenitor. At this point, the \emph{ejecta} moves \textbf{almost unimpeded into the surrounding interstellar or circumstellar medium.} 
Critically, the energy is \textbf{kinetic, not thermal} and the expansion is \textbf{momentum dominated} not pressure dominated like the Sedov scenario.

Given a total ejected mass $M_{\rm ej}$ with velocity $v_{\rm ej}$, this phase \textbf{conserves momentum}, so
\[
R_{\rm shock} \sim v_{\rm ej} t = 1.02 \left(\frac{v_{\rm ej}}{\rm 10^4\; km\;s^{-1}}\right)\left(\frac{t}{\rm 100\;yr}\right)\; {\rm pc}.
\]
This, of course, corresponds to a \textbf{total explosive energy budget}
\[
E_{0} = \frac{1}{2} M_{\rm ej} v_{\rm ej}^2 \implies v_{\rm ej} = 10^4\;\left(\frac{M_{\rm ej}}{M_\odot}\right)^{-1/2} \left(\frac{E_0}{10^{51}\;{\rm erg}}\right)^{1/2}\; {\rm km\;s^{-1}}
\]
Once the explosion has broken out of the star, the ejecta can move \textbf{ballistically}, so, for any given fluid element $i$,
\[
r_i(t) = v_i t,\;\text{and}\; dv_i/dt = 0
\]
Thus, each parcel will have
\[
\boxed{
v(r,t) = \frac{r}{t},
}
\]
corresponding to so-called \textbf{homologous expansion}.

\subsubsection{The Density Structure}

From the \textbf{continuity equation} in spherical symmetry,
\begin{equation}
\frac{\partial \rho}{\partial t}
+ \frac{1}{r^{2}}\frac{\partial}{\partial r}\!\left(r^{2}\rho v\right) = 0,
\label{eq:continuity-homologous}
\end{equation}
we now assume the ejecta undergo \textbf{homologous expansion}, so that every fluid element moves ballistically with velocity
\[
v(r,t) = \frac{r}{t}.
\]

To make progress, it is useful to introduce the similarity variable
\[
v \equiv \frac{r}{t},
\]
and seek a self--similar density profile of the separable form
\[
\rho(r,t) = t^{-\alpha} F(v),
\]
where $\alpha$ and the function $F$ are to be determined.

We therefore compute the derivatives in Eq.~\eqref{eq:continuity-homologous}.
First, using $v=r/t$ with $r$ held fixed in the partial derivative,
\[
\frac{\partial v}{\partial t} = -\frac{r}{t^{2}} = -\frac{v}{t},
\]
so that
\begin{align}
\frac{\partial \rho}{\partial t}
&= -\alpha\, t^{-(\alpha+1)} F(v)
   + t^{-\alpha} F'(v)\,\frac{\partial v}{\partial t} \\
&= t^{-(\alpha+1)}\!\left[-\alpha F(v) - v F'(v)\right].
\end{align}

Next, we evaluate the divergence term.  Noting that
\[
r^{2}\rho v = r^{2} t^{-\alpha} F(v)\,\frac{r}{t}
            = t^{2-\alpha} v^{3} F(v),
\]
and that
\[
\frac{\partial}{\partial r}
 = \frac{\partial v}{\partial r}\frac{d}{dv}
 = \frac{1}{t}\frac{d}{dv},
\]
we obtain
\begin{align}
\frac{1}{r^{2}}\frac{\partial}{\partial r}(r^{2}\rho v)
&= \frac{1}{(vt)^{2}}
   t^{1-\alpha} \frac{d}{dv}\!\left(v^{3}F(v)\right) \\
&= t^{-(\alpha+1)}
   \frac{1}{v^{2}} \frac{d}{dv}\!\left(v^{3}F(v)\right).
\end{align}

Substituting these expressions back into the continuity equation gives
\[
-\alpha F - vF'
\;+\;
\frac{1}{v^{2}}\frac{d}{dv}\!\left(v^{3}F\right)
= 0.
\]
Since
\[
\frac{1}{v^{2}}\frac{d}{dv}\!\left(v^{3}F\right)
= 3F + vF',
\]
the equation simplifies to
\[
(3-\alpha)F(v) = 0.
\]
For a nontrivial density profile $F(v)\neq 0$, we therefore require
\[
\boxed{\alpha = 3}.
\]

Thus the continuity equation together with homologous expansion forces the density to take the form
\begin{equation}
\boxed{
\rho(r,t) = t^{-3}\,F\!\left(\frac{r}{t}\right)
}
\label{eq:rho-self-similar}
\end{equation}
where the function $F(v)$ encodes the \emph{shape} of the ejecta profile. The continuity equation fixes only the $t^{-3}$ scaling (reflecting volume dilution in ballistic expansion), while the \textbf{detailed form of $F(v)$ is determined
by the explosion physics and the progenitor’s outer density structure.}

\subsubsection{Termination of Free Expansion}


The obvious question to ask is \textbf{at what point does deceleration matter?}
To answer this, we require that the swept up mass of the ICM/ISM be similar to the ejecta mass, meaning that we have been forced to accelerate a non-trivial amount of mass. 
In this scenario,
\[
M_{\rm ej} \sim M_{\rm ism}(t) = \frac{4}{3}\pi \rho_0 R_{\rm shock}^3 = \frac{4}{3}\pi \rho_0 v_{\rm ej}^3 t^3.
\]
We can therefore identify a time at the \textbf{end of free expansion} for which
\begin{equation}
    \boxed{
    t \sim \left(\frac{3M_{\rm ej}}{4\pi \rho_0}\right)^{1/3} \frac{1}{v_{\rm ej}}, \;R\sim \left(\frac{3 M_{\rm ej}}{4\pi \rho}\right)^{1/3}.
    }
\end{equation}
For standard scalings, this becomes
\[
\boxed{
\begin{aligned}
            R_{\rm FE} &\sim 2.5 \left(\frac{M_{\rm ej}}{M_\odot}\right)^{1/3} \left(\frac{\rho_0}{10^{-24}\;{\rm g/cm^3}}\right)^{-1/3}\; {\rm pc}. \;\text{(FE)}\\
        t_{\rm FE} &\sim 244\;\left(\frac{v_{\rm ej}}{10^4\;{\rm km/s}}\right)^{-1}\left(\frac{M_{\rm ej}}{M_\odot}\right)^{1/3} \left(\frac{\rho_0}{10^{-24}\;{\rm g/cm^3}}\right)^{-1/3}\; {\rm pc}. \;\text{(FE)}
\end{aligned}
}
\]

Beyond this point, the swept-up mass begins to dominate the dynamics, and the remnant enters the adiabatic, energy-conserving stage described by the \textbf{Sedov–Taylor solution}. \rmk{This is actually an aggressive choice for the truncation time. We might prefer something on the order of 500 - 1000 years from a more sophisticated analysis.}

\subsubsection{Thermalization of the Ejecta}

We have described how the Sedov-Taylor solution requires that we effectively thermalize the energy of the ejecta; however, we have not yet discussed the mechanism for doing so. The answer is the \textbf{reverse shock}, which effectively thermalizes the ejecta as it rams into the shock discontinuity caused by the deceleration of the shock by the ambient material. This occurs on the same general timescale as the mass equalization.

\subsubsection{Emission Features}

In the \textbf{free expansion phase}, emission is dominated by \textbf{optical emission} from the cold ejecta, which has not yet been shock heated. 
The optical emission will feature various \textbf{broad emission / absorption lines} characteristic of the cold expanding medium.

In the \textbf{x-ray}, the forward shock will produce a subdominant \textbf{thermal component} from the shock heated material which will grow with time as more material gets shocked. The dominant emission from the forward shock will be \textbf{radio synchrotron emission}.

Reverse shock X-ray emission will emerge only later on in this phase.

This big takeaway here is that the \textbf{optical emission} is characteristic of \textbf{free expansion} because most of the ejecta has not yet been shocked. Nonetheless, there is still a shockwave here which produces some X-ray and radio emission.

\subsection{Sedov–Taylor (Adiabatic) Phase}

Once the \textbf{free expansion phase} is completed at typical radii and times of 
\[
\boxed{
\begin{aligned}
            R_{\rm FE} &\sim 2.5 \left(\frac{M_{\rm ej}}{M_\odot}\right)^{1/3} \left(\frac{\rho_0}{10^{-24}\;{\rm g/cm^3}}\right)^{-1/3}\; {\rm pc}. \;\text{(FE)}\\
        t_{\rm FE} &\sim 244\;\left(\frac{v_{\rm ej}}{10^4\;{\rm km/s}}\right)^{-1}\left(\frac{M_{\rm ej}}{M_\odot}\right)^{1/3} \left(\frac{\rho_0}{10^{-24}\;{\rm g/cm^3}}\right)^{-1/3}\; {\rm pc}, \;\text{(FE)}
\end{aligned}
}
\]
we transition into the \textbf{Sedov-Taylor expansion} phase in which the ejecta has largely been \textbf{thermalized} by the reverse shock and we are now in the self-similar expansion characteristic of this type of blastwave.
In this case, there is \textbf{no longer a reverse shock, it has fully thermalized the shock interior.}

As we have previously derived, the \textbf{Sedov-Taylor} solution produces a shock with a radius scaling as
\begin{equation}
    \label{eq:sedov-taylor-radius}
    R_{\rm shock}(t) = \xi_0\left(\frac{E_0t^2}{\rho_0}\right)^{1/5}.
\end{equation}
  For $\gamma = 5/3$, $\xi_0 \approx 1.15$. Taking the derivative, we have
\begin{equation}
    \label{eq:sedov-taylor-shock-velocity}
    u_{\rm shock} = \frac{dR_{\rm shock}}{dt} = \frac{2}{5}\xi_0 \left(\frac{E_0}{\rho_0 t^3}\right)^{1/5} = \frac{2}{5}\xi_0^{5/2} \left(\frac{E_0}{\rho_0}\right)^{1/2} R_{\rm shock}^{-3/2}.
\end{equation}
If we calculate these for the typical scalings relevant to \textbf{supernova remnants}, we will find that
\[
R_{\rm shock} \approx 2.3 \left(\frac{E_0}{10^{51}\;\rm erg}\right)^{1/5} \left(\frac{\rho_0}{10^{-24}\;\rm g\;cm^{-3}}\right)^{-1/5}\left(\frac{t}{1000\;\rm yr}\right)^{2/5}\;{\rm pc}.
\]
and
\[
u_{\rm shock} \approx 9 \times 10^3 \left(\frac{E_0}{10^{51}\;\rm erg}\right)^{1/5} \left(\frac{\rho_0}{10^{-24}\;\rm g\;cm^{-3}}\right)^{-1/5}\left(\frac{t}{1000\;\rm yr}\right)^{-3/5}\;{\rm km/s.}
\]

At the shock front ($\xi=1$), the \textbf{Rankine--Hugoniot relations} provide the boundary values
of the flow variables relative to the upstream state $(\rho_0, u_0=0, P_0=0)$.

For a strong adiabatic shock, we have:
\begin{align}
\rho_2 &= \frac{\gamma+1}{\gamma-1}\rho_0 \implies \rho_2 = 4 \rho_0\\
u_2 &= \frac{2}{\gamma+1}\,\dot{R}, \implies  6.7 \times 10^3 \left(\frac{E_0}{10^{51}\;\rm erg}\right)^{1/5} \left(\frac{\rho_0}{10^{-24}\;\rm g\;cm^{-3}}\right)^{-1/5}\left(\frac{t}{1000\;\rm yr}\right)^{-3/5}\;{\rm km/s.} \\
P_2 &= \frac{2}{\gamma+1}\,\rho_0 \dot{R}^2 \implies  6 \times 10^{-7} \left(\frac{E_0}{10^{51}\;\rm erg}\right)^{2/5} \left(\frac{\rho_0}{10^{-24}\;\rm g\;cm^{-3}}\right)^{3/5}\left(\frac{t}{1000\;\rm yr}\right)^{-6/5}\;{\rm erg/cm^3.} \\
T_2 &= \frac{m_p\mu}{k_B} \dot{R}^2 \frac{2(\gamma-1)}{(\gamma+1)^2} \implies 1\times 10^{8} \mu \left(\frac{E_0}{10^{51}\;\rm erg}\right)^{2/5} \left(\frac{\rho_0}{10^{-24}\;\rm g\;cm^{-3}}\right)^{-2/5}\left(\frac{t}{1000\;\rm yr}\right)^{-6/5}\;{\rm K}.
\end{align}

\subsubsection{Termination of Sedov--Taylor Expansion}

Eventually, the Sedov--Taylor (ST) description breaks down because t\textbf{he post--shock gas cools efficiently.  }
The key comparison is between the \textbf{cooling time} of the shocked gas and the \textbf{dynamical (expansion) time} of the remnant.

For a fully ionized plasma with \textbf{post--shock number density} $n_2$ and \textbf{temperature} $T_2$, the volumetric cooling rate is
\[
\mathcal{L} = n_e n_H \Lambda(T,Z) \sim n_2^2\,\Lambda(T_2,Z),
\]
where $\Lambda(T,Z)$ is the cooling function.  
The thermal energy density of the gas is
\[
u_{\rm th} = \frac{3}{2} n_2 k_B T_2,
\]
so the characteristic cooling timescale is
\begin{equation}
    t_{\rm cool} \equiv \frac{u_{\rm th}}{\mathcal{L}}
    \simeq \frac{3}{2} \frac{k_B T_2}{n_2 \Lambda(T_2,Z)}.
    \label{eq:t_cool_def}
\end{equation}

In the Sedov--Taylor phase, $R_{\rm sh}\propto t^{2/5}$, so
\begin{equation}
    t_{\rm dyn} = \frac{R}{\dot{R}} \sim \frac{5}{2}\,t.
    \label{eq:t_dyn_sedov}
\end{equation}
For a strong shock in a monatomic gas ($\gamma = 5/3$),
\[
\rho_2 = 4\rho_0, \qquad n_2 \simeq 4 n_0,
\]
and the post--shock temperature is
\begin{equation}
    k_B T_2 = \frac{3}{16}\,\mu m_p\,u_{\rm sh}^2,
    \label{eq:T2_strong_shock}
\end{equation}
where $u_{\rm sh}$ is the shock speed in the upstream rest frame, $\mu$ is the mean molecular weight, and $m_p$ is the proton mass.
From the Sedov--Taylor solution,
\begin{equation}
    R_{\rm sh}(t) = \xi_0 \left( \frac{E_0 t^2}{\rho_0} \right)^{1/5},
    \qquad
    u_{\rm sh}(t) = \frac{dR_{\rm sh}}{dt}
    = \frac{2}{5}\,\xi_0 \left(\frac{E_0}{\rho_0}\right)^{1/5} t^{-3/5},
    \label{eq:sedov_R_u}
\end{equation}
with $\xi_0 \approx 1.15$ for $\gamma=5/3$.
Substituting Eq.~\eqref{eq:T2_strong_shock} and $n_2\simeq 4n_0$ into Eq.~\eqref{eq:t_cool_def} gives
\begin{align}
t_{\rm cool}
&= \frac{3}{2} \frac{k_B}{n_2 \Lambda(T_2,Z)}
    \left(\frac{3}{16}\mu m_p u_{\rm sh}^2\right) \\
&\simeq \frac{9}{32}\,\frac{\mu m_p}{4 n_0\,\Lambda(T_2,Z)}\,u_{\rm sh}^2 \\
&= \frac{9}{128}\,\frac{\mu m_p}{n_0\,\Lambda(T_2,Z)}\,u_{\rm sh}^2.
\end{align}
For an order--of--magnitude estimate, we treat the cooling function as approximately constant over the relevant temperature range,
\[
\Lambda(T_2,Z) \approx \Lambda_0 \sim 10^{-22}\ {\rm erg\,cm^3\,s^{-1}}
\]
(for $T\sim 10^6$--$10^7$ K and near-solar metallicity). Then
\begin{equation}
t_{\rm cool}(t)
= \frac{9}{128}\,\frac{\mu m_p}{n_0 \Lambda_0}\,u_{\rm sh}^2(t).
\end{equation}
Using Eq.~\eqref{eq:sedov_R_u},
\[
u_{\rm sh}(t) = K\,t^{-3/5},
\qquad
K \equiv \frac{2}{5}\,\xi_0\left(\frac{E_0}{\rho_0}\right)^{1/5},
\]
we obtain
\begin{equation}
t_{\rm cool}(t)
= \underbrace{\left(\frac{9}{128}\,\frac{\mu m_p}{n_0\Lambda_0}K^2\right)}_{\equiv A}
  t^{-6/5}
= A\,t^{-6/5}.
\label{eq:t_cool_vs_t}
\end{equation}
Thus, during the Sedov phase the cooling time \emph{decreases} with time ($t_{\rm cool}\propto t^{-6/5}$), while the dynamical time grows linearly ($t_{\rm dyn}\propto t$).

The Sedov--Taylor approximation ceases to be valid when the shocked gas cools on a timescale comparable to the age:
\begin{equation}
t_{\rm cool}(t_{\rm rad}) \simeq t_{\rm dyn}(t_{\rm rad}).
\end{equation}
Using Eqs.~\eqref{eq:t_dyn_sedov} and \eqref{eq:t_cool_vs_t},
\[
A\,t_{\rm rad}^{-6/5} \;\simeq\; \frac{5}{2} t_{\rm rad},
\]
so that
\[
t_{\rm rad}^{11/5} \simeq \frac{2}{5} A
\quad\Rightarrow\quad
t_{\rm rad} \simeq
\left(\frac{2}{5}A\right)^{5/11}.
\]

Substituting $A$ and $K$ and writing $\rho_0 = \mu m_p n_0$, one finds the scaling
\begin{equation}
t_{\rm rad} \;\propto\;
E_0^{2/11}\,\rho_0^{-7/11}\,\Lambda_0^{-5/11}
\;\;\;\;\text{or}\;\;\;\;
t_{\rm rad} \;\propto\;
E_0^{2/11}\,n_0^{-7/11}\,\Lambda_0^{-5/11}.
\end{equation}

Evaluating the numerical coefficient for $\xi_0 = 1.15$, $\mu = 0.6$, and
$\Lambda_0 = 10^{-22}\,{\rm erg\,cm^3\,s^{-1}}$, we obtain
\begin{equation}
\boxed{
t_{\rm rad} \;\approx\;
1.6\times 10^4\,
\left(\frac{E_0}{10^{51}\,{\rm erg}}\right)^{2/11}
\left(\frac{\rho_0}{10^{-24}\,{\rm g\,cm^{-3}}}\right)^{-7/11}
\left(\frac{\Lambda_0}{10^{-22}\,{\rm erg\,cm^3\,s^{-1}}}\right)^{-5/11}
\;{\rm yr}.
}
\end{equation}
The corresponding radius is just the Sedov radius evaluated at $t_{\rm rad}$:
\[
R_{\rm rad} = R_{\rm sh}(t_{\rm rad})
= \xi_0\left(\frac{E_0 t_{\rm rad}^2}{\rho_0}\right)^{1/5}.
\]
Using the scaling $t_{\rm rad}\propto E_0^{2/11}\rho_0^{-7/11}\Lambda_0^{-5/11}$, we find
\begin{equation}
R_{\rm rad} \;\propto\;
E_0^{3/11}\,\rho_0^{-5/11}\,\Lambda_0^{-2/11},
\end{equation}
or, in terms of $n_0$,
\[
R_{\rm rad} \;\propto\;
E_0^{3/11}\,n_0^{-5/11}\,\Lambda_0^{-2/11}.
\]
Numerically,
\begin{equation}
\boxed{
R_{\rm rad} \;\approx\;
18\,
\left(\frac{E_0}{10^{51}\,{\rm erg}}\right)^{3/11}
\left(\frac{\rho_0}{10^{-24}\,{\rm g\,cm^{-3}}}\right)^{-5/11}
\left(\frac{\Lambda_0}{10^{-22}\,{\rm erg\,cm^3\,s^{-1}}}\right)^{-2/11}
\;{\rm pc}.
}
\end{equation}

\subsubsection{Emission Features}

The \textbf{emission properties} of the Sedov--Taylor phase differ radically from those
in free expansion because the ejecta have been shock-heated and the entire interior follows a
hot, high–pressure, self–similar flow.

The dominant emission arises from the \textbf{forward-shocked ISM}, which is heated to
\[
T_2 \sim 10^7\text{--}10^8~{\rm K}
\]
for typical SNR shock speeds.  As a result:
\begin{itemize}
    \item Thermal bremsstrahlung produces a strong continuum (0.1--10 keV).
    \item Collisionally ionized heavy elements produce strong line emission
          (e.g.\ O, Ne, Mg, Si, S, Fe).
\end{itemize}
Because the reverse shock has already thermalized the ejecta, the interior is filled with
hot, metal-enriched gas.  ST remnants therefore show:
\begin{itemize}
    \item centrally peaked or shell-like X-ray morphologies,
    \item enhanced heavy-element lines,
\end{itemize}

Non-thermal electrons accelerated at the forward shock produce synchrotron radiation in the
compressed magnetic field.  The radio luminosity generally:
\[
L_\nu \propto R_{\rm sh}^3\,n_0\,u_{\rm sh}^2,
\]
and tends to decline slowly as the shock decelerates.

During the ST stage, optical emission is relatively weak, since the temperatures are too high
for efficient line cooling.  Optical emission becomes more prominent only when the shock slows
to a few hundred km/s and the post-shock gas cools to $\sim10^4$ K, marking the beginning of
the radiative phase.

\subsection{Radiative (Snowplow) Phase}

As the remnant continues to expand beyond the Sedov–Taylor stage, the post-shock gas cools to progressively lower temperatures.  
During the Sedov phase the shock remains hot ($T_s \gtrsim 10^6\,$K), and the cooling time greatly exceeds the expansion time, \textbf{ensuring that radiative losses are dynamically negligible.}  
However, because the shock velocity decreases as $v_s \propto t^{-3/5}$, the post-shock temperature eventually falls into the regime where radiative cooling---particularly via metal line emission---becomes extremely efficient.
As we have seen, when the losses due to cooling are no longer negligible, we have \textbf{left the Sedov phase} and entered the \textbf{radiative phase} (snowplow phase).

The immediate consequence of rapid cooling is that the shocked interstellar material \textbf{collapses into a thin, dense shell just behind the blast wave.}
The shock itself continues to sweep up and thermalize ambient material, but the post-shock gas now loses its thermal energy almost instantaneously and is deposited into the shell at temperatures of order $10^2$--$10^4\,$K.
Behind this cold shell lies the still-hot interior plasma---the relic of the Sedov phase---which has not yet cooled appreciably.
The interface between the high-pressure interior and the \textbf{thin radiative shell is highly unstable}, and Rayleigh--Taylor fingers develop as the lighter, overpressured interior gas pushes outward against the heavy shell.
These instabilities contribute to the complex filamentary morphology seen in many middle-aged supernova remnants.

Because the cooled shell is extremely dense and thin, its thermal pressure is negligible compared to its bulk kinetic energy. The remnant can therefore be approximated as a \textbf{momentum-conserving “snowplow”} in
which the dynamics are governed by the conservation of radial momentum rather
than total energy.

\subsubsection{The Dynamics of the Shock}

In the thin-shell approximation the swept-up mass is \textbf{concentrated almost entirely in a shell of radius} $R(t)$ and negligible thickness. Its mass is
\[
M_{\rm sh}(t) \simeq \frac{4\pi}{3} R^3 \rho_0,
\]
and the radial momentum of the shell is
\[
p(t) = M_{\rm sh} v_{\rm sh}.
\]
Once radiative losses remove the pressure support of the hot gas, the interior pressure no longer performs significant work on the shell; \textbf{thus the momentum is approximately conserved:}
\[
\frac{dp}{dt} \approx 0 \quad \Rightarrow \quad M_{\rm sh} v_s = p_0,
\]
where $p_0$ is the momentum inherited from the end of the Sedov--Taylor phase. \rmk{We're effectively coasting to a stop at this point. All the thermal energy is used up and we're just waiting for the shock to stop.}

Substituting for $M_{\rm sh}$ and using $v_s = dR/dt$,
\[
p_0 = \frac{4\pi}{3} \rho_0 R^3 \frac{dR}{dt}.
\]
Solving for the expansion rate gives
\[
\frac{dR}{dt} = C R^{-3},
\qquad
C \equiv \frac{3p_0}{4\pi \rho_0}.
\]
Integrating,
\[
\frac{1}{4}\left(R^4 - R_0^4\right) = C(t - t_0),
\]
and for $t \gg t_0$ and $R \gg R_0$ this asymptotes to the canonical
\textbf{snowplow scaling},
\begin{equation}
\label{eq:radiative_scaling}
R(t) \propto t^{1/4},
\qquad
v_s(t) \propto t^{-3/4}.
\end{equation}

This evolution \textbf{is significantly steeper than that of the Sedov–Taylor phase,}
reflecting the loss of pressure support: the shell must decelerate more rapidly
because its inertia continues to grow as $R^3$ while there is no internal energy
reservoir to sustain its expansion.

\subsubsection*{Radiative and Observational Features}

The onset of rapid cooling dramatically alters the appearance of the remnant.
The dense shell becomes the dominant emitter across many wavelengths.
Cooling through metal line emission produces strong optical and near-infrared
lines such as H$\alpha$, [O~III], and [S~II], often observed as narrow,
bright filaments tracing the shock front.  
These lines are sensitive to the density, temperature, and ionization structure
of the shell, providing powerful diagnostics of the shock velocity and the
ambient medium.

As the magnetic field is compressed in the thin shell, synchrotron radio
emission becomes concentrated in a limb-brightened ring.
This morphology is a characteristic signature of radiative SNRs.
Meanwhile, the X-ray emission fades markedly because the outer layers cool below
X-ray emitting temperatures and the hot interior occupies an increasingly
diminished fraction of the volume.

\subsubsection*{Late-Time Evolution}

Eventually the shock decelerates to velocities comparable to the ambient sound
speed ($v_s \sim 10\,$km\,s$^{-1}$).  
At this point the shell is no longer overpressured relative to the
surrounding medium, and the SNR gradually dissolves into the ISM.
The distinct structure of the remnant is lost, and the system merges with and
contributes its energy, momentum, and enriched material to the interstellar
environment.

\subsection{Summary of Dynamical Evolution}

\begin{center}
\begin{tabular}{lccc}
\toprule
\textbf{Phase} & \textbf{Dominant Energy} & \textbf{Radius Scaling} & \textbf{Typical Duration} \\
\midrule
Free Expansion & Kinetic ($E_{\rm kin}$) & $R \propto t$ & $\sim 10^2$ yr \\
Sedov--Taylor & Adiabatic (thermal) & $R \propto t^{2/5}$ & $\sim 10^4$ yr \\
Radiative (Snowplow) & Momentum-driven & $R \propto t^{1/4}$ & $\sim 10^6$ yr \\
\bottomrule
\end{tabular}
\end{center}

\begin{remark}
While these analytic regimes capture the broad evolutionary sequence of an SNR, real remnants can deviate substantially depending on the progenitor’s mass-loss history, the density and structure of the surrounding medium, and magnetic or cosmic-ray pressure contributions. Multi-wavelength observations (radio, optical, X-ray) are therefore essential for identifying which stage a particular remnant occupies.
\end{remark}

\section{The Morphology of Supernova Shocks}

Once the supernova shock has escaped the stellar progenitor, it generically separates into four dynamical regions:

\begin{enumerate}
    \item The \textbf{unshocked ISM (region I)} — the ambient interstellar or circumstellar gas, initially at rest with density $\rho_0$ and pressure $P_0$.
    \item The \textbf{shocked ISM (region II)} — material swept up by the forward shock and heated to high temperatures ($T \sim 10^{6}$–$10^{8}$ K).
    \item The \textbf{shocked ejecta (region III)} — ejecta that have passed through the reverse shock, now moving subsonically in the local frame.
    \item The \textbf{unshocked ejecta (region IV)} — freely expanding stellar material, often still cold and dense relative to the shocked regions.
\end{enumerate}

Between regions (II) and (III) lies a \textbf{contact discontinuity} that separates the shocked ISM from the shocked ejecta. The two shocks (forward and reverse) and the contact surface evolve self-consistently as the explosion expands into the surrounding medium.

\subsection*{Jump Conditions Across the Shock}

The jump conditions across a strong adiabatic shock (Mach number $M \gg 1$) follow from the conservation of mass, momentum, and energy (\textbf{Rankine–Hugoniot Conditions}):
\[
\begin{aligned}
\rho_1 v_1 &= \rho_2 v_2, \\
P_2 + \rho_2 v_2^2 &= P_1 + \rho_1 v_1^2, \\
\frac{1}{2}v_1^2 + \frac{\gamma}{\gamma-1}\frac{P_1}{\rho_1} &= \frac{1}{2}v_2^2 + \frac{\gamma}{\gamma-1}\frac{P_2}{\rho_2}.
\end{aligned}
\]
Here subscripts 1 and 2 refer to pre- and post-shock quantities respectively. Solving these for a given adiabatic index $\gamma$ yields the compression, pressure, and temperature jumps across the shock.

For a strong shock in a monatomic gas ($\gamma = 5/3$), one finds:
\[
\frac{\rho_2}{\rho_1} = \frac{\gamma+1}{\gamma-1} = 4,
\]
\[
\frac{P_2}{P_1} = \frac{2\gamma M_1^2 - (\gamma - 1)}{\gamma + 1} \approx \frac{2\gamma}{\gamma + 1} M_1^2 \quad (M_1 \gg 1),
\]
and thus for $\gamma=5/3$,
\[
\frac{P_2}{P_1} \approx \frac{5}{4} M_1^2.
\]
Finally, the post-shock temperature is given by
\[
T_2 \approx \frac{3}{16}\frac{\mu m_p}{k_B} v_s^2.
\]
For $v_s = 10^4\,{\rm km\,s^{-1}}$, this yields $T_2 \sim 10^9\,{\rm K}$, sufficient to produce strong thermal bremsstrahlung and line emission in the X-ray band.

\begin{remark}
The \textbf{reverse shock} arises naturally from momentum conservation. As the forward shock plows into the interstellar medium, it continuously sweeps up mass. The added inertia causes the outer layers of the expanding ejecta to decelerate. However, the inner ejecta continue to expand ballistically at higher velocities. The result is a ``pile-up'' of material at the interface, driving a compression wave \emph{backward} (in the ejecta frame). This reverse shock travels inward through the ejecta, converting their kinetic energy into heat and pressure.

At early times, the reverse shock is relatively weak and located near the outer ejecta. As the remnant evolves and the swept-up mass increases, the reverse shock moves progressively inward in mass coordinate, eventually \textbf{thermalizing much of the ejecta’s energy}. The region between the reverse and forward shocks becomes a hot, high-pressure bubble that dominates the dynamics of the remnant during the Sedov–Taylor phase.
\end{remark}


\section{Self-Similar Models}

Having surveyed the broad dynamical features of supernova shocks, we now turn to the more ambitious task of \textbf{modeling evolution} of these shocks.
Our strategy is to use \textbf{self-similar solutions} to characterize the structure of the shocked and unshocked gas immediately following shock breakout.  Only once this structure has been established can we meaningfully predict the radio and X-ray emission powered by the shock.

\subsection{Initial Conditions}

To set the stage, we begin by identifying the \textbf{initial conditions} for the self-similar interaction problem.  
As in the classical theory of blast waves, the supernova--circumstellar 
interaction naturally divides into four dynamical regions:

\begin{enumerate}

\item \textbf{Region I: Unshocked ejecta}.  
Immediately after breakout, the ejecta are in the phase of \textbf{free expansion}.  
Pressure gradients are negligible and each fluid element moves ballistically with
a \textbf{homologous velocity field}
\begin{equation}
    u_1(r,t) = \frac{r}{t},
\end{equation}
which fixes the form of the density profile:
\begin{equation}
    \label{eq:homologous_ejecta_density}
    \boxed{
    \rho_1(r,t) = t^{-3}\,F\!\left(\frac{r}{t}\right),
    }
\end{equation}
where \textbf{the structural function \(F\) is determined by the explosion physics.}  
Its explicit form will become important later, but for now we leave it general.  
Throughout this region we may neglect the pressure ($P_1 \approx 0$).

\item \textbf{Region II: Shocked ejecta}.  
As the ejecta expand outward, the interaction with the ambient medium drives a \textbf{forward shock} into the circumstellar gas.  The reaction to this is the formation of a \textbf{reverse shock} that propagates inward (in mass coordinates) into the freely expanding ejecta.  Material that has passed through this reverse shock constitutes Region~II.  Its detailed structure is determined by the Rankine--Hugoniot jump conditions applied to Region~I and, in the full solution, by the self-similar hydrodynamic equations.

\item \textbf{Region III: Shocked ambient medium}.  
This region lies between the forward shock and the contact discontinuity.  The upstream ambient medium (Region~IV) enters the forward shock \textbf{cold} and \textbf{at rest}, after which it is compressed, heated, and 
accelerated.  In the full self-similar framework this region is described by dimensionless 
profiles of density, velocity, and pressure, but its broad characteristics are 
set by the strong-shock jump conditions.

\item \textbf{Region IV: Unshocked ambient medium}.  
This is the external interstellar or circumstellar material into which the explosion expands.  
It is assumed to be cold, static, and well-described by a radial density law
\[
    \rho_4(r) = A_4\, r^{-s},
\]
where the index \(s\) encodes the nature of the surrounding environment:  
\(s=0\) for a uniform ISM and \(s=2\) for a steady stellar wind.  
More refined models of the circumstellar medium will be introduced below.
\end{enumerate}

These four regions, and the shocks and discontinuity that separate them, provide
the physical scaffolding on which both the self-similar dynamics and the
emission models are built.  
In what follows, we will develop the dynamical description of Regions~II and~III,
laying the groundwork needed to compute the resulting radio and X-ray emission.

\subsubsection{Models of the ISM: Region IV}

A supernova shock does not form in isolation. The expanding ejecta must sweep up and interact with \textbf{material surrounding the star}, and the density structure of this material provides the upstream conditions for the forward shock. There are several physically motivated scenarios to consider.

In the simplest possible setup, the progenitor system undergoes negligible mass loss prior to explosion. In this case, the supernova expands into a uniform interstellar medium with 
\[
\rho_{\rm ISM} = \rho_0 = \text{constant}.
\]
This is a mathematically convenient case and admits a particularly clean self-similar solution (the Sedov–Taylor solution), though it is generally \emph{not} the most realistic environment for massive-star core-collapse events.

A more realistic scenario for Type~II supernovae is that the immediate environment is not pristine ISM, but rather an \textbf{intermediate circumstellar medium} (ICM) formed by \textbf{stellar ejecta} from the progenitor itself. Massive stars---particularly red supergiants---lose substantial mass during the late stages of their evolution, ejecting material at a rate $\dot{M}(t)$ with an asymptotic wind velocity $v(t)$. 

This stellar wind fills the region around the star with material long before the explosion occurs. Our goal is to determine the density structure $\rho(r)$ of this wind-driven medium.

In the most general case, solving for $\rho(r)$ is non-trivial. A complete treatment would require solving the time-dependent fluid equations with a specified equation of state, radiative cooling, and an acceleration mechanism (e.g., thermal driving, line driving, dust driving). These equations typically do \emph{not} admit closed-form analytic solutions for arbitrary $\dot{M}(t)$ and $v(t)$, and shocks may form within the wind itself if faster material overtakes slower ejecta.

For core-collapse supernova progenitors, however, a major simplification is both physically appropriate and extremely useful. Far from the stellar surface the wind is typically \textbf{coasting}, having already achieved its terminal velocity. Pressure gradients, radiative acceleration, and gravity are negligible compared to the bulk inertia of the flow. In this regime, the wind behaves as a set of non-interacting mass shells expanding freely at constant speed. This motivates the following definition.

\begin{definition}[Ballistic Wind]
A \emph{ballistic wind} is a stellar outflow in which each parcel of gas, once ejected from the stellar surface, moves outward at a constant velocity \(v_w\) with negligible pressure forces, gravitational forces, or interactions with other parcels. The flow is therefore kinematic, characterized entirely by the mass-loss rate \(\dot{M}\) and terminal velocity \(v_w\).
\end{definition}

Under these assumptions, the wind can be modeled analytically. A mass shell ejected over a time interval $\delta t$ contains a mass 
\[
\delta M = \dot{M}\,\delta t.
\] 
After a time $t$ this shell occupies a spherical layer at radius $R = v_w t$ and thickness $v_w \delta t$, giving a density
\[
\rho = \frac{\delta M}{4\pi R^2 (v_w \delta t)}
     = \frac{\dot{M}}{4\pi v_w R^2}.
\]
Thus the circumstellar density profile becomes
\begin{equation}
\rho(R) = \frac{\dot{M}}{4\pi v_w R^2} 
\quad \propto \quad R^{-2}.
\end{equation}

This result is remarkably robust: \emph{any} steady, spherically symmetric, coasting wind yields a density profile falling as $R^{-2}$. For this reason, models of supernova shock propagation often adopt a generalized ambient medium of the form
\begin{equation}
\rho(R) \propto R^{-s},
\end{equation}
where $s = 2$ corresponds to a steady wind, and $s = 0$ corresponds to a constant ISM. This $R^{-s}$ formalism captures a wide variety of physical environments and allows the self-similar shock solutions developed above to be applied in a unified manner.

\subsubsection{Models of the Ejecta: Region I}

\begin{remark}
    As in the previous section, there is a great deal of detailed physics---both observational and theoretical---underlying the structure of supernova ejecta. For our purposes, however, we restrict attention to the essential ingredients needed to construct shock--interaction and emission models.
\end{remark}

Immediately following core collapse and shock revival, the outgoing shock must \textbf{break out} of the progenitor envelope.  The specifics of this breakout phase depend sensitively on the progenitor structure, the explosion mechanism, and the transport of radiation through the stellar envelope.  
However, once the shock has emerged from the surface, the material behind it expands freely into space with negligible pressure support.  
This early-time regime is referred to as the phase of \textbf{free expansion}.

The freely expanding ejecta provide the upstream conditions for both the forward and reverse shocks, and ultimately determine the morphology, energetics, and radiative output of the system.  
It is therefore essential to establish the general form of the ejecta density profile consistent with \textbf{homologous expansion} and the \textbf{continuity equation}.

\textbf{In the free-expansion limit}, pressure gradients are negligible and each fluid element moves ballistically with a velocity proportional to its radius:
\begin{equation}
    v(r,t) = \frac{r}{t}.
\end{equation}
This defines a \emph{homologous} velocity field: each fluid element carries a fixed velocity label~$v$, and its radius simply scales as $r(t) = v\,t$.  
The entire flow therefore expands self-similarly.

Applying mass conservation to a homologously expanding flow yields a density profile of the form
\begin{equation}
    \label{eq:density_self_similar}
    \boxed{
        \rho(r,t) = t^{-3}\,F(v)
        \;=\;
        t^{-3}\,F\!\left(\frac{r}{t}\right),
    }
\end{equation}
where $F$ encodes the intrinsic structure of the ejecta as a function of velocity.  
The explicit factor $t^{-3}$ reflects the dilution of mass as the flow expands in three dimensions.
\rmk{We derived this in more detail above in the section on dynamics.}

A remarkably robust feature of explosion models is that t\textbf{he \emph{outer} ejecta approach a steep power-law decline in velocity space.}
This behavior arises from the strong shock acceleration that occurs as the explosion propagates through the decreasing density of the stellar envelope.  
A widely used parametrization is therefore
\begin{equation}
    F(v) = A\,v^{-n}
    \quad\Rightarrow\quad
    \rho(r,t)
    = A\,t^{-3}\left(\frac{r}{t}\right)^{-n},
\end{equation}
where $n$ is \textbf{the \emph{outer density index}.}  
Typical values inferred from stellar-evolution and explosion models lie in the range $n \approx 8$--$12$ for Type~II progenitors.  
The constant $A$ is fixed by the total ejecta mass and kinetic energy.

This outer power-law envelope provides the initial condition for the subsequent interaction with the circumstellar medium.  
As we will see, in combination with an ambient density profile $\rho_{\rm CSM} \propto r^{-s}$, it leads naturally to the self-similar double-shock structure and scaling relations that underpin models of radio and X-ray supernova emission.

\subsection{The Self Similar Solution}

Now that we have established the \textbf{initial conditions}, we can go ahead and solve the problem. Here's the basic idea: we're going to use \textbf{region I} and \textbf{region IV} as \textbf{boundary conditions} and then seek \textbf{self-similar solutions} for the interior region of the shocked material (regions III and II). There are, in fact, multiple ways to achieve this. If one cares about the detailed structure of regions II and III, then a full treatement as described in \citet{chevalier_self-similar_1982} is necessary. Fortunately, in this scenario, we are going to rely on a foundational hypothesis: \textbf{the shocked material forms a thin shell}.

By making that assumption, we are able to assume that the \textbf{contact discontinuity}, the \textbf{forward shock front}, and the \textbf{reverse shock front} all coincide! We can then assign only single values of the flow fields to regions II and III and ignore any of the internal detail. This is the approach described in \citet{chevalier_radio_1982}, who's derivation we will now follow.

\subsubsection{The Position of the Shock}

Our first task is a relatively simple one: \textbf{we need to know the location of the shock}. To do this, we'll take advantage of a very useful idea:
\vspace{10pt}
\begin{fact}
    \textbf{Self-similarity requires that no dimensionless quantity contain explicit 
time-dependence.} In particular, the ratio of the upstream densities at the contact must remain 
constant.
\end{fact}
\vspace{10pt}
Let's assume that 
\[
\frac{\rho_1(R_c)}{\rho_4(R_c)} = C,
\]
where $C$ is a constant. Since we know that $\rho_1(r,t) = \varrho_1 t^{-3} (r/t)^{-n}$ (\rmk{we use $\varrho$ for the normalization constant}), and $\rho_4(r,t) = \varrho_4 r^{-s}$, we have the requirement that
\[
\frac{\varrho_1}{\varrho_4} t^{n-3} R_c^{s-n} = C \implies R_c^{s-n} = C \frac{\varrho_4}{\varrho_1} t^{3-n}.
\]
As such,
\begin{equation}
    \boxed{
    R_c(t) = \left(\frac{C\varrho_4}{\varrho_1}\right)^{1/(s-n)} t^{(3-n)/(s-n)} = R_0 t^\lambda,
    }
\end{equation}
where $\lambda$ is the \textbf{similarity exponent}. We can also state the \textbf{shock velocity}
\[
\dot{R}_c(t) = R_0 \lambda t^{\lambda -1} = \frac{\lambda R}{t}.
\]

\subsubsection{The Component Masses}

Now, we can clearly see that this discontinuity may be \textbf{acceleration} depending on the scenario. In order for this to occur, the \textbf{pressure gradient} across the two shocked regions must self-consistent. In order to conserve momentum,
\[
\frac{dp}{dt} = \frac{d}{dt}\left([M_1+M_2] \dot{R}_c\right)= 4\pi R_C^2 (P_3 -P_2).
\]
The masses in the shells are determined by the amount of swept up material. Region 3 sweeps up the \textbf{ISM/ICM}, which means that in a time $dt$, it gains
\[
dM_3 = 4\pi R^2 \dot{R}_C\rho_4(R,t) \;dt.
\]
Since $dR_c = \dot{R}_c dt$,
\[
dM_3 = 4\pi \varrho_4 R_c^{2-s}\; dR_c \implies M_3 = \frac{ 4\pi \varrho_4 }{3-s} \left(R_c^{3-s} - R_{c,0}^{3-s}\right).
\]
Given that $R_0$ should be a subdominant contribution, we can write now an equation for the \textbf{mass contained in region 3}:
\begin{equation}
    \boxed{
    M_3(R_c) = \frac{4\pi \varrho_4}{3-s} R_c^{3-s}.
    }
\end{equation}
Likewise, the material in $M_2$ is \textbf{post-shock ejecta}, which we can calculate on the basis that all of the material with $u_{\rm ej} t > R_c$ should be swept up. Recall that
\[
\rho_1(r,t) = \varrho_1 t^{-3} \left(\frac{r}{t}\right)^{-n},
\]
with a velocity field
\[
u_1(r,t) = \frac{r}{t}.
\]
Therefore, at any time $t$, the material beyond $R_c$ is
\[
M_2(t) = \int_{R_c}^{R_{\rm max}} 4\pi \xi^2\;\rho_1(\xi,t)\;d\xi.
\]
Here, $R_{\rm max} = v_{\rm ej,\;max} t$, so
\[
M_2(t) = 4\pi \varrho_1 t^{-3}\int_{R_c}^{R_{\rm max}} \xi^{2-n} \;d\xi = \frac{4\pi \varrho_1 t^{n-3}}{3-n} \left[(u_{\rm max} t)^{3-n} - R_c^{3-n}\right].
\]
Since we generally are interested in $n > 3$, the \textbf{dominant term} is
\begin{equation}
\boxed{
    M_2(t) \simeq \frac{4\pi \varrho_1 t^{n-3}}{n-3} R_c^{3-n}. 
    }
\end{equation}

\subsubsection{Shock Relations}

We also know things about the \textbf{density and pressure} inside of the shocked regions from \textbf{Rankine-Huguenot}. Let's assume \textbf{strong shocks} and determine the scalings for the \textbf{forward shock} first. In this case, we have region 3 propogating into a \textbf{cold ISM} with $T_4 \sim 0$ and $P_4 \sim 0$.

First on our list is the \textbf{density} of region III. We know that
\[
\frac{\rho_3}{\rho_4} = 4
\]
With the full scalings, we know that $\rho_3 = 4\varrho_4R_c^{-s}=4R_0^{-s}\varrho_4 t^{-\lambda s}$. We can also compute the \textbf{velocity} of region 3 in the rest frame of the ISM, which is
\[
u_3 = \frac{3}{4} \dot{R}_c = \frac{3}{4}R_0 \lambda t^{(\lambda-1)}.
\]
From the \textbf{pressure condition},
\[
P_3 = \rho_4u_4^2-\rho_3u_3^2 = \frac{3}{4} \rho_4 u_4^2 =\frac{3}{4} \varrho_4 R_0^{2-s} \lambda^2 t^{(2-s)\lambda -2}.
\]
 
For the \textbf{reverse shock}, the velocity is still $\dot{R}_c$. The \textbf{downstream density} is
\[
\rho_2(t) = 4\rho_1(R_c,t) = 4 \varrho_1 t^{-3} \left(\frac{R_c}{t}\right)^{-n} = 4\varrho_1 t^{n-3} R_0^{-n} t^{-\lambda n}.
\]
As such, the \textbf{density ratio} between the two regions is
\begin{equation}
    \boxed{
    \frac{\rho_2}{\rho_3} = \frac{4\varrho_1 t^{n-3} R_0^{-n} t^{-\lambda n}}{4\varrho_4 R_0^{-s}t^{-\lambda s}} = \frac{\varrho_1}{\varrho_4} R_0^{s-n} t^{\lambda(s-n)} t^{n-3} = \frac{\varrho_1}{\varrho_4} R_0^{s-n}.
    }
\end{equation}
\rmk{We have used here the definition of $\lambda$.}

In the \textbf{rest-frame of the shock}, the ejecta moves at
\[
u_{1,\rm rel} = u_1 - \dot{R}_c = \frac{R_c}{t} - \lambda \frac{R_c}{t} = (1-\lambda)\frac{R_c}{t}.
\]
As such, the \textbf{relative downstream velocity is}
\[
u_{2,\rm rel} = -\frac{1}{4} u_{1,\rm rel} = -\frac{(1-\lambda)}{4}\frac{R_c}{t}.
\]
From that, we have 
\[
u_2 = u_{2,\rm rel} + \dot{R}_c = \left(\frac{5\lambda -1}{4}\right) \frac{R_c}{t}.
\]
The resulting ratio in the velocities is
\begin{equation}
    \boxed{
    \frac{u_2}{u_3} = \frac{5\lambda-1}{3\lambda}.
    }
\end{equation}
\rmk{Intuitively, we might try to insist that this be unity; however, we are only using this to constrain the velocities \textbf{at the shocks}, not at the contact discontinuity.}

Finally, the pressure will be 
\[
P_2 = \frac{3}{4}\rho_1 u_{1,\rm rel}^2 = \frac{3}{4}\varrho_1 t^{n-3} R_c^{-n} \left(1-\lambda\right)^2 R_c^2 t^{-2}
\]
Using the definition of $R_c$ and simplifying,
\[
P_2 = \frac{3}{4} \left(1-\lambda\right)^2 \varrho_1 t^{n-5} R_c^{2-n} = \frac{3}{4} R_0^{2-n} \varrho_1 (1-\lambda)^2 t^{n-5+\lambda(2-n)}. 
\]
The resulting pressure ratio is
\[
\frac{P_2}{P_3} = \frac{\varrho_1}{\varrho_4}\frac{\left(1-\lambda\right)^2}{\lambda^2} R_0^{s-n} t^{\lambda(s-n)} t^{(n-3)}.
\]
Now, substituting in $\lambda$, we have
\begin{equation}
    \boxed{
    \frac{P_2}{P_3} = \frac{\varrho_1}{\varrho_4} \left(\frac{s-3}{3-n}\right)^2 R_0^{s-n}.
    }
\end{equation}

Having now accomplished these relations, we can proceed to determine the all important $R_0$ value. We have the requirement that
\[
\frac{dp}{dt} = \frac{d}{dt}\left([M_2+M_3] \dot{R}_c\right)= 4\pi R_C^2 (P_3 -P_2).
\]
Clearly, the total mass is
\[
M_2 + M_3 = 4\pi\left[\frac{\varrho_4}{3-s}R_c^{3-s} + \frac{\varrho_1 t^{n-3}}{n-3}R_c^{3-n}\right],
\]
If we substitute our ansatz for $R_c$, 
\[
M_{\rm sh} = 4\pi \left(\frac{\varrho_4}{3-s} R_0^{3-s}t^{\lambda(3- s)} + \frac{\varrho_1}{n-3} R_0^{3-n} t^{(\lambda-1)(3-n) }\right).
\]
With $\lambda = (n-3)/(n-s)$, this becomes
\begin{equation}
\boxed{
    M_{\rm sh} =M_{\rm sh,0} t^{\gamma}= 4\pi t^{\gamma} \left[\frac{\varrho_4}{3-s}R_0^{3-s} + \frac{\varrho_1}{n-3} R_0^{3-n}\right],\;\gamma = \frac{(n-3)(3-s)}{n-s}. 
    }
\end{equation}

The pressure differential is
\[
P_3 -P_2 = P_3\left(1-\frac{P_2}{P_3}\right) = P_3 \left(1- \frac{\varrho_1}{\varrho_4} \left(\frac{s-3}{3-n}\right)^2 R_0^{s-n}\right) =P_3\chi,
\]
where $\chi$ is simply a constant dependent on the initial conditions and flow parameters. 

If we now turn our attention to the momentum equation, we have

\[
\frac{d}{dt}\left(M_{\rm sh,0} t^{\gamma} \dot{R}_c\right) = 4\pi R_c^2 P_3\chi,
\]
which means that
\[
M_{\rm sh,0} R_0 \lambda \frac{d}{dt} (t^{\gamma} t^{\lambda -1}) = M_{\rm sh,0}R_0 \lambda (\gamma + \lambda -1) t^{\gamma + \lambda -2} = 4\pi R_0^2 t^{2\lambda} P_3\chi.
\]
Making the substitution for $P_3$, 
\[
M_{\rm sh,0}R_0 \lambda (\gamma + \lambda -1) t^{\gamma + \lambda -2} = 3 \pi \chi  \varrho_4 R_0^{4-s} \lambda^2 t^{(4-s)\lambda -2}.
\]
The \textbf{time dependence cancels}, and we have
\[
M_{\rm sh,0} R_0 \lambda (\gamma +\lambda - 1) = 3\pi \chi \varrho_4 R_0^{4-s} \lambda^2.
\]
Our goal now must be to write down an equation for $R_0$ in terms of the various constants in the problem, which is a pretty impressive achievement given the ammount of work that has gone into this manipulation. In the end, this equation will end up requiring an \textbf{algebraic closure} via a numerical solution unless there is a known closed form. 
\section{Radio Emission from Supernova Shocks}

Supernova blast waves drive strong forward shocks into the circumstellar medium (CSM).  These shocks amplify the magnetic field and accelerate electrons into a nonthermal power-law
distribution.  The resulting synchrotron emission produces the radio luminosity observed from young supernovae.  Our objective in this section is to understand how a single synchrotron SED (spectral energy distribution) allows us to infer key shock properties such as the radius, magnetic field, electron density,
and shock velocity.

\subsection{Features of Synchrotron Emission}

The first important component of this modeling is a basic understanding of synchrotron emission in a theoretical sense. More detailed notes are included in my notebook on radiative processes, but we include some highlights here.

\subsubsection{The Characteristic Frequency}

A single relativistic electron radiating in a magnetic field \textbf{emits most of its synchrotron power} near the characteristic frequency
\begin{equation}
    \nu_{\rm syn} =
    \frac{3}{4\pi}\,\gamma^2\,\frac{qB}{m_e c}\,\sin\alpha,
\end{equation}
where $\gamma$ is the electron Lorentz factor and $\alpha$ is the pitch angle.  For a population of power-law electrons, the highest energy electrons \textbf{determine the high-frequency cutoff}, while the lowest energy electrons would determine \textbf{the low-frequency turnover} in the absence of additional absorption processes.

This characteristic frequency arises from Fourier theory and the beaming effects that are associated with synchrotron. Effectively, the synchrotron beam is highly localized temporally, meaning that it must extend in frequency space out to about $\nu_{\rm syn}$. Beyond this, we expect an \textbf{exponential cutoff}.

\subsubsection{The Cooling Frequency}

Electrons lose energy via synchrotron emission at a rate
\begin{equation}
    P_{\rm syn} =
    \frac{4}{3}\sigma_T c\,\beta^2\gamma^2 U_B,
    \qquad
    U_B = \frac{B^2}{8\pi}.
\end{equation}
The corresponding cooling timescale is
\begin{equation}
    t_{\rm syn}
    = \frac{3m_e c}{4\sigma_T \beta^2 \gamma U_B},
\end{equation}
so equating this with the dynamical time $t_{\rm dyn}$ yields the \textbf{cooling Lorentz factor}
\begin{equation}
    \gamma_{\rm cool}
    \sim
    \frac{3 m_e c}{4\sigma_T \beta^2 U_B t_{\rm dyn}}.
\end{equation}
This corresponds to a cooling frequency
\begin{equation}
    \nu_{\rm cool} \sim
    \frac{27}{\sigma_T^2 \beta^4}
    \frac{q m_e c \pi}{B^3 t_{\rm dyn}^2}
    \sin\alpha.
\end{equation}
Above $\nu_{\rm cool}$, synchrotron cooling \textbf{modifies the spectrum.} Over time, we will lose synchrotron power at those frequencies, creating another cutoff.

\subsubsection{Synchrotron Self-Absorption}

A key feature of radio supernova spectra is \textbf{synchrotron self-absorption} (SSA).  For a power-law electron distribution $N(\gamma)=K\gamma^{-p}$, Rybicki \& Lightman (1986) and Pacholczyk (1970) show that the absorption coefficient is
\begin{equation}
    \alpha_\nu =
    c_6(p)\,
    K\,B^{(p+2)/2}\,
    \nu^{-(p+4)/2},
\end{equation}
where $c_6(p)$ is a known function of $p$, containing the relevant Gamma-function combinations.  The optical depth is
\begin{equation}
    \tau_\nu = \alpha_\nu R,
\end{equation}
and the self-absorption frequency $\nu_{\rm sa}$ is defined by
\begin{equation}
    \tau_{\nu_{\rm sa}} = 1
    \quad\Longrightarrow\quad
    \nu_{\rm sa}^{(p+4)/2}
    = c_6(p)\,K\,B^{(p+2)/2}R.
\end{equation}
The synchrotron emissivity for the same power-law distribution is
\begin{equation}
    j_\nu =
    c_5(p)\,
    K\,B^{(p+1)/2}\,
    \nu^{-(p-1)/2},
\end{equation}
where $c_5(p)$ is the corresponding emissivity coefficient from Pacholczyk (1970).  The synchrotron source function is therefore the ratio
\begin{equation}
    S_\nu = \frac{j_\nu}{\alpha_\nu}
    = c_1^{-1/2}\,
      \frac{c_5(p)}{c_6(p)}\,
      (B\sin\alpha)^{-1/2}\,
      \nu^{5/2}.
\end{equation}
This universal $\nu^{5/2}$ scaling reflects the fact that self-absorbed synchrotron emission behaves like a
brightness-temperature-limited ``pseudo-blackbody".

For a homogeneous emitting region of radius $R$ and thickness $s$, the emergent specific intensity is
\begin{equation}
    I_\nu = S_\nu\,J(y) = S_\nu(1-e^{-\tau_\nu}),
\end{equation}
where $y \equiv \nu/\nu_{\rm sa}$ and the dimensionless function $J(y)$ captures the transition between optically thick and thin regimes.  For a uniform slab,
\begin{equation}
    J(y) = 1 - \exp\!\left[-\,y^{-(p+4)/2}\right],
\end{equation}
so that
\begin{align}
    y \ll 1:
    &\qquad I_\nu \approx S_\nu,
    \\
    y \gg 1:
    &\qquad I_\nu \approx
    S_\nu\,y^{-(p+4)/2}
    \propto
    \nu^{-(p-1)/2}.
\end{align}
These two asymptotic limits generate the characteristic broken power-law synchrotron SED.

If the emitting region is a thin spherical shell of radius $R$ and thickness $s$, its volume is
\begin{equation}
    V = \pi R^2 s.
\end{equation}
It is common in the radio supernova literature to describe this in
terms of a filling factor $f$, defined so that
\begin{equation}
    V = f \cdot \frac{4\pi R^3}{3}.
\end{equation}
Typical values are $f \sim s/R \ll 1$ for a geometrically thin
shell.

\subsection{The Magnetic Field}

We know that the magnetic field from the progenitor star is \textbf{not strong enough} to produce the magnetic fields that result in electron acceleration and emission. As such, various microphysical processes must be postulated to produce / amplify the magnetic field. The primary source expected here is \textbf{Rayleigh-Taylor instability driven dynamo processes}. 

Fortunately, we don't really need to understand the microphysics of the magnetic field amplification. Instead, we introduce the \textbf{magnetic parameter} $\epsilon_B$, which determines the faction of the thermal energy contained in the magnetic field:
\begin{equation}
    \boxed{
    u_B = \frac{B^2}{8\pi} = \epsilon_B u_{\rm th}.
    }
\end{equation}
In general, we know $\epsilon_B$ to be quite small, on the order of $10^{-3}$ or $10^{-4}$. With this, we are able to sufficiently model the magnetic field. A useful feature of this picture is that the synchrotron emission should come \textbf{predominantly from the forward shocked material}, which has undergone $B$ field amplification. If we model the shock as a strong shock, we have the typical result that
\[
u_{\rm th} = \frac{9}{8} \rho_{\rm 4}(R_c) \dot{R}_c^2.
\]
Presuming the above coupling, we have
\begin{equation}
    \boxed{
    u_{B} = \frac{9}{8}\epsilon_B \rho_4(R_c) \dot{R}_c^2.
    }
\end{equation}

\subsection{The Electron Population}

The classical mechanism by which to produce \textbf{relativistic electrons} is \textbf{diffusive shock acceleration}, which is discussed in detail in the next chapter. The big takeaway from this is that the electrons should populate an \textbf{effective power-law} such that
\[
N(E) dE = K E^{-p} \;dE
\]
for some $p$. As we will see later, this $p$ is strictly dependent on the power-law index for the radio emission. We also commonly use a distribution in the \textbf{Lorentz Factor} instead:
\[
N(\Gamma) \;d\Gamma = K_\Gamma \Gamma^{-p} \;d\Gamma.
\]
Since not all $\Gamma$ can produce synchrotron, we generally introduce some $\Gamma_{\rm min}$ and $\Gamma_{\rm max}$ to truncate the distribution. The total energy is
\[
E_{e} = K_\Gamma m_e c^2\int_{\Gamma_{\rm min}}^{\Gamma_{\rm max}} \Gamma^{-p}\;d\Gamma = \frac{K_\Gamma m_ec^2}{1-p} \left[\Gamma_{\rm min}^{1-p} - \Gamma_{\rm max}^{1-p}\right] 
\]
We can use this to normalize the distribution. Generically, we assume that the electron energy density is also some portion of the thermal energy density, so 
\begin{equation}
\boxed{
    E_e = \epsilon_E u_{\rm thr} = K_\Gamma m_e c^2 \int_{\Gamma_{\rm min}}^{\Gamma_{\rm max}} \Gamma^{-p}\;d\Gamma.
    }
\end{equation}
Another important feature is the \textbf{mean Lorentz factor}, which comes up in a number of scenarios. Clearly
\[
\overline{\gamma} = \frac{\int_{\gamma_{m}}^{\infty} \Gamma^{1-p} d\Gamma }{\int_0^\infty \Gamma^{-p}\;d\Gamma} = \frac{1-p}{2-p} \gamma_m.
\]
This, we often argue that
\[
\bar{\gamma} \sim \gamma_m.
\]
If each particle gets some
\[
\epsilon_e \frac{u_{\rm thrm}}{n} = \epsilon_e v_{\rm shock}^2 m_p \approx \bar{\gamma} m_e c^2
\]

\subsection{Radio Emission: Phenomenology}

Before discussing the modeling of the radio emission, it is worth first discussing the qualitative features of the radio emission both from theory and from observational constraint. 

\subsubsection{The Radio Lightcurve}

Radio observations of young supernovae exhibit a remarkably characteristic lightcurve morphology.  At essentially all frequencies, the flux density $F_\nu(t)$ shows a \emph{rapid rise} to peak brightness followed by a more gradual \emph{power-law decline}.  
This qualitative shape arises naturally from the expansion of a synchrotron-emitting
shock into an ionized circumstellar medium (CSM).  
Although the general behavior is universal among radio supernovae,
the details---in particular the time of the peak and the height of the peak---are
strongly frequency dependent, reflecting the role of absorption processes in the CSM. Using our model which we will describe in this chapter, we will be able to predict this behavior quite well.

\paragraph{Early-time rise: absorption-dominated phase.}
At very early times after explosion the radio-emitting region is still compact, dense,
and \textbf{deeply embedded in the progenitor's wind or immediate circumstellar environment.}
Even though shock acceleration produces relativistic electrons and amplified magnetic
fields promptly, the emerging synchrotron radiation is \emph{heavily absorbed}.
Two absorption mechanisms may be important:

\begin{itemize}
    \item \textbf{Free--free absorption (FFA)} by ionized circumstellar gas.
    For a stellar wind with $\rho \propto r^{-2}$, the free--free optical depth scales as
    \[
        \tau_{\rm ff} \propto \nu^{-2.1} \int n_e^2\,ds,
    \]
    and \textbf{is therefore large at low frequencies and at early times} while the shock is
    still propagating through dense material close to the star.

    \item \textbf{Synchrotron self-absorption (SSA)} within the shocked shell.
    If the emitting region is sufficiently compact and the magnetic field sufficiently strong,
    the synchrotron radiation is reabsorbed by the same electron population that emits it,
    producing an optically thick spectrum $F_\nu \propto \nu^{5/2}$.
\end{itemize}

In most core-collapse supernovae, particularly those exploding into dense red-supergiant winds,
\emph{free--free absorption in the CSM is the dominant effect} during the initial rise.
As the shock expands, the line-of-sight column of ionized material decreases,
and the free--free optical depth drops rapidly.
Since $\tau_{\rm ff} \propto R^{-3}$ for a steady wind, even modest expansion causes
the formerly opaque CSM to become transparent.
This is why the radio lightcurve exhibits a rapid, almost exponential early rise at low frequencies.

\paragraph{Frequency-dependent peak times.}
A defining observational hallmark of radio supernovae is that
\emph{higher frequencies peak earlier and at higher flux}, while
\emph{lower frequencies peak later and more weakly}.
This follows from the strong frequency dependence of the absorption:

\begin{itemize}
    \item For free--free absorption, $\tau_{\rm ff}\propto\nu^{-2.1}$, so the CSM is
    transparent at high $\nu$ long before it is at low $\nu$.
    \item For synchrotron self-absorption, the SSA turnover frequency decreases as the
    shock expands and the magnetic field weakens.
\end{itemize}

Thus each observing band ``turns on'' when $\tau_\nu(t)\approx 1$.
The condition $\tau_\nu(t)=1$ defines a peak time $t_{\rm pk}(\nu)$ that decreases
monotonically with frequency.
This behavior produces the characteristic ``multi-frequency ladder'' often seen
in radio lightcurve plots: high-frequency lightcurves peak within days,
while low-frequency (e.g., 1\,GHz) emission may take weeks or months.

\paragraph{Late-time decline: optically thin synchrotron phase.}
Once the absorbing material becomes optically thin at a given frequency,
the observed flux follows the intrinsic synchrotron luminosity of the shocked gas.
In the optically thin regime,
\[
    F_\nu \propto N_e\,B^{(p+1)/2}\,R^3\,\nu^{-\alpha},
\]
where $N_e$ is the number of relativistic electrons,
$B$ is the post-shock magnetic field strength,
$p$ is the electron energy index, and $\alpha = (p-1)/2$ is the observed spectral index.
Because the shock expands and decelerates with time,
\begin{itemize}
    \item the post-shock energy density declines ($u_{\rm th}\propto R^{-2}$),
    \item the magnetic field weakens ($B\propto R^{-1}$ for adiabatic expansion),
    \item the relativistic electron distribution ages (radiative and adiabatic losses),
\end{itemize}
and thus the synchrotron luminosity drops as a power law in time,
\[
    F_\nu(t) \propto t^{-\beta},
\]
with $\beta$ typically between $0.7$ and $1.5$ depending on the shock dynamics
and microphysical parameters ($\epsilon_e$, $\epsilon_B$, $p$).

\subsubsection{The Synchrotron SED}

At any given epoch in the evolution of a radio supernova, the spectral energy
distribution (SED) is well described by a \emph{broken power law} with a
frequency-dependent turnover set by absorption.  
The high-frequency portion of the spectrum corresponds to \textbf{optically thin}
synchrotron emission from shock-accelerated electrons, while the low-frequency
portion is shaped by either \textbf{synchrotron self-absorption (SSA)} within the shocked
region or \textbf{free--free absorption (FFA) in the circumstellar medium.}

\paragraph{Optically thin regime.}
At sufficiently high frequencies the emission emerges unattenuated from the
shocked shell.  The relativistic electron population established by diffusive
shock acceleration is well approximated by a power-law energy distribution
\[
    N(E)\,dE = K\,E^{-p}\,dE,
\]
with $p \approx 2$ for strong shocks.  
Standard synchrotron theory then yields a flux density
\[
    F_\nu \propto \nu^{-\alpha}, \qquad \alpha = \frac{p-1}{2},
\]
so that for $p\simeq 2$ one obtains the characteristic spectral slope
$\alpha\simeq 0.5$ commonly observed in radio supernovae.
This high-frequency segment directly probes the underlying electron distribution
and the magnetic field in the shocked gas.

\paragraph{Optically thick regime: synchrotron self-absorption.}
At low enough frequencies the synchrotron source becomes optically thick to
its own emission.  
In the regime where SSA dominates, the radiative transfer equation gives
\[
    F_\nu \propto \nu^{5/2},
\]
for a homogeneous source of size $R$ and magnetic field $B$. The SSA turnover frequency $\nu_{\rm SSA}$ is defined implicitly by $\tau_{\rm SSA}(\nu_{\rm SSA}) = 1$; for $\nu < \nu_{\rm SSA}$ the spectrum rises steeply as $\nu^{5/2}$, while for $\nu > \nu_{\rm SSA}$ it transitions to the optically thin $\nu^{-\alpha}$ behavior.
Although SSA can dominate in very compact explosions
(e.g.\ Type~Ib/c supernovae or GRB afterglows), in Type~II events exploding into
dense red-supergiant winds it is often subdominant to free--free absorption.

\paragraph{Optically thick regime: free--free absorption.}
When the supernova expands into a dense ionized circumstellar wind, free--free
absorption in the unshocked CSM is highly effective at suppressing low-frequency
radiation.  In this case the specific intensity is attenuated by a factor
$\exp(-\tau_{\rm ff})$, where
\[
    \tau_{\rm ff} \propto \nu^{-2.1} \int n_e^2\,ds.
\]
Because $\tau_{\rm ff}$ depends strongly on frequency, the SED exhibits an
\emph{exponential} cutoff at low $\nu$ rather than the shallower $\nu^{5/2}$ rise
of SSA.  This distinguishes FFA-dominated spectra from SSA-dominated ones.
The turnover frequency $\nu_{\rm ff}$ decreases rapidly with time as the
shock expands and the absorbing column density declines ($\tau_{\rm ff}\propto R^{-3}$
for a steady wind).

\paragraph{Combined SED structure.}
Putting these pieces together, the synchrotron SED at time $t$ takes the generic form
\[
    F_\nu(t) \;=\;
    \begin{cases}
        F_{\nu,0}(t)\,\mathrm{e}^{-\tau_{\rm ff}(\nu,t)},
            & \text{FFA--dominated regime},\\[6pt]
        F_{\nu,0}(t)\,
        \left( \nu/\nu_{\rm SSA}(t) \right)^{5/2},
            & \text{SSA--dominated regime},\\[6pt]
        F_{\nu,0}(t)\,
        \left( \nu/\nu_{\rm br}(t) \right)^{-\alpha},
            & \text{optically thin regime},
    \end{cases}
\]
where $F_{\nu,0}(t)$ encodes the intrinsic synchrotron luminosity of the shocked shell
and $\nu_{\rm br}(t)$ is the break frequency between the optically thick and
thin regions. As the supernova shock expands, both $\nu_{\rm SSA}$ and $\nu_{\rm ff}$ decrease
monotonically with time, causing the turnover in the SED to migrate to lower
frequencies. This temporal evolution of the broken power law---a steep optically thick
rise transitioning to an optically thin $\nu^{-\alpha}$ tail---is one of the defining
observational signatures of radio supernovae.
