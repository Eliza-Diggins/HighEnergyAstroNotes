\textbf{Gamma--ray bursts (GRBs)} are among the most luminous and enigmatic
transient phenomena in the universe, releasing enormous amounts of energy over
remarkably short timescales.  Their discovery in the late 1960s by the
\emph{Vela} satellites---instruments originally designed to monitor
nuclear tests rather than astrophysical sources---marked the beginning of a
new field in high--energy astrophysics.  The first published detections (in
1973) revealed brief flashes of MeV--energy photons with durations ranging from
milliseconds to minutes.  These early observations already hinted at the rich
complexity of GRB behavior: \textbf{multiple emission pulses}, \textbf{rapid
variability}, and \textbf{intermittent quiescent intervals} all appeared
commonly in the data.

We now refer to this initial flash fof high--energy radiation as the
\textbf{prompt emission}.  It typically peaks in the $0.2$--$1.5\ {\rm MeV}$
range, lasts less than $\sim 100\ {\rm s}$, and carries an astonishing
$\sim 10^{51}\ {\rm erg}$ of energy in gamma rays alone.\footnote{Throughout,
we refer to ``isotropic--equivalent'' energies unless otherwise stated.}  Even
a cursory inspection of BATSE, \emph{Swift}, or \emph{Fermi} light curves shows
that GRB prompt emission is anything but simple: it is highly structured,
highly variable, and often indicative of central--engine activity over the full
duration of the burst.

\section{Prompt Emission from GRBs}

\begin{figure}[!ht]
    \centering
    \includegraphics[width=0.75\linewidth]{Pictures/figures/grb_pulse_gamma.png}
    \caption{A typical GRB featuring a number of rapid $\gamma$-ray pulses. The central pulse is a great example of a FRED-like pulse.}
    \label{fig:grb_counts}
\end{figure}

GRBs are observed primarily in the \textbf{soft $\gamma$--ray band}, with total
durations spanning \textbf{fractions of a second up to several hundred
seconds}.  Early localization studies demonstrated that GRBs are distributed
\textbf{isotropically} across the sky, immediately suggesting an
\emph{extragalactic} origin and ruling out models tied to Galactic structure.
The key observational quantity for describing the prompt phase is the
\textbf{fluence}, the total energy per unit area detected during the burst:

\vspace{10pt}
\begin{definition}[Fluence]
The \textbf{fluence} $F$ of a GRB is the total received energy per unit area,
integrated over the duration of the burst and summed over the detector's
energy band:
\[
F = \int d\nu \;\int dt \;F_\nu(t).
\]
\end{definition}
\vspace{10pt}

Typical fluences span
\[
10^{-4}\ {\rm erg\,cm^{-2}} \;\text{to}\; 10^{-7}\ {\rm erg\,cm^{-2}},
\]
depending on the GRB's distance, intrinsic luminosity, and spectral shape.
From the fluence $F$ and burst duration $\Delta t$, we can estimate the
\textbf{isotropic--equivalent luminosity}
\[
L_{\rm iso}
    \approx 4\pi D^2\,\frac{F}{\Delta t}
    \sim 10^{46}
    \left(\frac{D}{{\rm 1\ Mpc}}\right)^2
    \left(\frac{\Delta t}{1\ {\rm s}}\right)^{-1}
    \left(\frac{F}{10^{-4}\ {\rm erg\,cm^{-2}}}\right)
    \ {\rm erg\,s^{-1}}.
\]
For cosmological distances, these luminosities routinely exceed
$10^{50}\ {\rm erg\,s^{-1}}$, outshining entire galaxies for the duration of
the burst.

Following the prompt phase, every GRB produces a \textbf{multiwavelength
afterglow}, a longer--lived emission component radiated at X--ray, optical,
and radio wavelengths as the relativistic ejecta interact with the surrounding
medium.  Afterglows can persist for weeks, months, or even years, providing
critical clues to the explosion environment, jet geometry, and the underlying
relativistic shock physics.  The next sections explore the physical processes
responsible for both the prompt MeV emission and the broadband
afterglow that follows.

\subsection{Spectroscopy}

\begin{figure}[ht!]
    \centering
    \includegraphics[width=1\linewidth]{Pictures//figures/band_function.png}
    \caption{The spectrum of a GRB during its prompt emission phase featuring a characteristic \textbf{Band function} spectrum peaking around 1 MeV in energy.}
    \label{fig:grb_spectrum}
\end{figure}

The prompt emission of GRBs is characterized by a \textbf{nonthermal} spectrum with a spectral peak typically located at a few hundred~keV.  
In many bursts, the spectrum extends to much higher energies, occasionally reaching the GeV range (as observed, for example, by the \emph{Fermi}/LAT).  
The diversity of observed spectral shapes led to the development of empirical fitting functions, the most successful of which is the \textbf{Band function}, a smoothly broken power law of the form:
\[
N(E) = 
\begin{cases}
N_0 \left(\frac{E}{E_0}\right)^{\alpha} \exp\!\left(-\frac{E}{E_0}\right), & E \le (\alpha-\beta)\,E_0, \\[8pt]
N_0 \left(\frac{(\alpha-\beta)E_0}{E_0}\right)^{\alpha-\beta} 
\exp(\beta-\alpha)\,
\left(\frac{E}{E_0}\right)^{\beta}, & E > (\alpha-\beta)\,E_0.
\end{cases}
\]
Here
\begin{itemize}
    \item $\alpha$ is the low--energy photon index,
    \item $\beta$ is the high--energy photon index,
    \item $E_0$ is an exponential rollover energy,
    \item and the spectral peak in $\nu F_\nu$ occurs at $E_{\rm p} = (2+\alpha)\, E_0$.
\end{itemize}

Despite the diversity of GRB light curves, the \textbf{distribution of peak energies $E_{\rm p}$ is surprisingly narrow,} clustering around a few hundred~keV.  
It remains uncertain whether this narrowness reflects a genuine physical property of the emission mechanism or arises from detector sensitivity and bandpass limitations.  
Approximately $10\%$ of bursts\textbf{ exhibit a very hard high--energy tail with $\beta > -2$, implying that the $\nu F_\nu$ spectrum does not peak within the instrumental window. } 
Conversely, a subset known as \textbf{NHE bursts} (``no high energy'') displays extremely soft spectra, with very steep high--energy indices (large negative $\beta$) and no discernible hard component.

\medskip
An important empirical trend emerges when comparing spectral hardness to burst duration: \textbf{short GRBs tend to be spectrally harder than long GRBs}.  
This hardness--duration correlation reinforces the idea that short and long GRBs reflect distinct physical origins.  

\medskip
Finally, the temporal evolution of the spectrum often reveals additional complexity.  
Individual pulses frequently display \textbf{FRED-like} morphologies (fast rise, exponential decay), and both spectral hardening and softening can occur over the course of the burst.  
Such variability hints at dynamic internal conditions in the relativistic outflow responsible for the prompt emission.

\subsection{Temporal Features}

The temporal behavior of GRB prompt emission is extraordinarily diverse, spanning over five orders of magnitude in duration—from the shortest events at $\sim 10^{-2}\,{\rm s}$ to the longest at $\sim 10^{2}\,{\rm s}$.  
To characterize this duration in a manner robust to instrumental noise, observers define the parameter $t_{90}$: the interval containing the central $90\%$ of the observed fluence, measured between the $5\%$ and $95\%$ cumulative-fluence levels of the light curve.  
This convention avoids the low-flux regions where detector systematics dominate and provides a reasonably uniform measure of burst length across different instruments.

\medskip
A hallmark of GRB prompt emission is its extreme \textbf{variability}.  
Bursts commonly show $\sim 100\%$ flux variations on timescales far shorter than the total event duration, indicating internal dissipation within a highly relativistic outflow.  
Individual emission episodes rarely resemble one another: pulses can be isolated or overlapping, smooth or spiky, and often exhibit pronounced substructure.  
Nevertheless, many pulses share a characteristic asymmetric morphology known as a \textbf{FRED}.

\vspace{6pt}
\begin{definition}[FREDs]
A \emph{fast-rise, exponential-decay (FRED)} pulse is an asymmetric light-curve shape characterized by a rapid increase in flux followed by a more gradual exponential decline.  
Quantitatively, the rise time is typically a factor of $\sim 3$ shorter than the decay time.
FREDs provide a useful phenomenological description of many isolated GRB pulses.
\end{definition}

\medskip
Beyond their basic shape, GRB pulses exhibit a number of systematic temporal–spectral correlations:

\begin{itemize}
    \item \textbf{Spectral lags:} Low-energy emission in a pulse typically peaks \textbf{later than the high-energy emission.}
    These lags are \emph{anti-correlated} with burst luminosity—\textbf{more luminous bursts show longer lags.}
    This \emph{lag–luminosity relation} provides a potential luminosity indicator for cosmological distance estimation.

    \item \textbf{Energy-dependent pulse width:} Lower-energy light curves are broader than their high-energy counterparts.  
    Empirically, the pulse width scales roughly as $W(E) \propto E^{-0.4}$ .

    \item \textbf{Width–symmetry–intensity correlation:} Brighter pulses tend to be more symmetric (i.e., have a smaller decay-to-rise ratio) and exhibit shorter spectral lags.

    \item \textbf{Hardness–intensity correlation:} The instantaneous spectral hardness tracks the instantaneous flux—the spectrum hardens during the rise and softens during the decay.
\end{itemize}

These correlations indicate that the temporal and spectral evolution of GRBs is governed by coherent physical processes inside the jet—likely involving time-dependent particle acceleration, magnetic dissipation, or internal shocks.

\subsubsection{Populations}

Using $t_{90}$ as the classification parameter, GRBs separate into two primary observational classes:
\begin{itemize}
    \item \textbf{Long GRBs (LGRBs):} $t_{90} \gtrsim 2\,{\rm s}$,
    \item \textbf{Short GRBs (SGRBs):} $t_{90} \lesssim 2\,{\rm s}$.
\end{itemize}
This division is empirical, and both populations have broad, overlapping tails.  
Some early studies suggested the presence of a third, ``intermediate'' class with durations $2.5 \lesssim t_{90} \lesssim 7\,{\rm s}$, though subsequent analyses questioned the statistical significance of this claim.

Despite these uncertainties, the bimodal structure in the duration distribution—combined with the hardness–duration correlation—strongly indicates that long and short GRBs arise from distinct astrophysical progenitors, a theme to which we will return in later sections.

\subsection{Astrometry}

Early GRB observations revealed a striking fact: the bursts appear \textbf{isotropically distributed} across the sky.  
No preferred direction, no clustering along the Galactic plane, and no concentration toward the Galactic center were found.  
This immediately raised a fundamental question:  
\begin{center}
\emph{Are GRBs of Galactic origin, or are they extragalactic?}
\end{center}

If \textbf{GRBs were Galactic}, their isotropy would require a spatial distribution extending well into the Galactic halo.  
Because the apparent fluences are modest ($10^{-7}$–$10^{-4}\,{\rm erg\,cm^{-2}}$), a Galactic-halo distance scale ($D\sim 100\ {\rm kpc}$) would imply typical isotropic energies of only
\[
E_{\rm iso} \sim 4\pi D^2 F \sim 10^{40}\,{\rm erg},
\]
entirely reasonable for a variety of Galactic compact-object phenomena.

By contrast, if GRBs are at cosmological distances ($D\sim{\rm Gpc}$), then the same fluences correspond to truly enormous energy releases:
\[
E_{\rm iso} \sim 4\pi D^2 F \sim 10^{51}\,{\rm erg},
\]
comparable to or exceeding the kinetic energy of a supernova.  
Such energies seemed extreme prior to the discovery of afterglows and redshift measurements, and for decades \textbf{the community debated whether GRBs were a local or cosmological population.}
 
Suppose GRBs are standard-candle sources with a constant comoving number density and are distributed in Euclidean space.  
For a source of isotropic luminosity $L$, the observed flux is
\[
S = \frac{L}{4\pi r^2}.
\]
The number of sources within radius $r$ scales as $N(<r) \propto r^3$, and therefore the number of sources brighter than flux $S$ scales as
\[
N(>S) \propto S^{-3/2}.
\]
Thus, an \emph{unbiased} Euclidean distribution produces a clean power law with slope $-3/2$ in a plot of $\log N$ versus $\log S$.

\medskip
However, observations from BATSE demonstrated a clear \textbf{deviation from the $S^{-3/2}$ slope at low fluxes}: the number of faint bursts is significantly \emph{deficient} relative to the Euclidean prediction.  
This deficiency cannot be explained by detector incompleteness alone.  
Instead, it strongly suggests that the GRB population is \emph{not} a local, homogeneous, Euclidean distribution.

Two broad explanations naturally arise:
\vspace{1cm}
\begin{enumerate}
    \item The sources lie at cosmological distances, where the geometry of spacetime, luminosity distance scaling, and cosmic expansion all modify the $\log N$--$\log S$ relation.
    \item The comoving rate density of bursts is not constant with distance (e.g., tracking the cosmic star formation history).
\end{enumerate}
\vspace{1cm}
Either way, the observed departure from $N \propto S^{-3/2}$ is incompatible with a simple population of Galactic-halo sources, and instead points toward a cosmological origin.  
This conclusion was later confirmed definitively through afterglow localization and spectroscopic redshift measurements.

\section{Theory of The Prompt GRB Emission}

Before developing the physical models that power GRB prompt emission, it is
helpful to summarize the essential observational clues:

\begin{itemize}
    \item GRB prompt spectra are \textbf{nonthermal}, typically fit by the empirical
          Band function, with a broken power--law form spanning keV to MeV energies.
    \item GRBs show \textbf{rapid variability}, with characteristic timescales
          ($t_{90}$, $\delta t$) down to milliseconds.
    \item The integrated fluences and peak fluxes imply \textbf{enormous
          isotropic--equivalent luminosities}, often exceeding 
          $10^{51}$--$10^{52}\ {\rm erg~s^{-1}}$.
\end{itemize}

Taken together, these statements imply that GRBs are both \emph{extremely
energetic} and \emph{extremely compact} sources.  As we will now see, this leads
directly to the ``compactness problem,'' the central argument requiring GRBs to
arise from ultra--relativistic outflows.

\subsection{The Compactness Problem of GRBs}

Let us begin from a single, robust observational fact: the prompt GRB spectrum is \textbf{nonthermal}.
A nonthermal spectrum is only produced if the emitting region is effectively
\emph{\textbf{optically thin}}---otherwise repeated scatterings and pair interactions
would force the spectrum toward a thermal shape. As such, we can make the argument that, whatever the source of our radiation, it must be optically thin while satisfying our other features of GRBs.

Thus the emitting region must satisfy
\[
\tau \ll 1,
\qquad
\tau = \int n\,\sigma\,ds,
\]
where $n$ is the \textbf{number density of scatterers} and $\sigma$ is the relevant
interaction cross section. Since we have an energy peaking around 1 MeV, we might presume that the correct interaction model is \textbf{pair-production}, and therefore that we can use $\sigma_T$ as the cross section. Thus, if the emission traverses some characteristic length $L$, then
\[
\tau \sim n \sigma_T L.
\]
We immediately deduce that, for any density $n$, if I decrease $L$, I can get a sufficiently small $\tau$. As such, I'd like to have a constraint on $L$.

Now consider an individual pulse in the prompt light curve with observed width $\delta t$.
The shortest variability timescale places a strict upper bound on the linear size of the region emitting that pulse:
\[
R \lesssim c\,\delta t.
\]
This is a simple consequence of causality: if the source were larger than $c\,\delta t$, different parts of the emitting region could not coordinate their variability on timescales as short as $\delta t$. 
\rmk{This effectively means the mechanisms wouldn't be able to shut off together or turn on together in a coherent fashion.}

Thus, for a pulse with a given variability timescale $\delta t$, we have 
\[
R \sim c\,\delta t,
\]
so
\[
\boxed{
R_{\rm classical} \sim 3\times 10^{7}\;\left(\frac{\delta t}{10^{-3}\;{\rm sec}}\right)\; {\rm cm}.
}
\]

Now, within that $\delta t$, I need to correctly produce a sufficient \textbf{fluence} 
to power the observed GRB. Thus, there is an \textit{isotropic equivalent luminosity} of
\[
\boxed{
L_{\rm iso, classical} \sim 4\pi D^2 \frac{F}{\delta t}
\sim 1.2\times 10^{53}
\left(\frac{D}{\rm 100\; Mpc}\right)^2 
\left(\frac{F}{\rm 1\times 10^{-4} \;erg\;cm^{-2}}\right) 
\left(\frac{\delta t}{\rm 10^{-3}\;s}\right)^{-1} 
\;{\rm erg/s}
}
\]

This luminosity is spread over a region of size $R\sim c\delta t$, meaning that
\[
\boxed{
n_\gamma \sim \frac{L_{\rm iso,classical}}{4\pi (c\,\delta t)^2\, c\,\langle E_\gamma\rangle}
}
\]

For typical values, this becomes
\[
n_\gamma 
\sim 2\times 10^{31}\;
\left(\frac{D}{\rm 100\; Mpc}\right)^2 
\left(\frac{F}{\rm 1\times 10^{-4} \;erg\;cm^{-2}}\right) 
\left(\frac{\delta t}{10^{-3}\;{\rm s}}\right)^{-3} 
\left(\frac{\langle E_\gamma\rangle}{\rm 1\; MeV}\right)^{-1}
\;{\rm cm^{-3}}.
\]

With these values, the optical depth to Thomson scattering becomes
\[
\tau_T \sim n_\gamma\,\sigma_T\,R
        \sim
        \frac{L_{\rm iso}\,\sigma_T}{4\pi c^2\,\delta t\,\langle E_\gamma\rangle}.
\]

Numerically, this evaluates to
\[
\tau_T \sim 10^{11}\,
    \left(\frac{L_{\rm iso}}{10^{51}\ {\rm erg\,s^{-1}}}\right)
    \left(\frac{\delta t}{10^{-3}\ {\rm s}}\right)^{-1}
    \left(\frac{\langle E_\gamma\rangle}{1\ {\rm MeV}}\right)^{-1}.
\]

\textbf{Thus the emitting region should be \emph{extraordinarily opaque}, with optical
depths far exceeding unity.}


This conclusion becomes even stronger when we consider \textbf{photon–photon pair production}.  
Photons above $m_e c^2 = 511~\rm keV$ readily annihilate with photons of comparable energy:
\[
\gamma + \gamma \;\rightarrow\; e^+ + e^-.
\]
Since GRB spectra routinely extend to MeV or even GeV energies, the pair production optical depth
\[
\tau_{\gamma\gamma}
\]
is typically even larger than the Thomson depth.  If $\tau_{\gamma\gamma}\gg1$, then the high--energy photons we observe should be \emph{completely absorbed}, and their energy rapidly converted into an electron–positron pair plasma.
This would in turn thermalize the emission, producing a Wien or blackbody spectrum---in sharp contradiction with observations.

\vspace{8pt}
\begin{ideabox}
The compactness problem: The combination of extreme luminosity and rapid variability implies photon densities so large that $\gamma\gamma$ pair production and Thomson scattering should make GRB sources highly opaque.
Yet we observe broad, nonthermal spectra extending well above the pair production threshold.  This contradiction demands a physical mechanism that \emph{reduces the inferred photon density} and \emph{decreases the optical depth}.
\end{ideabox}
\vspace{8pt}

\subsubsection{Relativistic Motion as the Resolution}

The solution emerges naturally once we consider a relativistically expanding source.  If the emitting region moves toward us with Lorentz factor $\Gamma\gg1$, then several relativistic effects modify the inferred size and
photon density:

\begin{enumerate}
    \item \textbf{Time dilation increases the true size of the source.}  
          A variability timescale $\delta t$ observed by us corresponds to a larger comoving size:
          \[
          R' \sim \Gamma^2 c\,\delta t.
          \]
          This comes from our relativistic case where we have \textbf{both time-dilation}: $\delta t_{\rm events} = \Gamma \delta \tau$, and
          from the \textbf{detection correction}: $\delta t_{\rm correction} = \delta t_{\rm events}/\delta(\theta)$, which provides a factor of $\Gamma^2$.
          
          increasing the volume---and reducing the particle density---by
          factors of $\Gamma^6$. Notably, this changes our fiducial understanding of the emission scale from
          \[
          R_{\rm \gamma} \sim c\delta t\sim 10^7 \left(\frac{\delta t}{10^{-3}\;{\rm s}}\right)\;{\rm cm}, 
          \]
          to
          \[
          R_{\rm \gamma} \sim \Gamma^2 c\delta t\sim 10^{13} \left(\frac{\Gamma}{1000}\right)^2\left(\frac{\delta t}{10^{-3}\;{\rm s}}\right)\;{\rm cm}, 
          \]
          which is more similar to an AU or so.

    \item \textbf{Photon energies are blue--shifted.}  
          The high--energy photons we observe originated at lower comoving
          energies, reducing the fraction of photons above the pair--production
          threshold in the comoving frame.

    \item \textbf{Photon number density is reduced by Lorentz transformation.}  
          The comoving photon density satisfies
          \[
          n'_\gamma \sim n_\gamma/\Gamma,
          \]
          and the relevant path length scales as $R'/\Gamma$, giving an overall
          suppression of optical depth by $\sim \Gamma^{-2}$--$\Gamma^{-6}$,
          depending on geometry.

    \item \textbf{Aberration of angles suppresses head--on $\gamma\gamma$
          collisions.}  
          Photons become beamed into a cone of opening angle $\sim 1/\Gamma$,
          reducing the number of photon pairs with large center--of--mass
          energy and strongly decreasing $\tau_{\gamma\gamma}$.
\end{enumerate}

Taken together, these effects can easily reduce $\tau_{\gamma\gamma}$ from
$\sim10^{15}$ to $\ll 1$ provided that
\[
\Gamma \gtrsim 100.
\]
This number represents a typical lower limit on GRB Lorentz factors derived
from compactness arguments alone.

\vspace{8pt}
\begin{bigidea}
The compactness problem forces the conclusion that the emitting region of a
GRB must be moving ultra-relativistically toward the observer.  Only with
Lorentz factors $\Gamma \sim 100$--$1000$ do the inferred optical depths fall
below unity, allowing high-energy photons to escape unattenuated and preserving
the observed nonthermal spectra.
\end{bigidea}

This realization---that GRBs must involve ultra-relativistic outflows---is one
of the foundational insights in GRB physics and underlies all modern models of
their prompt emission and afterglow.

\subsection{The Need for GRB Jets}

We have resolved the compactness problem by insisting that the outflows
responsible for the prompt GRB emission are ultra--relativistic, with Lorentz
factors $\Gamma \sim 10^2$--$10^3$.  However, this immediately raises a deeper
question:
\begin{center}
\emph{How can any astrophysical engine provide enough energy to accelerate
material to such extreme speeds?}
\end{center}
This question becomes especially pressing if we imagine, even temporarily, that
the GRB emission is \emph{isotropic}: that the gamma rays are radiated equally
in all directions.  As we shall see, this assumption leads to an
energetic catastrophe and forces the conclusion that GRB emission must be
\emph{collimated into a relativistic jet}.


Let us consider the typical observed fluence $F$ of a GRB at a luminosity
distance $D$.  If the emission is isotropic, the total radiated energy in
gamma rays alone is
\[
E_{\rm iso} = 4\pi D^2 F.
\]
Even for modest burst parameters, this number is staggering.  For example,
taking a representative long GRB with
\[
F \sim 10^{-4}\ {\rm erg\,cm^{-2}}, \qquad D \sim 3\ {\rm Gpc},
\]
we find
\[
E_{\rm iso}
    \sim 4\pi (3\times 10^{27}\ {\rm cm})^2
        (10^{-4}\ {\rm erg\,cm^{-2}})
    \sim 10^{53}\ {\rm erg}.
\]
This energy is emitted over a timescale of order seconds to tens of seconds,
giving isotropic--equivalent luminosities
\[
L_{\rm iso} \sim 10^{51} - 10^{52}\ {\rm erg\,s^{-1}},
\]
making GRBs briefly the most luminous objects in the universe.

However, this energy budget is only the \emph{prompt gamma--ray output}.  
If the emitting region is a relativistic outflow with Lorentz factor $\Gamma$,
the gamma--ray energy is only a fraction of the total energy in the flow.
The kinetic energy of the ultra--relativistic baryons (or Poynting flux)
carrying the jet is typically comparable to or larger than the prompt
radiative energy.  Thus, an isotropic fireball would require a true explosion
energy
\[
E_{\rm true} \gtrsim {\rm few}\times 10^{53}\ {\rm erg}.
\]


Such energies far exceed what known astrophysical engines can supply:

\begin{itemize}
    \item A canonical core--collapse supernova has kinetic energy
          \[
          E_{\rm SN} \sim 10^{51}\ {\rm erg},
          \]
          two orders of magnitude too small.
    \item The gravitational binding energy of a neutron star is
          \[
          E_{\rm bind} \sim 3\times 10^{53}\ {\rm erg},
          \]
          but almost all of this escapes as neutrinos, and only
          $\sim 10^{51}$ erg couples to the baryonic ejecta.
    \item Even the rotational energy of a maximally spinning neutron star
          or Kerr black hole,
          \[
          E_{\rm rot} \sim (10^{52} - 10^{53})\ {\rm erg},
          \]
          would need to be converted to gamma--ray radiation with unrealistically
          high efficiency.
\end{itemize}

In other words: \emph{no known astrophysical process can produce an isotropic
$10^{54}\ {\rm erg}$ explosion and simultaneously accelerate material to
$\Gamma \sim 10^2$--$10^3$.}

The resolution is simple and elegant: the emission is not isotropic.
Instead, GRBs produce \textbf{relativistic jets}, with opening half--angle
$\theta_j$.  Radiation from within the jet is strongly beamed: an observer
sees an isotropic--equivalent energy larger than the true energy by the factor
\[
f_b^{-1} = \frac{1}{1 - \cos\theta_j}
         \approx \frac{2}{\theta_j^2}
\qquad (\theta_j \ll 1).
\]
Thus the true energy is
\[
E_{\rm true} = f_b\, E_{\rm iso}
              \approx \frac{\theta_j^2}{2}\, E_{\rm iso}.
\]
For typical observed jet opening angles $\theta_j \sim 0.05$--$0.2$ radians, we obtain
\[
E_{\rm true} \sim 10^{50} - 10^{51}\ {\rm erg},
\]
\emph{fully compatible} with the energy reservoirs of stellar--mass black holes,
magnetars, or compact binaries.

\section{The Fireball Model}

We now turn to the ``fireball'' model of GRB outflows, the earliest and still
one of the most conceptually useful frameworks for understanding how GRBs
accelerate material to ultra--relativistic speeds.  The central idea is that \textbf{a
small region near the central engine is suddenly loaded with an enormous amount
of thermal energy}, primarily in the form of photons and electron--positron
pairs.  This photon--pair plasma initially has a temperature
\[
k_{\rm B} T \gtrsim m_e c^2,
\]
so that \textbf{photon--photon collisions efficiently produce pairs}:
\[
\gamma + \gamma \leftrightarrow e^+ + e^-.
\]
The pair--production cross section is large near threshold (comparable to, but somewhat smaller than, $\sigma_T$), and because the photon number density is extremely high, the fireball is initially \emph{highly opaque} to both
Thomson scattering and pair creation.  As a result, \textbf{all components remain in
near--perfect thermal equilibrium}.  In this regime, the total pressure is dominated by radiation:
\[
P = \frac{1}{3} a T^4
\qquad\Rightarrow\qquad
P = \frac{\rho}{3},
\]
exactly as in the early radiation--dominated universe in cosmology. 
This fireball will be the mechanism by which we accelerate baryons to ultrarelativistic speeds.

\subsection{The Radiation Dominated Phase}

The first phase of the expansion is the \textbf{radiation dominated phase} of the fireball.
During this period, the plasma is \textbf{opaque} to photons: photon–photon pair production and Thomson scattering keep photons, electrons, and positrons in near-perfect thermal equilibrium.
In this regime the energy density is dominated by radiation, and the fluid behaves like a relativistic
gas with equation of state
\[
P = \frac{\varepsilon}{3},
\]
where $\varepsilon$ is the \emph{internal} energy density of the radiation field.

\subsubsection*{The Adiabatic Index of a Radiation Fluid}

For a general ideal fluid, we can write the equation of state in the form
\[
P = (\hat{\gamma}-1)\,u,
\]
Thus, a photon–pair plasma in thermal equilibrium behaves as a fluid with
adiabatic index $4/3$, exactly as in the radiation-dominated era of the early
universe.

In the comoving frame of the fireball, the expansion during this phase is well approximated as \textbf{adiabatic}.  The first law of thermodynamics for a fluid element reads
\[
d(\varepsilon V) + P\,dV = 0,
\]
where $V$ is the comoving volume.  Using $P = (\hat{\gamma}-1)\varepsilon$ and
$\hat{\gamma} = 4/3$, we have
\[
d(\varepsilon V) + (\hat{\gamma}-1)\varepsilon\,dV
= d(\varepsilon V) + \frac{1}{3}\varepsilon\,dV = 0.
\]
Expanding the first term,
\[
V\,d\varepsilon + \varepsilon\,dV + \frac{1}{3}\varepsilon\,dV = 0
\quad\Rightarrow\quad
V\,d\varepsilon + \frac{4}{3}\varepsilon\,dV = 0.
\]
Divide by $V\varepsilon$:
\[
\frac{d\varepsilon}{\varepsilon} + \frac{4}{3}\frac{dV}{V} = 0
\quad\Rightarrow\quad
\frac{d\varepsilon}{\varepsilon} = -\frac{4}{3}\frac{dV}{V}.
\]
Integrating,
\[
\varepsilon \propto V^{-4/3}.
\]
If the fireball expands approximately spherically, then $V \propto R^3$, so
\[
\boxed{
\varepsilon \propto R^{-4}.
}
\]
Since the radiation energy density is related to the temperature by the
blackbody relation
\[
\varepsilon = a T^4,
\]
with $a$ the radiation constant, we obtain
\[
a T^4 \propto R^{-4}
\quad\Rightarrow\quad
\boxed{
T \propto R^{-1}.
}
\]

Thus, during the radiation-dominated, adiabatic expansion phase of the GRB
fireball, the comoving temperature falls inversely with radius, and the energy
density in radiation scales as $R^{-4}$.  The decrease in internal energy
reflects the fact that the radiation is doing $P dV$ work on the baryons,
accelerating them to ultra-relativistic speeds.

Because the plasma is opaque, the photons remain trapped and do work on the 
baryons (even a tiny baryon load is sufficient to absorb the momentum).  
The result is an extremely efficient conversion of internal energy into bulk 
kinetic energy:
\[
\Gamma \propto R,
\]
until the bulk Lorentz factor saturates at a terminal value
$\Gamma_{\rm max} \sim \eta$, where $\eta$ is the dimensionless entropy
(energy per unit baryon rest mass).

Once the temperature drops below $\sim m_e c^2$, the pairs annihilate faster
than they are produced.  The pair density decreases dramatically, the opacity
falls, and the fireball becomes transparent to its own radiation.  At this 
``photospheric'' radius, photons decouple and begin to free--stream.

Beyond this point, the dynamics split into two conceptually distinct phases:

\begin{enumerate}
    \item \textbf{Coasting phase:}  
          If the fireball has already converted most of its internal energy
          into bulk motion before transparency, the Lorentz factor freezes out
          at $\Gamma \simeq \Gamma_{\rm max}$ and the plasma coasts at constant
          velocity.

    \item \textbf{Matter--dominated expansion:}  
          With radiation pressure no longer dominant, the outflow behaves like
          a cold, relativistic wind.  Its density and temperature continue to
          fall as $R^{-2}$ and $R^{-2/3}$, respectively.
\end{enumerate}

Thus, the fireball model divides GRB acceleration into two broad stages:

\begin{description}
    \item[Radiation--dominated acceleration:]  
          The photon--pair plasma expands adiabatically, doing work on the
          baryons and accelerating them to ultra--relativistic speeds.

    \item[Matter--dominated coasting and expansion:]  
          Once transparency is reached and the internal energy is depleted,
          the baryons coast with fixed Lorentz factor.
\end{description}

Although the modern understanding of GRB jets incorporates additional physics
(internal shocks, magnetic dissipation, reconnection), the fireball model
provides a simple and powerful baseline: \emph{a GRB outflow is a 
radiation--dominated explosion that naturally accelerates to extreme Lorentz
factors before becoming transparent}.


\section{Jet Break Models}
