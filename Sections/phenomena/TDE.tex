A \textbf{tidal disruption event} occurs in scenarios where a star's orbit brings it \textbf{within a close enough radius of a black hole to disrupt the star.} The material from the star may then be either partially accreted or fully accreted in a luminous transient event. In this section, we'll discuss the theory and relevant observational context of these events.

\section{TDE Theory}

Before we dig into the observational details of TDEs, we'll first discuss the theory of TDE formation and the relevant scalings and typical equations in play. 

\subsection{Tidal Disruption}

Tidal forces on a star of radius $R_\star$ and mass $M_\star$ become relevant only when their corresponding gravitational field \textbf{varies significantly from one side of the star to the other}. Assuming that the star is originally at radius $r$ from the point mass it is orbiting, the star will then rest in the potential
\[
\Phi = -\frac{GM}{r}.
\]
In such a potential, the center of mass of the star moves in an orbit as described by standard mechanics. The relative force $dF_i$ in direction $x^i$ due to a shift in position $dx^j$ is 
\[
dF_i = \nabla_i \Phi(x + dx^j),
\]
which defines a \emph{tidal tensor} that characterizes the second derivatives of the gravitational potential.
\vspace{10pt}
\begin{definition}[Tidal Tensor]
The \textbf{tidal tensor} describes the differential gravitational acceleration across a finite-sized body in an external potential. It is given by
\begin{equation}
    \label{eq:tidal_tensor}
    \boxed{
    T_{ij} = -\Phi_{;ij} = \frac{\partial^2 \Phi}{\partial x^i\partial x^j}.
    }
\end{equation}
The corresponding \textbf{differential force} due to a shift $\delta {\bf x}$ is 
\[
\delta {\bf F} = \bf{T} \delta {\bf x}.
\]
\par
For a point-mass potential $\Phi = -GM/r$, this becomes
\[
\boxed{
T_{ij} = \frac{GM}{r^3}\left(3 - \delta_{ij}\right).
}
\]
The eigenvalues of $T_{ij}$ quantify the degree of stretching or compression along different directions in space.
\end{definition}
\vspace{10pt}
If the differential tidal acceleration across the star \textbf{exceeds its own self-gravity}, the star can no longer maintain hydrostatic equilibrium and is torn apart. At the surface of a star of radius $R_\star$, the tidal tensor tells us that the differential force is
\[
{\bf F}_{\rm tidal} = {\bf F}(r + R_\star) - {\bf F}(r) 
\simeq \left(\frac{d{\bf F}}{dr}\right) R_\star
= \left(\frac{GM}{r^3}\right) R_\star,
\]
which corresponds to a differential acceleration
\[
a_{\rm tidal} \sim \frac{GM R_\star}{r^3}.
\]
This is the characteristic acceleration difference felt between the near and far sides of the star due to the black hole’s gravitational field.

The star’s own self-gravity at its surface, which provides the restoring acceleration maintaining hydrostatic balance, is
\[
a_{\rm self} \sim \frac{G M_\star}{R_\star^2}.
\]
When $a_{\rm tidal} \gtrsim a_{\rm self}$, the black hole’s tidal forces dominate over the star’s self-gravity, and the star is disrupted. Setting the two equal defines the \textbf{tidal radius} $r_t$:
\[
\frac{G M R_\star}{r_t^3} = \frac{G M_\star}{R_\star^2}
\quad \Longrightarrow \quad
\boxed{r_t = R_\star \left(\frac{M}{M_\star}\right)^{1/3}}.
\]
This simple scaling captures the essential physics: \textbf{more massive black holes have larger tidal spheres of influence}, while more compact (smaller $R_\star$) or massive ($M_\star$) stars are harder to disrupt. In practice, the exact disruption condition depends on the internal structure of the star, the encounter geometry, and relativistic corrections for very massive black holes.
\par
It is worth looking at this scaling in a fiducial context. For stars of approximately solar mass and radius and black holes of mass around $10^6\;{\rm M_\odot}$, the relationship
\[
\begin{aligned}
    r_t &\approx 7 \times 10^{12} \left(\frac{R_\star}{R_\odot}\right) \left(\frac{M}{10^6\;{\rm M_\odot}}\right)^{1/3} \left(\frac{M_\star}{M_\odot}\right)^{-1/3} \; {\rm cm}\\
    &\approx 0.5 \left(\frac{R_\star}{R_\odot}\right) \left(\frac{M}{10^6\;{\rm M_\odot}}\right)^{1/3} \left(\frac{M_\star}{M_\odot}\right)^{-1/3} \; {\rm AU}
\end{aligned}
\]
Notably, the \textbf{radius of the star} is a more powerful scaling in this equation than the mass ratio. As such, it is much easier to disrupt large, diffuse stars.
\par
Another feature of these systems which is worth being aware of is the $r_t$ is \textbf{not always outside the event horizon}. For a Schwarzchild black hole, 
\[
r_s = \frac{2GM}{c^2} \implies \frac{r_t}{r_s} =\frac{c^2}{2G}\; R_\star M_\star^{-1/3} M^{-2/3}.
\]
For relevant scalings,
\[
\frac{r_t}{r_s} \approx 5 \left(\frac{R_\star}{R_\odot}\right) \left(\frac{M_\star}{M_\odot}\right)^{-1/3}\left(\frac{M}{10^7\;{\rm M_\odot}}\right)^{-2/3}.
\]
And so for black holes larger than about $10^8$, \textbf{tidal disruption simply cannot occur.}
\par
If the \textbf{pericenter} of the stellar orbit passes within $r_t$ of the black hole, then disruption will occur. The degree of disruption depends sensitively on the ratio $\beta \equiv r_t / r_p$, known as the \textbf{penetration factor}. For $\beta \lesssim 1$, the star \textbf{may only be partially stripped}, losing a small fraction of its envelope. For $\beta \gtrsim 1$, the \textbf{star is fully disrupted}, and its material is stretched into a long, thin stream by tidal forces.

\subsection{Accretion onto the Black Hole}

Assuming the \textbf{ballistic approximation}, the star is completely torn apart as it crosses the tidal radius $r_t$, and the stellar debris moves thereafter in the gravitational field of the black hole \textbf{without significant hydrodynamic interaction}. Each fluid element inherits approximately the velocity of the stellar center of mass at the moment of disruption but originates from a slightly different position within the star. This positional offset \textbf{leads to a spread in the specific orbital energy of the debris.}

The specific orbital energy of a test particle in the gravitational potential of the black hole is
\[
\epsilon = \frac{1}{2}v^2 - \frac{GM}{R}.
\]
Expanding about the tidal radius $r_t$, we find that the differential in energy across the stellar diameter is approximately
\[
\Delta \epsilon \approx \left.\frac{d\Phi}{dr}\right|_{r_t} R_\star = \frac{GM}{r_t^2} R_\star = \epsilon_\star \left(\frac{M}{M_\star}\right)^{1/3},
\]
where $\epsilon_\star = GM_\star / R_\star$ is the characteristic binding energy per unit mass of the star. Thus, the debris inherits a roughly uniform distribution of energies between
\[
-\Delta \epsilon \leq \epsilon \leq +\Delta \epsilon.
\]

Because the original stellar orbit is nearly parabolic ($\epsilon \approx 0$), half of the debris ends up with $\epsilon < 0$ and remains gravitationally bound to the black hole, while the other half with $\epsilon > 0$ becomes unbound and escapes on hyperbolic trajectories. \textbf{Importantly, both bound and unbound debris streams \emph{initially pass through the same pericenter} $r_p \approx r_t$}, because they are all launched from approximately the same point in space at the moment of disruption. What differs between them is their \emph{orbital energy}, and hence their semimajor axis and eccentricity.

\paragraph{Eccentricities of the Bound Debris.}
For the bound material, the semimajor axis of a given debris element is related to its specific energy by
\[
a = -\frac{GM}{2\epsilon}.
\]
Since the most tightly bound debris has $\epsilon = -\Delta \epsilon$, its semimajor axis is
\[
a_{\rm min} = \frac{GM}{2\Delta \epsilon} \approx \frac{r_t^2}{2R_\star} \approx \frac{1}{2}R_\star \left(\frac{M}{M_\star}\right)^{2/3}.
\]
The corresponding orbital eccentricity is determined by the relation between pericenter and semimajor axis:
\[
e = 1 - \frac{r_p}{a}.
\]
Substituting $r_p \approx r_t$ and $a = a_{\rm min}$ gives
\[
e_{\rm min} = 1 - 2\frac{R_\star}{r_t}
\simeq 1 - 2\left(\frac{M_\star}{M}\right)^{1/3}.
\]
For a solar-type star disrupted by a $10^6\,M_\odot$ black hole, this yields $e_{\rm min} \approx 0.9998$: \textbf{the bound orbits are extremely eccentric. } Immediately after disruption, the bound debris occupies a family of highly eccentric orbits with (nearly) common pericenter $r_p \simeq r_t$ but different semimajor axes $a(\epsilon) = -GM/(2\epsilon)$. In the absence of dissipation, these orbits would not circularize. However, \textbf{apsidal precession} produced by general relativity (and, to a lesser degree, pressure gradients) rotates the orbit between pericenter passages, causing the outgoing stream to intersect the incoming stream. The resulting \textbf{shocks} dissipate orbital energy at (approximately) fixed specific angular momentum, enabling the debris to settle into a compact, near-circular flow.

Conservation of specific angular momentum, $j \simeq \sqrt{2GM r_p}$ for a (nearly) parabolic encounter, implies that the circular orbit with the same $j$ has
\begin{equation}
\label{eq:R_circ}
\boxed{\,R_{\rm circ} \;=\; \frac{j^2}{GM} \;\simeq\; 2\,r_p \;\approx\; 2\,r_t.\,}
\end{equation}
Thus, \textbf{efficient dissipation drives the debris toward a \emph{circularization radius} of order a few times the tidal radius.} If precession is weak or the stream is thick/cool, self-intersection may occur farther out and circularization can be delayed; conversely, strong precession produces deep, prompt intersections near $r_p$ and rapid disk formation.
\par
For a Schwarzschild black hole, the advance of pericenter per orbit for a test particle of semimajor axis $a$ and eccentricity $e$ is
\begin{equation}
\Delta\varpi \;=\; \frac{6\pi GM}{c^2 a(1-e^2)}.
\end{equation}
Using $r_p = a(1-e)$ and $1-e^2 \simeq 2(1-e)$ for $e\to 1$, we obtain the very convenient near-parabolic form
\begin{equation}
\label{eq:precession_rp}
\boxed{\,\Delta\varpi \;\simeq\; \frac{3\pi r_s}{r_p} \;=\; \frac{6\pi GM}{c^2 r_p}.\,}
\end{equation}
Even a modest precession angle ($\gtrsim$ a few degrees) is sufficient to bend the outgoing stream into the inbound trajectory on the next pass, ensuring a strong self-intersection shock. For Kerr black holes, additional \emph{nodal} (Lense–Thirring) precession tilts the orbital plane and can either aid or delay intersection, depending on spin magnitude and orientation.

Label the specific binding energy of a debris element by $\epsilon<0$. Its Kepler period is
\[
P(\epsilon) \;=\; \frac{2\pi a^{3/2}}{\sqrt{GM}}
\;=\; \frac{2\pi}{\sqrt{GM}}\left(\frac{GM}{2|\epsilon|}\right)^{3/2}
\;=\; \frac{\pi GM}{\sqrt{2}\,|\epsilon|^{3/2}}.
\]
The most bound debris has $|\epsilon|=\Delta\epsilon$, and thus returns to pericenter after
\begin{equation}
\label{eq:tmin}
\boxed{\,t_{\min} \;\equiv\; P(\Delta\epsilon) \;=\; \frac{\pi GM}{\sqrt{2}\,\Delta\epsilon^{3/2}}.\,}
\end{equation}
Using $\Delta\epsilon \simeq (GM/r_t^2)R_\star$ and $r_t = R_\star(M/M_\star)^{1/3}$ yields the standard scaling
\[
t_{\min} \;\sim\; \frac{\pi}{\sqrt{2}}\,
\frac{GM}{\left[(GM/R_\star^2)\,R_\star\,(M/M_\star)^{2/3}\right]^{3/2}}
\;\propto\; M^{1/2}\,R_\star^{3/2}\,M_\star^{-1}.
\]
Numerically, for a solar-type star and $M=10^6\,M_\odot$,
\[
t_{\min} \sim \text{a few weeks}.
\]
Because the mass in debris is (to leading order) uniformly distributed in specific energy near $\epsilon=0$, i.e.\ $dM/d\epsilon \approx \text{const}$, and because $t\propto |\epsilon|^{-3/2}$, one finds the classic \textbf{fallback rate}:
\begin{equation}
\label{eq:fallback}
\boxed{\,\dot M_{\rm fb}(t) \;=\; \frac{1}{3}\,\frac{M_\star}{t_{\min}}
\left(\frac{t}{t_{\min}}\right)^{-5/3} \quad (t \gtrsim t_{\min}).\,}
\end{equation}
The normalization (the factor $1/3$) is conventional and encodes the near-uniform $dM/d\epsilon$ assumption; detailed stellar structure, partial disruptions ($\beta\lesssim 1$), and relativistic effects can modify both the peak and early-time behavior.

\section{Observational Properties}

\subsection{X-Ray Emission}

- A thermal soft component is sometimes present emerging from the region close to $r_g$ of the black hole. This is emission from the material that is actively accreting.

- Hard X-rays can also be observed from both synchrotron and comptomization. 

- TDEs may have one, both, or neither of these components.

\subsection{Optical \& UV Emission}

- Much less constrained. Seems to emerge from much larger radii $r \gg r_g$. This could be reprocessing of the X-rays (Roth+16, Dai+18, etc.) The reprocessing material could be winds (Metzger+16)  or due to collision induced outflows CIO (Lu+20).

In the later case, we anticipate that the emission trace the fallback rate of $t^{-5/3}$, which is commonly observed.

Piran+15 suggests that this emission is not reprocessing but simply direct emission from stream interaction.

Loeb97, Metzger22 introduced a new idea where the optical emission is from a quasi-spherical pressure supported envelope. This predicts $t^{-3/2}$.

There is also the emerge

\subsection{Radio Emission}

Largely driven by synchrotron emission in material much further than $r_g$. Some TDEs appear to have on-axis jets. (some of these are relativistic, some are not).

Some TDEs are not detected immediately in the radio.

In some recently reported scenarios, there is radio emission years after the discovery of the TDE in x-ray or optical.

