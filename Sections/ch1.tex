High energy astrophysics is, to first order, the study of violent / extreme processes which produce high energy light (X-ray and $\gamma$-ray). In the modern view, this includes things like gravitational radiation, high energy particles, etc. In general, high energy phenomena need to bring together two relevant processes: \textbf{energy generation} and \textbf{energy dissipation}. Generally, high energy processes are driven by a conversion to kinetic energy and then a secondary dissipation into thermal energy.

\section{What Makes High Energy Different}

There are a number of things that distinguish the high energy regime of astrophysics from lower energy regimes. A few relevant ones are
\vspace{0.5cm}
\begin{enumerate}
    \item High energy photons are $\sim 1\; \rm{MeV}$, so for the same energy budget, you can expect to get \textbf{only a fraction as many photons}.
    \item The high energy sky is \textbf{quite}, there are relatively few environments in which you can generate high energy photons. These tend to be point sources,
    although exceptions do exist.
    \item These photons often require some specialized techniques to be stopped.
\end{enumerate}

We also get emission from a large number of different sources, all of which have very different physics and scale very differently:

\begin{table}[htp!]
\centering
\renewcommand{\arraystretch}{1.3}
\begin{tabular}{|p{3cm}|p{3cm}|p{8cm}|}
\hline
\textbf{Energy Range} & \textbf{Thermal Equivalent} & \textbf{Processes Probed} \\
\hline
$E < 10\;\mathrm{keV}$ 
& $T < 10^{8}\;\mathrm{K}$ 
& 
\begin{itemize}
  \item K, L shell line emission ($T \lesssim 5\times 10^{7}\,$K), e.g. iron K$\alpha$ line at 6.4 keV  
  \item Bremsstrahlung (galaxy clusters, SNRs, AGN coronae)  
  \item Blackbody emission
\end{itemize} \\
\hline
$10\;\mathrm{keV} - 10\;\mathrm{MeV}$ 
& typically all atoms fully ionized, electrons stripped 
& 
\begin{itemize}
  \item Isomeric transitions from metastable nuclei  
  \item $\gamma$-ray lines that are nucleosynthetic signatures (e.g. $^{60}$Co $\rightarrow$ $^{60}$Ni + $e^-$ + $\bar{\nu}_e$ + $\gamma$, $E_\gamma \approx 1.1\;\mathrm{MeV}$; SN remnants)  
  \item Radioactive decay: excited nucleus $\rightarrow$ ground state + $\gamma$-ray
\end{itemize} \\
\hline
$E \sim 511\;\mathrm{keV} = m_e c^2$ 
& $T \sim 10^{9}\;\mathrm{K},\; \Gamma \sim 1$ 
& 
\begin{itemize}
  \item $e^+ e^-$ annihilation lines: $e^+ + e^- \rightarrow \gamma + \gamma$
\end{itemize} \\
\hline
$140\;\mathrm{MeV} - 10\;\mathrm{GeV}$ 
& $T \sim 10^{12}\;\mathrm{K},\; \Gamma \sim 1000$ (up to $10^{6}$) 
& 
\begin{itemize}
  \item Pion mass scale (strong force $\rightarrow$ hadrons)  
  \item Baryons ($uud$, $udd$; 3 quarks)  
  \item Mesons ($u\bar{d}$, etc.; 2 quarks, includes pions)
\end{itemize} \\
\hline
$E > 10\;\mathrm{GeV}$ 
& $\Gamma > 10^{6}$ 
& 
\begin{itemize}
  \item Non-thermal processes:  
  \begin{itemize}
    \item Inverse Compton scattering  
    \item Particle acceleration
  \end{itemize}
\end{itemize} \\
\hline
\end{tabular}
\caption{Physics probed at different photon energies.}
\end{table}


\section{Where Does the Energy Come From}

There are 3 major processes which produce the energy that drives high energy phenomena:
\vspace{0.5cm}
\begin{enumerate}
    \item \textbf{Gravitational Energy:}  
    In high-energy astrophysics, gravitation is the dominant source of kinetic energy. Material falling in a gravitational potential well is accelerated to high velocities; this energy is then thermalized and radiated.  

    For a uniform-density sphere, the total gravitational binding energy is
    \[
        U = -\frac{3}{5}\frac{GM^2}{R}.
    \]
    The characteristic \textit{specific energy} (energy per unit mass) is therefore
    \[
        \tilde{u} \sim \frac{GM}{R}.
    \]
    In cgs units, this scales as
    \[
        \tilde{u} \approx 1.9 \times 10^{15}
        \left(\frac{M}{M_\odot}\right)
        \left(\frac{R}{R_\odot}\right)^{-1}
        \;\mathrm{\frac{erg}{g}}.
    \]
    \begin{remark}
        One thing to keep in mind is that of these three, gravitation is the only one that is decidedly variable. For instance, in a stellar system, we have around $10^{15}\;{\rm erg \;g^{-1}}$, but in a highly compact system $(GM/R \sim 1)$, we have instead $10^{21}\;{\rm erg\;g^{-1}}$.
    \end{remark}
    \item \textbf{Nuclear Energy:}  
    Nuclear fusion reactions in stellar cores release $\sim 1\;\mathrm{MeV}$ per baryon, corresponding to a specific energy of order
    \[
        \tilde{u}_{\rm nuc} \sim 10^{18}\;\mathrm{\frac{erg}{g}}.
    \]

    \item \textbf{Chemical Energy:}  
    Chemical reactions typically release energy on the $\sim 1\;\mathrm{eV}$ per baryon scale, giving a specific energy of order
    \[
        \tilde{u}_{\rm chem} \sim 10^{13}\;\mathrm{\frac{erg}{g}}.
    \]
\end{enumerate}

\section{How is the Energy Dissipated?}

Regardless of the initial source, high energy phenomena leave us with many particles moving at extreme velocities. To produce the observable high-energy radiation, this kinetic energy must be \textbf{thermalized} or otherwise converted into photon emission. There are three major dissipation channels:

\begin{itemize}
    \item \textbf{Shocks:}  
    When supersonic flows collide, they form shock fronts. Shocks provide an efficient way to convert bulk kinetic energy into random particle motion. In astrophysics, strong shocks are ubiquitous --- e.g.\ supernova remnants, accretion flows, and jets --- and they often accelerate particles to non-thermal distributions (Fermi acceleration).

    \item \textbf{Viscous Dissipation:}  
    Even in the absence of strong shocks, turbulent or shearing flows can gradually convert bulk kinetic energy into thermal energy through viscosity. In astrophysical plasmas the ``viscosity'' is often anomalous, mediated by small-scale instabilities and wave–particle interactions (rather than molecular collisions). This mechanism is critical in accretion disks.

    \item \textbf{Magnetic Fields:}  
    Magnetized plasmas store and redistribute energy through magnetic tension and reconnection. Magnetic reconnection events rapidly convert magnetic energy into particle kinetic energy and heat, often producing non-thermal particle populations (e.g.\ solar flares, pulsar magnetospheres). Synchrotron and inverse-Compton processes then radiate this energy efficiently.
\end{itemize}
